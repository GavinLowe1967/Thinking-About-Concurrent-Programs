\section{Example: An Exchanger}
\label{sec:exchanger}

In this section, we consider how to use a client-server mechanism to solve a
synchronisation problem.  

An \emph{exchanger} is a concurrent object that
allows pairs of threads to exchange values: each thread should pass in a
value, and receive the other thread's value back, via an operation
\begin{scala}
  def exchange(x: A): A 
\end{scala}
(where |A| is a polymorphic type parameter of the object).  This exchange
represents a \emph{synchronisation}: the two executions of |exchange| must
necessarily overlap in time.

A straightforward implementation of an exchanger using a server is in
Figure~\ref{fig:exchanger}.  The client creates a reply channel, sends its
value and the reply channel to the server, and waits to receive the other
thread's value on its reply channel.  The server repeatedly receives two
requests, and passes each value to the other thread.

%%%%%

\begin{figure}
\begin{scala}
class Exchanger[A]{
  /** Reply channels, for the server to return results to clients. */
  private type ReplyChan = OnePlaceBuffChan[A]

  /** Channel from clients to the server. */
  private val toServer = new SyncChan[(A, ReplyChan)]

  /** Exchange x with another thread. */
  def exchange(x: A): A = {
    val c = new ReplyChan; toServer!(x, c); c?()
  }

  /** The server thread. */
  private def server = thread("Exchanger"){
    repeat{
      val (x1, c1) = toServer?(); val (x2, c2) = toServer?()
      c1!x2; c2!x1
    }
  }

  fork(server)

  /** Shut down the server. */
  def shutdown() = toServer.endOfStream()
}
\end{scala}
\caption{An exchanger, using a server.}
\label{fig:exchanger}
\end{figure}

%%%%%

\begin{instruction}
Make sure you understand the details of the exchanger.
\end{instruction}

%%%%%

We now consider testing.  The property we need to check is that if
thread~$t_1$ receives thread~$t_2$'s value, then $t_2$ receives $t_1$'s value.
Note that this necessarily implies that the two operation executions overlap
in time: if, say, $t_1$ returned before $t_2$ called the operation, there
would be no way for $t_1$ to obtain $t_2$'s value.  

We can test this property by arranging for each thread to store the value it
receives, and subsequently trying to pair up the threads that exchanged.  It
is possible to come up with an algorithm to find such a pairing of threads, if
one exists.
However, if two threads pass in the same value, there might be multiple
candidates for pairing, which complicates the algorithm.

We can make things easier by arranging for threads to pass in distinct values.
This approach is sound because the implementation is \emph{data independent}:
each data value is passed in as an input (as a parameter of |exchange|),
passed between threads, stored in variables, and output (as the returned value
from |exchange|); but no operation is
performed on it that depends on its actual value.  This means that if there
were an incorrect behaviour that didn't satisfy our requirements, there would
also be an incorrect behaviour where all the inputs were replaced by distinct
values.

A function to perform a single test is below; this can be executed many times.
Each test runs |n| threads with an exchanger, where |n| is a random even
number.  In fact, for convenience, we arrange that each thread submits its
identity to the exchanger.  It then stores the value it receives back in the
array |results|, indexed by its identity.  The correctness condition then is
that for each~|i|, if thread~|i| received~|x|, then $\sm x \ne \sm i$ (so
thread~|i| exchanged with a \emph{different} thread), and thread~|x|
received~|i| (so those two threads exchanged with each other); this is easily
checked.
%
\begin{scala}
  def doTest = {
    val n = 2*scala.util.Random.nextInt(10); val results = new Array[Int](n)
    val exchanger = new Exchanger[Int]
    def worker(me: Int) = thread(s"worker($me)"){ 
      val x = exchanger.exchange(me); results(me) = x 
    }
    run(|| (for(i <- 0 until n) yield worker(i)))
    for(i <- 0 until n){ val x = results(i);  assert(x != i && results(x) == i) }
    exchanger.shutdown()
  }
\end{scala}
%

One detail concerning this testing strategy is that each thread performed just
a single exchange.  It is tempting to arrange for a worker to perform multiple
exchanges, perhaps submitting a different value each time.  However, this would
lead to the possibility of the testing system deadlocking: it could reach a
state where one thread has two exchanges still to make, but all the other
threads have terminated.  When each thread performs just a single exchange, we
avoid this possibility of deadlock. 

\begin{instruction}
Study the details of the |doTest| function.
\end{instruction}

%%%%%%%%%%%%%%%%%%%%%%%%%%%%%%%%%%%%%%%%%%%%%%%%%%%%%%%

\section{Example: A Filter Channel}
\label{sec:filter-chan}

We now consider another example of a synchronisation object.  A \emph{filter
  channel} acts much like a standard synchronous channel, except a receiver
can specify a property that it requires the value it receives to satisfy.
Thus a filter channel implements the following trait.
%
\begin{scala}
trait FilterChanT[A]{
  /** Synchronously send £x£. */
  def send(x: A): Unit

  /** Synchronously receive a value that satisfies £p£. */
  def receive(p: A => Boolean): A 
}
\end{scala}
%
Executions of |send(x)| and |receive(p)| can synchronise with each other
if and only if |p(x)|; in this case, |receive(p)| returns~|x|.  Each operation
blocks until it is able to synchronise with a suitable operation of the other
kind. 

%%%%%

\begin{figure}
\begin{scala}
class FilterChan[A] extends FilterChanT[A]{
  /** Channel senders uses to send to the server. */
  private val fromSender = new SyncChan[(A, Chan[Unit])]

  def send(x: A) = { 
    val replyChan = new OnePlaceBuffChan[Unit]
    fromSender!(x, replyChan); replyChan?()
  }

  /** Channel receivers use to send to the server. */
  private val fromReceiver = new SyncChan[(A => Boolean, Chan[A])]
  
  def receive(p: A => Boolean): A = {
    val replyChan = new OnePlaceBuffChan[A]
    fromReceiver!(p, replyChan); replyChan?()
  }

  private val shutdownChan = new SyncChan[Unit]

  def shutdown() = shutdownChan!() 

  private def server = thread("FilterChan"){ ... } // See Figure £\ref{fig:FilterChan-server}£.

  fork(server)
}
\end{scala}
\caption{Outline of the {\scalashape FilterChan} class.}
\label{fig:FilterChan}
\end{figure}

%%%%%%%%%%

Some of the code for the |FilterChan| class is in
Figure~\ref{fig:FilterChan}.  The |send(x)| and |receive(p)| send,
respectively, |x| and~|p|, together with a reply channel to the server, and
then wait to receive on the reply channel.  As normal, the |shutdown|
operation sends a simple signal to the server. 

\pagebreak[2]

The server is in Figure~\ref{fig:FilterChan-server}.  This maintains queues
|pendingSends| and |pendingRecs| of pending requests from senders and
receivers.  An invariant is that no entries in these queues are compatible:
for each |(x,cs)| in |pendingSends| and for each |(p,cr)| in |pendingRecs|,\,
|p(x)| does not hold.  However, the code does not necessarily maintain the
order in the queues: a queue is simply a convenient collection class to use
here, allowing easy iteration. 

%%%%%

\begin{figure}
\begin{scala}
  private def server = thread("FilterChan"){
    // Queue of pending requests from senders.
    val pendingSends = new Queue[(A, Chan[Unit])]
    // Queue of pending requests from receivers.
    val pendingRecs = new Queue[(A => Boolean, Chan[A])]
    var isShutdown = false
    serve(!isShutdown)(
      fromSender =?=> { case (x,cs) => 
        // Traverse £pendingRecs£ to see if a request matches.
        var i = 0; var done = false; val len = pendingRecs.length
        while(i < len && !done){
          val (p,cr) = pendingRecs.dequeue()
          if (p(x)){ cr!x; cs!(); done = true }
          else{ pendingRecs.enqueue((p, cr)); i += 1 }
        }
        if(!done) pendingSends.enqueue((x,cs))
      }
      | fromReceiver =?=> { case (p,cr) =>
        // Traverse £pendingSends£ to see if a value matches.
        var i = 0; var done = false; val len = pendingSends.length
        while(i < len && !done){
          val (x,cs) = pendingSends.dequeue()
          if(p(x)){ cr!x; cs!(); done = true }
          else{ pendingSends.enqueue((x, cs)); i += 1 }
        }
        if(!done) pendingRecs.enqueue((p,cr))
      }
      | shutdownChan =?=> { _ => isShutdown = true }
    )
    fromSender.close(); fromReceiver.close()
    for((_,c) <- pendingSends) c.endOfStream()
    for((_,c) <- pendingRecs) c.endOfStream()
  }
\end{scala}
\caption{The server for the {\scalashape FilterChan} class.}
\label{fig:FilterChan-server}
\end{figure}

When the server receives a pair |(x,cs)| from a sender, it traverses
|pendingRecs| to try to find a pair |(p,cr)| such that |p(x)|.  If is finds a
match, it sends |x| to the receiver on~|cr|, and signals to the sender
on~|cs|.  If it doesn't find a match, it adds |(x,cs)| to |pendingSends|.

The server acts very similarly when it receives a message from a receiver
(there is an argument for factoring out the common code into a separate
function).  

When the server receives a message to shut down, it sets the |isShutdown|
flag, exits the loop, and closes all the channels, including those in its
queues, to signal to any waiting clients.

\begin{instruction}
Make sure you understand the implementation of |FilterChan|.
\end{instruction}

%%%%%

We now consider how to test the implementation.  We need to
ensure that the filter channel is \emph{synchronous}, in particular that the
sender does not return before the corresponding receiver calls the operation:
in the implementation, the use of a reply channel was designed to ensure this.
We will therefore use the technique of logging, as we did with the resource
server.  We will run a number of senders and receivers that call the
operations, but that write into the log both before and after the operation.
This will allow us to identify which operation executions overlap, and then we
can try to pair up corresponding sends and receives.  
%
This pairing up of corresponding operation executions will be easier if we
arrange for each |send| operation to send a different value.  


%%%%%

\begin{figure}
\begin{scala}
  trait LogEvent // Events to include in the log. 
  case class BeginSend(id: Int, x: Int) extends LogEvent
  case class EndSend(id: Int) extends LogEvent
  case class BeginReceive(id: Int) extends LogEvent
  case class EndReceive(id: Int, x: Int) extends LogEvent
  type Log1 = Log[LogEvent]

  def sender(me: Int, chan: FilterChanT[Int], log: Log1) 
  = thread(s"sender($me)"){
    for(x <- me*iters until (me+1)*iters){
      log.add(me, BeginSend(me, x)); chan.send(x); log.add(me, EndSend(me))
    }
  }

  def receiver(me: Int, chan: FilterChanT[Int], log: Log1) 
  = thread(s"receiver($me)"){
    def p(x: Int) = x%n == me
    val logId = me+n // Identity for logging purposes.
    for(x <- 0 until iters){
      log.add(logId, BeginReceive(me)); val x = chan.receive(p)
      log.add(logId, EndReceive(me, x))
    }
  }

  def checkLog(events: Array[LogEvent]): Boolean = ... // See Figure £\ref{fig:FilterChan-test2}£.

  def doTest = {
    val chan: FilterChanT[Int] = new FilterChan[Int]; val log = new Log1(2*n)
    val senders = || (for(i <- 0 until n) yield(sender(i, chan, log)))
    val receivers = || (for(i <- 0 until n) yield (receiver(i, chan, log)))
    run(senders || receivers)
    assert(checkLog(log.get))
    chan.shutdown()
  }
\end{scala}
\caption{Most of the code for testing the {\scalashape FilterChan}.}
\label{fig:FilterChan-test1}
\end{figure}

%%%%%

Most of the testing code is in Figure~\ref{fig:FilterChan-test1}. 
%%  (It is
%% trivial to adapt it to test other implementations of the |FilterChanT| trait.)
We log using events of type |LogEvent|, with subclasses corresponding to the
begin and end of |send|s and |receive|s; the events include the identity of
the thread in question (to allow pairing of corresponding |Begin| and |End|
events), and, in some cases, the value sent or received (to allow matching a
|send| to the corresponding |receive|).

We run |n| senders and |n| receivers.  The sender with identity |me| (with
$\sm{me} \in \interval{0}{\sm n}$) sends |iters| values, namely
$\interval{\sm{me} \times \sm{iters}}{(\sm{me}+1) \times \sm{iters}}$, so all
sent values are distinct.  The receiver with identity~|me| (again with
$\sm{me} \in \interval{0}{\sm n}$) is willing to receive values~|x| such that
$\sm{x}\%\sm{n} = \sm{me}$, so the receivers collectively are willing to
receive all the values.

Each thread writes suitable events into the log before calling the operation,
and after it returns.  Recall that each thread includes its identity in
operations to add events to the log.  In order to ensure that the senders and
receivers use different identities for this purpose, we add |n| to the
identity of each receiver.

The log slightly overestimates the time period during which a thread is
actually in the operation, because there is a slight delay between logging the
|Begin| event and calling the operation, and a slight delay between the
operation returning and logging the |End| event.  However, this discrepancy is
not a problem as it will not lead to false positives.  If the implementation
is faulty, this might lead to false negatives; but this is unlikely, and other
runs are likely to find the error.

The |doTest| operation performs a single test.  It creates the |FilterChan|
and log, and runs the senders and receivers.  It then uses the |checkLog|
function, given in Figure~\ref{fig:FilterChan-test2} and described below, to
test whether the log represents a valid execution, where senders and receivers
correctly synchronise and exchange values.


%%%%%

\begin{figure}
\begin{scala}
  def checkLog(events: Array[LogEvent]): Boolean = {
    def giveError(i: Int) = 
      println(s"\nUnmatched End event at index $i: "+events(i)+"\n"+
        events.mkString("\n"))
    val Out = 0; val Unmatched = 1; val Matched = 2
    val senderState = Array.fill(n)(Out); val receiverState = Array.fill(n)(Out)
    val senderValue = new Array[Int](n); val receiverValue = new Array[Int](n)
    for(i <- 0 until events.length) events(i) match{
      case BeginSend(s, x) => 
        assert(senderState(s) == Out)
        senderState(s) = Unmatched; senderValue(s) = x
      case BeginReceive(r) =>
        assert(receiverState(r) == Out); receiverState(r) = Unmatched
        var j = i+1; var done = false
        while(!done) events(j) match{
          case EndReceive(r1, x) if r == r1 => receiverValue(r) = x; done = true
          case _ => j += 1
        }
      case EndSend(s) =>
        if(senderState(s) == Matched) senderState(s) = Out
        else{
          assert(senderState(s)==Unmatched); var r = 0; val x = senderValue(s)
          while(r < n && senderState(s) == Unmatched)
            if(receiverState(r) == Unmatched && receiverValue(r) == x){
              senderState(s) = Out; receiverState(r) = Matched }
            else r += 1 // End of £while£.
          if(senderState(s) == Unmatched){ giveError(i); return false }
        }
      case EndReceive(r, x) =>
        if(receiverState(r) == Matched) receiverState(r) = Out
        else{
          assert(receiverState(r) == Unmatched); var s = 0
          while(s < n && receiverState(r) == Unmatched)
            if(senderState(s) == Unmatched && senderValue(s) == x){
              senderState(s) = Matched; receiverState(r) = Out }
            else s += 1 // End of £while£.
          if(receiverState(r) == Unmatched){ giveError(i); return false }
        }
    } // End of £for£/£match£.
    true
  }
\end{scala}
\caption{The {\scalashape checkLog} function for testing a {\scalashape
    FilterChan}.} 
\label{fig:FilterChan-test2}
\end{figure}

The |checkLog| function traverses the log, trying to match corresponding
|send|s and |receive|s.  It keeps track, in arrays |sender|\-|State| and
|receiverState|, whether each thread is currently outside an operation call
(value~|Out|), inside an operation call but has not yet been matched with
another operation (value~|Unmatched|), or inside an operation call and matched
with another operation call (value~|Matched|).  It also keeps track, in array
|senderValue| of the value being sent by a sender, and in array
|receiverValue| of the value that a |receive| operation will return.

When the traversal encounters a |Begin| operation, it updates values
appropriately.  In the case of a |BeginReceive|, it searches forward through
the log to find the corresponding |EndReceive| event, to record the value the
operation will return.

When the traversal encounters an |EndSend| event, if this operation has not
already been matched, it searches through the receivers to try to find one
that is also unmatched and that receives the corresponding value.  If so, it
records that |receive| operation as matched, and the |send| operation as
complete.  If it fails to find a match, this represents an error: it prints
appropriate information to help with debugging, and returns false.  When the
traversal encounters an |EndReceive| event, it acts in a very similar way.

\begin{instruction}
Make sure you understand how the testing program works, and, in particular,
the |checkLog| function.
\end{instruction}

This testing program was designed so that each execution of |send| could be
matched by a \emph{unique} execution of |receive|, and vice versa.  This made
identifying the matching easier.  In fact, this is not necessary: we can allow
non-unique matchings of individual operation executions, and then apply a
standard graph algorithm.  Consider a graph where each node is an execution of
|send| or |receive|, and there is an edge between |send(x)| and |receive(p)|
if those executions overlapped in the log and |p(x)|, i.e., those two
executions are candidates for being matched.  This graph is \emph{bipartite},
i.e., the nodes can be partitioned intro two subsets (here the |send|s and
|receive|s) such that every edge connects two nodes from different subsets.
We then want to ask whether there is a subset of the edges that includes every
node exactly once: in graph-theoretic terms, this is known as a \emph{perfect
  matching}.  There are standard algorithms for finding a perfect matching, if
one exists, for example the Ford-Fulkerson method~\cite{ford-fulkerson, CLRS}.
We can therefore use such an algorithm to decide whether the log represents a
valid execution.  This technique can be applied more generally for testing
synchronisation objects.

%% that are binary (i.e.,~each synchronisation is
%% between precisely two operation executions), stateless (i.e., synchronisations
%% are independent, and don't depend on synchronisations that have happened
%% previously), and heterogeneous (i.e.,~each synchronisation is between two
%% executions of \emph{different} operations, unlike with the exchanger where
%% each synchronisation is between two executions of the same operation).
