At present, we can write definitions of threads that try to send or receive on
a \emph{single} channel at a time.  However, it's often useful for a thread to
be able to try to send or receive on either of two or more channels: the
\emph{alternation}, or \emph{alt}, construct does that for us.
%
In this chapter, we will study alternation: we start by describing the
basic syntax of alternation in SCL, and then use it in examples.

%%%%%

The following construct defines a thread that tries to receive from any one of
several different in-ports:
%
\begin{scala}
  alt( in£$_1$£ =?=> {f£$_1$£} | ... | in£$_n$£ =?=> {f£$_n$£} )
\end{scala}
%
This waits until one of the in-ports \SCALA{in}$_1$, \ldots, \SCALA{in}$_n$ is
ready to communicate, reads a value~$v$ from the port, and applies the
relevant function |f|$_i$ to~$v$.  If |in|$_i$ is an in-port passing data of
type~|A|, then $\sm f_i$ must be a function that takes an argument of
type~|A|.

In such constructs, it is common to define each function using a
\emph{function literal}: the notation |x => c| (pronounced ``|x|~maps
to~|c|'') represents a function that, given argument~|x|, performs~|c|.  Thus,
if a branch |in =?=> { x => c }| receives a value~|v|, then the argument~|x|
of the function gets bound to~|v|, and |c| is performed.

Here's a very simple example. 
%
\begin{scala}
  alt(
    c1 =?=> { x => println(s"$x received on c1") }
    | c2 =?=> { x => println(s"$x received on c2") }
  )
\end{scala}
%
In each case, the variable~|x| gets bound to the value received, and then the
|println| statement is performed. 

%%%%%

The following thread repeatedly receives a value from one of two input ports,
tags the value, and outputs it.
%
%\begin{mysamepage}
%  def tagger[T](l: ??[T], r: ??[T], out: !![(Int, T)]) = thread{
\begin{scala}
  repeat{
    alt ( l =?=> { x => out!(0, x) } | r =?=> { x => out!(1, x) } )
  }
\end{scala}
%\end{mysamepage}

%
%Exercise: design a corresponding de-tagger.  \framebox{??}

%%%%% \heading{Guards}

It's sometimes useful to specify that a particular in-port should be
considered only if some boolean condition, or \emph{guard}, is true.  In the
construct%
%
\begin{scala}
  alt( guard£$_1$£ && in£$_1$£ =?=> {f£$_1$£} | ... | guard£$_n$£ && in£$_n$£ =?=> {f£$_n$£} )
\end{scala}
%
a communication on each |in|$_i$ is possible only if the corresponding
|guard|$_i$ is true.  Thus, \SCALA{in =?=> ...} is equivalent to \SCALA{true
  && in =?=> ...}.  Each guard is evaluated once, and should not have side
effects.

If a guard evaluates to true and the in-port is open, we say that the
corresponding branch is \emph{feasible}.  If no branch is feasible, the alt
throws an |AltAbort| exception (a subclass of \SCALA{Stopped}): this
corresponds to the case where the alt will never be able to communicate.

If a branch is feasible and the in-port is available for communication
(i.e.,~another thread is trying to send on the channel), then we say that the
branch is \emph{ready}.
%
The alt waits until a branch is ready, and receives from the in-port.  If
several are ready, it chooses between them.  If all the branches subsequently
become infeasible (because of channels being closed), the alt throws an
|AltAbort| exception.

%%%%% \heading{\scalashape serve}

It is very common to use an \SCALA{alt} inside a \SCALA{repeat}.  Consider the
construct
%
\begin{scala}
  repeat{ 
    alt( g£$_1$£ && in£$_1$£ =?=> {f£$_1$£} | ... | g£$_n$£ && in£$_n$£ =?=> {f£$_n$£} ) 
  }
\end{scala}
%
Suppose no branch of the alt is feasible, that is, for every branch, either
the guard is false or the port is closed.  Then the alt will throw an
|AltAbort| exception, which the |repeat| will catch.  Thus the above construct
will repeatedly execute the alt until all branches become infeasible, at which
point it terminates cleanly.  

% (or one of the |f|$_i$ throws a |Stopped| exception).

Note that the above construct evaluates each guard expression~|g|$_i$ and each
port expression |in|$_i$ on each iteration.

%%%%% \heading{\scalashape serve}

The construct 
%
\begin{scala}
  serve( g£$_1$£ && in£$_1$£ =?=> {f£$_1$£} | ... | g£$_n$£ && in£$_n$£ =?=> {f£$_n$£} )
\end{scala}
%
is very similar to the previous |repeat{ alt(...) }| construct, but with two
differences.
%
One difference is that the earlier construct creates a new alternation object
(from class |ox.scl.channel.Alt|) on each iteration, whereas the |serve|
creates a single alternation object which is used repeatedly.  This gives a
small gain in efficiency. 

%%%%% \heading{Fairness}

The implementation of |alt| initially tests whether the branches are ready in
the order given, selecting the first one that is ready (if any).  In the
|repeat{alt(...)}| construct, a new |Alt| object is created for each
iteration, so if the first branch is repeatedly ready, it will be repeatedly
selected, and the other branches will be starved.
%
By contrast, the |serve| aims to be fair.  It uses the same |Alt| on each
iteration.  It remembers which branch was selected on the previous iteration,
and tests whether branches are ready starting from the following one (looping
round).
%
This means that if a particular branch is continuously ready, and the |serve|
performs enough iterations, then that branch will eventually be selected.  We
say that the |serve| is \emph{fair} to each branch.

%% Both the |serve| and the |repeat{alt(...)}| constructs evaluate each guard
%% expression~|g|$_i$ and each port expression |in|$_i$ on each iteration.

%%%%%

Here's the tagger again:
%
\begin{scala}
  serve( l =?=> { x => out!(0, x) } | r =?=> { x => out!(1, x) } )
\end{scala}
%
This is fair to its two input ports: once one becomes ready, the
|serve| will receive from it after at most one communication on the other
port.
%
The |serve| loop terminates when either both |l| and~|r| are closed, or |out|
is closed. 

Henceforth, we will use the word ``alternation'' to mean either an |alt| or
|serve| construct. 


%%%%% \heading{About fairness}

Fairness crops up in a number of scenarios in concurrent programming.
%
A~typical fairness property is that if a particular option is continuously
logically possible, then it eventually happens.  Fairness for an alternation
means that if a particular branch is continuously ready, and the alternation
performs enough iterations, then eventually that branch is selected.

Note that ``fair'' doesn't necessarily mean equal shares: a construct that
chooses option~$A$ 99\% of the time, and chooses option~$B$ 1\% of the time is
fair, despite not giving equal shares.

Fairness is psychologically attractive: certainly, being fair to other people
in everyday life is a good thing.  However, it's not always appropriate in a
concurrent program.  In some scenarios, achieving fairness has a performance
overhead (although it's negligible in the case of a |serve|).  And sometimes
fairness can be achieved only by a more complicated program.  You should
consider whether or not fairness is desirable.  Often you can achieve higher
throughput, and a simpler program, without fairness.

%%%%%  Generalised

There are generalised versions of the alternation operators that take an
arbitrary sequence of branches~|bs|, denoted \SCALA{alt( \|(bs) )} and
\SCALA{serve( \|(bs) )}.  For example, given an array |ins| of in-ports, we can
form a generalisation of the tagger as follows.  It is built from a sequence
of branches that is generated using a |for| expression.
% 
\begin{scala}
  serve( | (for (i <- 0 until ins.length) yield ins(i) =?=> { x => out!(i, x) }) )
\end{scala}
%
The branches are indexed by the variable~|i|; we will sometimes refer to an
alternation like this as \emph{indexed}.  The branch for index~|i| is willing
to input on |ins(i)|; if it receives~|x|, it outputs the pair |(i, x)|.  This
|serve| is fair to each in-port: if a particular in-port is continually ready,
it will be selected within at most |ins.length| iterations.  The construct
terminates when all the ports in |ins| are closed, or |out| is closed.

There is a restriction on the use of alternations.  A port may not be
simultaneously feasible in two alternations; in other words, a port may not be
shared between alternations.  This is a design decision that makes the
implementation easier and slightly more efficient.  It does not seem to be
problematic in practice.  The implementation detects such misuses, and throws
an exception.  However, a port may be shared between an alternation and a
thread performing a simple receive via the ``|?|'' operation.
