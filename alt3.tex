\section{Avoiding Deadlocks}
\label{sec:avoid-deadlocks}

\def\waitingFor{\vdash}

The deadlock in the Dining Philosophers example corresponds to a cycle of
waiting.  Each philosopher is trying to pick up their right-hand fork, and is
waiting for that fork.  Each fork is waiting for their right-hand philosopher
to put it down.  Let's write $t_1 \waitingFor t_2$ to signify that thread
$t_1$ is blocked, waiting for thread~$t_2$.  Then we have the following cycle
of waiting:
\[\mstyle
\sm{phil}(0) \waitingFor \sm{fork}(4) \waitingFor \sm{phil}(4) \waitingFor
  \sm{fork}(3) \waitingFor \ldots \waitingFor \sm{fork}(0) \waitingFor
  \sm{phil}(0).
\]
This corresponds to an anti-clockwise cycle in
Figure~\ref{fig:dining-philosophers-2}.

In fact, deadlocks correspond to cycles of waiting more generally.  Consider a
system based on message passing.  Suppose that for every channel~$c$, there
is some thread that sometimes sends on~$c$, and some thread that sometimes
receives on~$c$ (or else the system is badly configured).  Consider a
deadlocked state, and suppose no thread has terminated or is performing an
infinite amount of internal computation.  Then necessarily every thread is
trying to send or receive on a channel.  Pick a thread $t_1$.  It is trying to
send or receive on some channel~$c_1$, so necessarily there is some
thread~$t_2$ that sometimes receives or sends (respectively) on~$c_1$; hence
$t_1$ is waiting for~$t_2$, i.e.~$t_1 \waitingFor t_2$.  By the same argument,
there is some thread $t_3$ such that $t_2 \waitingFor t_3$.  Continuing in
this way, we can construct an infinite sequence of threads such that each is
waiting on the next.  However, there are necessarily finitely many threads, so
this infinite sequence must contain the same thread twice, and so contains a
waiting cycle.

If the system uses an alt, then a thread may be waiting for either of two or
more threads.  For example, in the dining philosophers example, a fork in its
initial state is waiting for either its left- or right-hand philosopher
(although there is no deadlocked system state corresponding to this state for
the fork).  In such cases, a deadlocked state may contain multiple waiting
cycles, and breaking any one of them would remove the deadlock.  A similar
fact is true when a channel is shared by several receivers, so a sender might
be waiting for any one of those receivers; and similarly when a channel is
shared by several senders.

There are a couple of ways of adapting the dining philosophers example so as
to remove the possibility of deadlock.  Each involves removing the possibility
of a waiting cycle.  Exercise~\ref{ex:diningPhils} asks you to implement
these. 

Note that whenever a philosopher is holding a fork but trying to drop it, that
action is never blocked.  Thus a philosopher that is waiting must be trying to
pick up a fork, but that fork is held by another philosopher.  That fork,
then, is waiting for the latter philosopher to drop it.  And the latter
philosopher must be waiting to pick up the next fork.  Continuing in this way,
we see that any waiting cycle must involve \emph{all} the philosophers and
forks. 

One way to avoid deadlocks is to introduce a ``butler'' thread that ensures
that there are never more than four philosophers seated.  We argue by
contradiction that there is then no deadlocked state.  Suppose otherwise.  Not
all philosophers can be seated, by design.  Without loss of generality,
suppose philosopher~$0$ is not currently seated.  Then necessarily that
philosopher cannot be holding fork~$0$ or fork~$4$.  But this then removes the
possibility of a waiting cycle, because neither of those forks is waiting for
philosopher~$0$.  This contradicts the observation in the previous paragraph
that any waiting cycle most involve all the threads.

In the earlier version of the dining philosophers example, all philosophers
picked up their left fork before their right fork.  Another way to avoid the
deadlock is for some (but not all) of the philosophers to pick up their right
fork first.  If that is the case, there must be an adjacent pair of
philosophers neither of whom picks up their shared fork first.  Without loss
of generality, suppose philosopher~$0$ picks up fork~$4$ before fork~$0$, and
philosopher~$1$ picks up fork~$1$ before fork~$0$.  This then removes the
possibility of a waiting cycle.  If either philosopher~$0$ or philosopher~$1$
holds fork~$0$ then they have necessarily already picked up their other fork,
and so can drop a fork.  Alternatively, if neither philosopher holds fork~$0$,
then that fork cannot be part of a waiting cycle.

Another way to avoid deadlocks is to detect that a potential deadlock has been
reached, and to react to remove that possibility, by backtracking.  For
example, if a philosopher holds one fork, but is unable to pick up their
second fork, they could drop their first fork, and try again later (this is
probably how real philosophers would act).

Within SCL, this could be achieved using a timeout.  Each channel has an
operation
%
\begin{scala}
  def sendWithin(millis: Long)(x: A): Boolean 
\end{scala}
%
This tries to send~|x| for up to |millis| milliseconds; but if no receiver is
ready within that time (or no space is available, in the case of a buffered
channel), it times out.  The operation returns a |Boolean| to indicate whether
the send was successful.  Likewise there is an operation
%
\begin{scala}
  def receiveWithin(millis: Long): Option[A] =
\end{scala}
%
that tries to receive a value for up to |millis| milliseconds.  The operation
returns a result of type |Option[A]| (see Scala box~\ref{sb:option-type}): if it
successfully receives a value~|x|, it returns~|Some(x)|; otherwise it
returns~|None| to indicate a timeout.  There are also operations
|sendWithinNanos| and |receiveWithinNanos| where the time is given in
nanoseconds.

The last part of Exercise~\ref{ex:diningPhils} asks you to implement this
technique of timing out, dropping the fork, and trying again later.

%%%%%%%%%%%%%%%%%%%%%%%%%%%%%%%%%%%%%%%%%%%%%%%%%%%%%%%

\section{Alts with Out-ports}

So far, we have only seen alts that provide a choice between in-ports.
However, it is also possible to use an out-port in an alt, using the following
syntax for a branch:
%
\begin{scala}
  bool && out =!=> { expression }
\end{scala}
%
If the (optional) boolean guard |bool| is true, when the out-port |out| is
ready for communication, |expression| is evaluated and the result sent.
%
Sometimes it's necessary to do something \emph{after} the value has been sent,
using a continuation, with the following syntax.
%
\begin{scala}
   bool && outport =!=> { expression } ==> { command }
\end{scala}

Here's another definition for the |tee| function, which produces
a thread that inputs on one in-port, and outputs on two out-ports, in either
order. 
%
%\begin{mysamepage}
\begin{scala}
  def tee[T](in: ??[T], out1: !![T], out2: !![T]) = thread{
    repeat{ 
      val v = in?()
      alt( out1 =!=> { v } ==> { out2!v } | out2 =!=> { v } ==> { out1!v } )
    }
  }
\end{scala}
%
This sends~|v| on whichever out-port is ready first, and then sends~|v| on the
other out-port.
%\end{mysamepage}

%%%%%% A two-place buffer

In-ports and out-ports can be mixed within an \SCALA{alt}.
%
The following code copies data from |in| to |out|, and can hold up to two
pieces of data at a time: it is a two-place buffer. 
%
\begin{scala}  
  def buff2[T](in: ??[T], out: !![T]): ThreadGroup = {
    def empty(): Unit = { val x = in?(); full(x) }
    def full(x: T): Unit = {
      alt( out =!=> { x } ==> { empty() } | in =?=> { y => out!x; full(y) } )
    }
    thread{ attempt{empty()}{} }
  }
\end{scala}
%%   /** Two place buffer. */
%%   def buff2[T](in: ??[T], out: !![T]): Unit = {
%%     val x = in?(); buff2A(in, out, x)
%%   }
%%   /** Two place buffer holding £x£. */
%%   def buff2A[T](in: ??[T], out: !![T], x: T): Unit = {
%%     alt(
%%       out =!=> { x } ==> { buff2(in, out) }
%%       | in =?=> { y => out!x; buff2A(in, out, y) }
%%     )
%%   }  
%% \end{scala}
%
This runs a thread that executes |empty()|, catching |Stopped| exceptions.
When it  holds a value~|x|, in state |full(x)|, it can either
output~|x|, or input a new value~|y|, at which point it is full so must
output~|x|, after which it is just holding~|y|.


Here's an alternative definition.  The variable \SCALA{empty} records whether
the buffer is empty.  When $\sm{empty} = \sm{false}$,\, \SCALA{x} stores the
next value to be output.  (The expression |null.asInstanceOf[T]| gives the
null value of type~|T|: zero for numeric types, or the null reference for
reference types.  This is a convenient way to initialise a variable of a
polymorphic type.)
%
%\begin{mysamepage}
\begin{scala}
  def buff2Alt[T](in: ??[T], out: !![T]) = thread{
    var x: T = null.asInstanceOf[T]  // Contents, possibly invalid.
    var empty = true // Is the buffer empty?
    serve(
      !empty && out =!=> { empty = true; x }
      | empty && in =?=> { v => x = v; empty = false }
      | !empty && in =?=> { v => out!x; x = v }
    )
  }
\end{scala}
%\end{mysamepage}
%
\noindent
In the first branch, the code \SCALA{\{ empty = true; x \}} is an expression
whose value is~|x|, but which has the side effect of setting |empty| to |true|. 
The last two branches could be merged, and an \SCALA{if} statement used. 
%
\begin{scala}
    | in =?=> { v => if(empty){ x = v; empty = false } else { out!x; x = v } }
\end{scala}

%% --- Moved to exercise
%% Here's another example, namely a thread that receives data on |in|, and
%% outputs it on |out|.  It acts as a buffer, internally holding data in a
%% |Queue| from the Scala Application Programming Interface.  It is equivalent to
%% an |UnboundedBuffChan| (although the latter doesn't use a separate thread
%% internally).
%% %
%% \begin{scala}
%%   def buffer[A](in: ??[A], out: !![A]) = thread("buffer"){
%%     val q = new scala.collection.mutable.Queue[A]
%%     serve(
%%       in =?=> { x => q.enqueue(x) }
%%       | q.nonEmpty && out =!=> { q.dequeue() }
%%     )
%%     in.close(); out.endOfStream()
%%   }
%% \end{scala}
%% %
%% Note that in the second branch, it is important that the alternation evaluates
%% the expression |q.dequeue()| only \emph{after} another thread has committed to
%% communicating on~|out|: otherwise the dequeued value would be lost if no
%% communication takes place. 

%%%%%%%%%%%%%%%%%%%%%%%%%%%%%%%%%%%%%%%%%%%%%%%%%%%%%%%

As with in-ports, an out-port may not be simultaneously feasible in two
alternations.  Further, both ports of a channel may not simultaneously be
feasible in alternations, i.e.~you can't have one alternation trying to send
on a channel, and another alternation trying to receive on the same channel.
The implementation detects violations of these two restrictions, and throws an
exception.
