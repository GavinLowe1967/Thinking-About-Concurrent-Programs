\chapter{Synchronous Data Parallel Programming}
\label{chap:dataParallel}

%%%%%

In this chapter we will study a particular style of data parallel programming
where the threads proceed \emph{synchronously}.  The program proceeds in
\emph{rounds}.  In each round, each thread performs some computation.
However, at the end of each round, each thread needs to wait for the other
threads to finish the round before they all proceed to the next round: this
requires a global synchronisation.

%%%%%

Typically, each thread will operate on one section of the data, but  may
need to read data updated by other threads.  The synchronisation can be used
to ensure that one thread obtains the updates to the data made by other
threads on the previous round.

The data can be distributed between threads by two different techniques.
%
\begin{itemize}
\item
By sending messages; this works well when each piece of data has to be
passed to only a few other threads;

\item
By writing to shared variables.
\end{itemize}

%These algorithms are sometimes known as heart-beat algorithms.

%%%%%

Applications include: image processing, where different threads operate on
different parts of the image; solving differential equations, for example in
weather forecasting or fluid dynamics, where different threads operate on
different areas; matrix calculations, where different threads operate on
different parts of the matrix.

%%%%%


The global synchronisation at the end of each round is sometimes known as a
\emph{barrier synchronisation}, since it represents a barrier than no thread
may pass until all have reached that point.

Suppose we have |p| threads with identities $\interval{\sm 0}{\sm p}$.  Then a
suitable barrier synchronisation object may be created in SCL by:
%
\begin{scala}
  val barrier = new Barrier(p)
\end{scala}
%
A thread with identity~|me| performs the barrier synchronisation by executing%
%
\begin{scala}
  barrier.sync(me)
\end{scala}
%
No call to \SCALA{sync} will return until all \SCALA{p} threads have called
it. 

%%%%%

\begin{figure}
\begin{scala}
class ServerBarrier(p: Int){
  private val arrive = new SyncChan[Unit]
  private val leave = new SyncChan[Unit]

  def sync(me: Int) = { arrive!(); leave?() }

  private def server = thread{
    while(true){
      for(i <- 0 until p) arrive?()
      for(i <- 0 until p) leave!()
    }
  }

  fork(server)
}
\end{scala}
\caption{A simple implementation of a barrier synchronisation.}
\label{fig:serverBarrier}
\end{figure}

Figure~\ref{fig:serverBarrier} gives a possible implementation of a barrier
using a server.  The server waits to receive a message on channel |arrive|
from each of the |p|~threads before sending then a message on channel~|leave|,
telling them they can continue.

The above implementation means that each synchronisation takes time $O(\sm p)$,
assuming all the threads call |sync| at the same time.  In fact, the
implementation in SCL is more sophisticated, and allows each synchronisation
to take time $O(\log \sm p)$.  Exercise~\ref{ex:log-barrier} asks you to
implement a barrier with this property.

One important feature of a barrier synchronisation is that it also ensures
cache consistency between different threads.  If one thread writes to a shared
variable before a synchronisation, and another thread reads that variable
after the synchronisation, then the read is guaranteed to see the effects of
the write.  Further, compiler optimisations are not allowed to break this
property.  

%%%%%%%%%%%%%%%%%%%%%%%%%%%%%%%%%%%%%%%%%%%%%%%%%%%%%%%%%%%%

\section{Example: simulating astronomical bodies}

In this section, we will implement a concurrent simulation of astronomical
bodies, such as stars, planets and moons, moving relative to one another under
gravity.  Each concurrent worker will be responsible for updating the state of
some of the bodies in each step of the simulation.  However, each worker will
need to read the states of all of the bodies.  We will use barrier
synchronisations to coordinate the threads and to avoid races.

We will need to record the position and velocity of each astronomical body.
Each of these is a vector in three-dimensional space.  The support class
|Vector|, in Figure~\ref{fig:vector}, models such vectors.

%%%%%

\begin{figure}
\begin{scala}
/** A vector in three-dimensional space. */
case class Vector(x: Double, y: Double, z: Double){
  /** The length of this vector. */
  def length = Math.sqrt(x*x + y*y + z*z)

  /** This plus £v£. */
  def + (v: Vector) = Vector(x+v.x, y+v.y, z+v.z)

  /** The minus £v£. */
  def - (v: Vector) = Vector(x-v.x, y-v.y, z-v.z)

  /** This times £s£. */
  def * (s: Double) = Vector(x*s, y*s, z*s)
}

object Gravity{
  /** Gravitational constant. */
  val G = 6.6743E-11

  /** The time for one step of the simulation. */
  val Timestep = 100_000
}
\end{scala}
\caption{The class {\scalashape Vector} and the object {\scalashape Gravity}.}
\label{fig:vector}\label{fig:Gravity}
\end{figure}

%%%%%

Given two bodies, of mass $m_1$ and $m_2$ at a distance~$d$ from each other,
each exerts a force on the other of magnitude $G \times m_1 \times m_2 / d^2$,
where $G \approx 6.6743 \times 10^{-11}$ is the gravitational constant.  These
are attractive forces: the force on each body is directly towards the other.
The force on each body gives an acceperation equal to the force divided by the
body's mass.

The object |Gravity| in Figure~\ref{fig:Gravity} defines the
gravitational constant~|G|, and also the time (in seconds) represented by each
step of the simulation. 

%%%%%

\begin{figure}
\begin{scala}
/** A single astronomical body, with mass £mass£, initial position £position0£,
  * and initial velocity £velocity0£. */
class AstronomicalBody(val mass: Double, position0: Vector, velocity0: Vector){
  /** The body's current position. */
  private var position = position0

  /** The body's current velocity. */
  private var velocity = velocity0

  /** Update the velocity of this based on the gravitational attraction from
    * £other£. */
  def updateVel(other: AstronomicalBody) = {
    val towards = other.position-position // Vector to £other£.
    val d = towards.length // Distance to £other£.
    val force = Gravity.G*mass*other.mass/(d*d) // Force on this.
    val dv = towards*(force/mass/d*Gravity.Timestep) // Change in velocity.
    velocity += dv
  }

  /** Move this for one timestep.  Return the new position. */
  def move() = { position += velocity*Gravity.Timestep; position }
}
\end{scala}
\caption{The class {\scalashape
    AstronomicalBody}.} 
\label{fig:AstronomicalBody}
\end{figure}

%%%%%

Each object of the class |AstronomicalBody| represents an astronomical body.
The object records the body's mass, position and velocity.  The operation
|updateVel(other)| updates the velocity of the body based on gravitational
attraction from |other|, for one step of the simulation.  The operation |move|
updates the position of the body based on its current velocity, for one step
of the simulation, and returns the new position.

Figure~\ref{fig:SequentialSimulation} gives code for a sequential simulation.
The parameter |bodies| of the class represents the astronomical bodies to
simulate.  At each step, the velocity of each body is updated based on the
position of each other body, and then the position of each body is updated.
The simulation returns an array giving the bodies' positions at each
timestep. 

%%%%%

\begin{figure}
\begin{scala}
/** A sequential simulation for bodies. */
class SequentialSimulation(bodies: Array[AstronomicalBody]){
  private val n = bodies.length

  /** Simulate for steps steps, returning an array of the bodies' positions at
    * each timestep. */
  def apply(steps: Int): Array[Array[Vector]] = {
    val result = Array.ofDim[Vector](steps,bodies.length)
    for(step <- 0 until steps){
      for(i <- 0 until n; j <- 0 until n; if i != j) 
        bodies(i).updateVel(bodies(j))
      for(i <- 0 until n) result(step)(i) = bodies(i).move()
    }
    result
  }
}
\end{scala}
\caption{A sequential simulation}
\label{fig:SequentialSimulation}
\end{figure}

%%%%%

\begin{instruction}
Study the implementation of the sequential simulation.
\end{instruction}

We now consider how to parallelise the simulation.  We will use a number of
workers, each of which is responsible for updating some of the astronomical
bodies.  However, we need to keep the workers synchronised: we will use a
barrier synchronisation at the end of each step, to ensure that all the
workers are on the same step at the same time.  

Further, we need to avoid race conditions.  The worker for a particular
body~|b| updates |b|'s position within the |move| operation.  However, a
different worker will read that position when it executes |b1.updateVel(b)| on
one of its own bodies~|b1|.  This is a race.  In particular, if our
implementation allows these operations to happen in either order, the results
will be nondeterministic: the update to |b1| might depend on either the
previous or the new position of~|b|.

To avoid such a race, we arrange for a barrier synchronisation after all the
threads have finished the calls to |updateVel|, but before any call to
|move|.  That means that each call to |updateVel| will depend on the previous
positions of the bodies, as in the sequential program. 

The resulting concurrent simulation is in
Figure~\ref{fig:ConcurrentSimulation}.  The threads perform two barrier
synchronisations on each round, splitting the round into two subrounds.  In
the first subround, threads read the positions of bodies, and update their own
bodies' velocities.  In the second subround, threads update only their own
bodies' positions.  Thus there are no races. 

%%%%%

\begin{figure}
\begin{scala}
/** A concurrent simulation for bodies. */
class ConcurrentSimulation(bodies: Array[AstronomicalBody]){
  private val n = bodies.length

  /** Simulate for steps steps, using £numWorkers£ workers, returning an array of
    * the bodies' positions at each timestep. */
  def apply(steps: Int, numWorkers: Int): Array[Array[Vector]] = {
    // Array to hold the results.
    val result = Array.ofDim[Vector](steps,bodies.length)

    // The barrier for coordinating workers.
    val barrier = new Barrier(numWorkers)

    // A single worker.
    def worker(me: Int) = thread(s"worker($me)"){
      // This worker is responsible for £bodies[start..end)£. 
      val start = me*n/numWorkers; val end = (me+1)*n/numWorkers
      for(step <- 0 until steps){
        for(i <- start until end; j <- 0 until n; if i != j)
          bodies(i).updateVel(bodies(j))
        barrier.sync(me)
        for(i <- start until end) result(step)(i) = bodies(i).move()
        barrier.sync(me)
      }
    }

    // Run the workers. 
    run(|| (for(i <- 0 until numWorkers) yield worker(i)))
    result
  }
}
\end{scala}
\caption{The concurrent simulation.}
\label{fig:ConcurrentSimulation}
\end{figure}

\begin{instruction}
Study the implementation of the concurrent simulation.
\end{instruction}

We can test the concurrent simulation using the standard technique of
comparing it to the sequential simulation.  We generate some input data at
random, run both simulations, and compare the results. 

In previous chapters, we avoided shared variables, other than read-only
variables.  We have relaxed this restriction here, but avoided races by
imposing rules on when different threads can read or write shared variables,
enforced via barrier synchronisations.  It is important to be clear about the
rules for accessing shared variables in such cases.

%%%%%%%%%%%%%%%%%%%%%%%%%%%%%%%%%%%%%%%%%%%%%%%%%%%%%%%%%%%%




%%%%%%%%%%%%%%%%%%%%%%%%%%%%%%%%%%%%%%%%%%%%%%%%%%%%%%%%%%%%
% Commented out section
%%%%%%%%%%%%%%%%%%%%%%%%%%%%%%%%%%%%%%%%%%%%%%%%%%%%%%%


%% \begin{slide}
%% \heading{Particle computations}

%% We now consider the problem of simulating the evolution of a large collection
%% of $N$ particles (e.g. stars or planets) that evolve under gravity.  

%% We can do a discrete time simulation, with time quantum \SCALA{deltaT}.

%% In particular, we'll consider how to construct a concurrent program for this
%% task.

%% For each particle, we need to record its mass, position and velocity:
%% %
%% \begin{scala}
%% val mass = new Array[Double](N)
%% type Vector = (Double, Double, Double)
%% val position = new Array[Vector](N)
%% val velocity = new Array[Vector](N)
%% \end{scala}
%% \end{slide}

%% %%%%%

%% \begin{slide}
%% \heading{Some physics}

%% Particle~$i$ will exert a force on particle~$j$ of magnitude
%% $G.mass(i).mass(j)/distance^2$, where $G \approx 6.67 \times 10^{-11}$ is the
%% gravitational constant, and $distance$ is the distance between them.  This is
%% an attractive force, with direction along the vector from $position(j)$ to
%% $position(i)$.  

%% We could calculate the total force exerted on each particle by all other
%% particles, and store the results in 
%% %
%% \begin{scala}
%% val force = new Array[Vector](N)
%% \end{scala}

%% We could then update the velocity of each particle~\SCALA{i}
%% by:\footnote{assuming we have defined \SCALA{+}, \SCALA{*} and \SCALA{/} to
%%   operate over \SCALA{Vector}}
%% %
%% \begin{scala}
%% velocity(i) += deltaT*force(i)/mass(i)
%% \end{scala}

%% And we could update the position of particle~\SCALA{i} by
%% %
%% \begin{scala}
%% position(i) += deltaT*velocity(i)
%% \end{scala}
%% \end{slide}

%% %%%%%

%% \begin{slide}
%% \heading{Calculating the forces}

%% Note that the force exerted by particle $i$ on particle $j$ is the same as the
%% force exerted by particle $j$ on particle~$i$ (except in the opposite
%% direction).  For reasons of efficiency, we do not want to calculate this
%% quantity twice.  

%% What we will do is allocate each thread some set~$S$ of particles.  For each
%% particle~$i \in S$, the thread will calculate the forces between $i$ and all
%% particles~$j$ with $j>i$.  These will be added to the total forces on both $i$
%% and $j$.  Something like:
%% %
%% \begin{scala}
%% for(i <- S; j <- i+1 until N){
%%   val thisForce = ... // force exerted on i by j
%%   force(i) += thisForce
%%   force(j) -= thisForce
%% }
%% \end{scala}
%% \end{slide}

%% %%%%%

%% \begin{slide}
%% \heading{Avoiding race conditions}

%% The code on the previous slide has an obvious race condition: several
%% threads might be trying to write to the same \SCALA{force(i)}
%% simultaneously. 

%% Instead we arrange for each thread~$me$ to write to its own vector of forces.
%% %
%% \begin{scala}
%% val force1 = new Array[Array[Vector]](p,N)
%% \end{scala}
%% %
%% Something like:
%% %
%% \begin{scala}
%% for(i <- S; j <- i+1 until N){
%%   val thisForce = ... // force exerted on i by j
%%   force1(me)(i) += thisForce
%%   force1(me)(j) -= thisForce
%% }
%% \end{scala}
%% \end{slide}

%% %%%%%

%% \begin{slide}
%% \heading{Calculating the total forces}

%% Once all the \SCALA{force1} values have been calculated, the threads can
%% perform a barrier synchronisation.  

%% Then the thread with set of particles~$S$ can, for each particle~$i \in S$,
%% calculate the total force and update the velocity and position:
%% %
%% \begin{scala}
%% for(i <- S){
%%   var force: Vector = (0.0, 0.0, 0.0)
%%   for(k <- 0 until p) force += force1(k)(i)
%%   velocity(i) += deltaT*force/mass(i)
%%   position(i) += deltaT*velocity(i)
%% }
%% \end{scala}
%% %
%% We expect to have $p \ll N$, so the cost of this extra summation is
%% comparatively small. 

%% The threads can then perform another barrier synchronisation before the next
%% round. 
%% \end{slide}

%% %%%%%

%% \begin{slide}
%% \heading{The pattern of synchronisation}

%% This pattern of synchronisation is very common:
%% %
%% \begin{scala}
%% <initialisation>
%% barrier.sync
%% while(true){
%%   <read all variables>
%%   barrier.sync
%%   <write own variables>
%%   barrier.sync
%% }
%% \end{scala}
%% %
%% The final synchronisation on each iteration could be replaced by a
%% synchronisation using a combining barrier, to decide whether to continue.
%% \end{slide}

%%%%%

%% \begin{slide}
%% \heading{Load balancing}

%% We want to choose the sets $S$ allocated to different threads so as to
%% balance the total load.

%% Note that the cost of calculating all the forces for particle~$i$ is
%% $\Theta(N-i)$: not all particles are equal in this regard.

%% One way to balance the load is to split the $N$ particles into $2p$
%% segments, each of size $segSize = N/2p$.  Then we can allocate process~$me$
%% the segments~$me$ and $2p-me-1$, i.e.\ particles $[me.segSize \upto
%%   (me+1).segSize)$ and $[(2p-me-1).segSize \upto (2p-me).segSize)$.
%% \end{slide}




 % intro; astronomical bodies simulation
\section{Example: prefix sums}

% An interval, open on the left, closed on the right. 
\def\revint#1#2{(#1 \upto #2]}

We now consider a problem from linear algebra.  Suppose we have an array~|a|
holding |n| integers.  We want to calculate the \emph{prefix sums}, i.e., the
sums of the first $j$ elements, for $j = 1, \ldots, \sm n$, and store them in
an array~|sum|, also of size~|n|.  
\[\mstyle
\forall k \in \set{0, \ldots, \sm n-1} \spot 
  \sm{sum}(k) = \textstyle\sum \sm{a}[0 \upto k] 
\]
The following simple sequential program achieves this.
\begin{scala}
  val sum = new Array[Int](n); var s = 0
  for(i <- 0 until n){ s += a(i); sum(i) = s }
\end{scala}
%
This sequential program runs in time $\Theta(\sm n)$.  Our concurrent program
will use|n| threads, and compete in $\Theta(\log \sm n)$ rounds. 

The idea of the algorithm is that each thread is responsible for one entry in
the result array.  Each thread~|me| sets $\sm{sum}(\sm{me})$ to $\sum \sm
a\revint{0}{\sm{me}}$, for $me = 0,\ldots,\sm n-1$.

The program proceeds in rounds, with a barrier synchronisation at the end of
each round.  Each thread has a thread-local variable~|s|, which is the sum it
has calculated so far.  At the end of round~$r$, thread~|me| will have
\[\mstyle
\sm s = \textstyle\sum \sm a(\sm{me}-\sm{gap} \upto \sm{me}]
   \mbox{ where } \sm{gap} = 2^r
\]
i.e., the sum of a segment of size~|gap|, ending at position~|me|.  Here, we
invent fictitious elements $\sm a(i) = 0$ for $i < 0$, so that the above
segment sum is defined.  If $\sm{gap} > \sm{me}$ then $\sum \sm
a\revint{\sm{me}-\sm{gap}}{\sm{me}} = \sum \sm a\revint{0}{\sm{me}}$, i.e.~the
prefix sum that |me| is trying to calculate.

On each round, thread~|me| obtains the value of thread $(\sm{me}-\sm{gap})$'s
|s|, which is $\sum \sm a\revint{\sm{me}-2 \times
  \sm{gap}}{\sm{me}-\sm{gap}}$, since thread $\sm{me}-\sm{gap}$ is running the
same program, and is on the same round, because of the use of barrier
synchronisations.  This allows thread~|me| to calculate $\sum \sm
a\revint{\sm{me}-2 \times \sm{gap}}{\sm{me}}$, to maintain the above
invariant.

The program will terminate when $\sm{gap} \ge \sm n$.  At this point,
$\sm{gap} > \sm{me}$ for each thread~|me|, so, as noted above, each thread has
calculated the desired prefix sum. 


The program is in Figure~\ref{fig:PrefixSums}.  This uses a barrier
synchronisation |barrier| to synchronise the threads, so that they are all on
the same round at each point.  It uses an array |toSummers| of channels for
threads to send the partial sums to each other.  This array is indexed by the
receiver's identity.  Each channel needs to be buffered to avoid deadlocks. 

%%%%%

\begin{figure}
\begin{scala}[numbers = left]
/** Calculate prefix sums of array £a£ of size £n£ in £$\Theta(\log \sm n)$£ (parallel)
  *  steps.  */
class PrefixSums(n: Int, a: Array[Int]){
  require(n == a.size)

  /** Shared array, in which sums are calculated. */
  private val sum = new Array[Int](n) 

  /** Barrier synchronisation object. */
  private val barrier = new Barrier(n)

  /** Channels on which values are sent, indexed by receiver's identity. */
  private val toSummers = Array.fill(n)(new OnePlaceBuffChan[Int](1))

  /** An individual thread.  £summer(me)£ sets £sum(me)£ to £$\sum \sm a\revint{0}{\sm{me}}$£. */
  private def summer(me: Int) = thread(s"summer($me)"){
    // Invariant: £$\sm{gap} = 2^{\ss r}$£ and £$\sm s = \sum \sm a\revint{\sm{me}-\sm{gap}}{\sm{me}}$£ (with fictious
    // values £$\sm a(i) = 0$£ for £$i < 0$£).  £r£ is the round number.
    var r = 0; var gap = 1; var s = a(me)

    while(gap < n){
      if(me+gap < n) toSummers(me+gap)!s // Pass my value up the line.
      if(gap <= me){ £\hspace{23mm}£  // Receive from £$\sm{me}-\sm{gap}$£.
	val inc = toSummers(me)?()  // £$\sm{inc} = \sum \sm a\revint{\sm{me}-2\times\sm{gap}}{\sm {me}-\sm{gap}}$£.
	s = s + inc £\hspace{27mm}£ // £$\sm s = \sum \sm a\revint{\sm{me}-2\times\sm{gap}}{\sm{me}}$£.
      } £\hspace{47.5mm}£ // £$\sm s = \sum \sm a\revint{\sm{me}-2\times\sm{gap}}{\sm{me}}$\label{line:skip-if}£.
      r += 1; gap += gap £\hspace{15.2mm}£  // £$\sm s = \sum \sm a\revint{\sm{me}-\sm{gap}}{\sm{me}}$£.
      barrier.sync(me)
    }
    sum(me) = s
  }

  /** Calculate the prefix sums. */
  def apply(): Array[Int] = {
    run(|| (for (i <- 0 until n) yield summer(i)))
    sum
  }
}
\end{scala}
\caption{The prefix sum program.}
\label{fig:PrefixSums}
\end{figure}

%%%%%

Each thread then proceeds as described above.  The comments in the code
justify that the invariant is maintained.  Note that a thread does not need to
send its partial sum if $\sm{me}+\sm{gap} \ge \sm{n}$: there is no thread to
send to in this case.  Likewise, a thread does not need to receive if
$\sm{gap} > \sm{me}$: it already holds the relevant prefix sum, and the
property claimed at line~\ref{line:skip-if} still holds, despite the thread
skipping the body of the |if| statement.

\begin{instruction}
Study the details of the implementation.  In particular, make sure you
understand why the claimed invariant is maintained.
\end{instruction}

It is not immediately obvious why we need the barrier synchronisation.  It
might appear that the channel communications keep the threads synchronised, so
the barrier synchronisation is superfluous; however, this turns out not to be
true.  

Suppose we didn't use a barrier synchronisation.  Consider a particular
thread~|me| on round~|r|, and let $\sm{gap} = 2^{\ss r}$.  It expects to
receive from thread~$\sm{me}-\sm{gap}$; but suppose that thread is slow.  And
suppose thread~$\sm{me}-2 \times\sm{gap}$ is fast and has proceeded to
round~|r+1|.  Then thread~|me| will instead receive from thread~$\sm{me}-2
\times\sm{gap}$.  This doesn't actually affect the result thread~|me| ends up
with, as it will receive from thread~$\sm{me}-\sm{gap}$ on a later round.
However, it does mean that the value thread~|me| sends on the next round is
incorrect; and that means that the correctness argument no longer holds.  

The algorithm uses $\Theta(\log \sm n)$ rounds.  Each barrier synchronisation
takes $\Theta(\log \sm n)$ time.  This makes $\Theta((\log \sm n)^2)$ in total
(ignoring the initialisation time), assuming we have |n| machine threads.
This latter assumption is rather unrealistic; nevertheless, the example
illustrates some interesting ideas.

We can test the concurrent implementation against a sequential implementation,
in the normal way. 

Exercise~\ref{ex:prefix-sums-shared} asks you to adapt the program so that the
threads communicate via shared variables rather than channels.

%%%%%

%% \begin{selfnote}
%% Without the barrier synchronisation, a summer could receive values out of
%% order.  For example, \SCALA{Summer(3)} could first receive \SCALA{a(0)+a(1)}
%% from \SCALA{Summer(1)}, and then receive \SCALA{a(2)} from \SCALA{Summer(2)},
%% if Summer(1) happens to be faster than Summer(2).  Despite this
%% \SCALA{Summer(3)} ends up with the right answer; but it will pass
%% \SCALA{a(0)+a(1)+a(3)} to \SCALA{Summer(5)} in the second step (instead of
%% \SCALA{a(2)+a(3)}), so \SCALA{Summer(5)} ends up with the wrong value.
%% \end{selfnote}
 % prefix sums
\section{Jacobi iteration}

In this section, we study a problem from linear algebra, namely to find an
approximate solution to a large system of simultaneous linear equations.  We
will assume some familiarity with the topic. 

Given an $n$-by-$n$ matrix~$A$ with entries $a_{ij}$ for $i,j \in
\interval{0}{n}$, and a vector~$b$ of size~$n$, we want to find a vector~$x$
of size~$n$ such that $Ax \approx b$.
%
The standard technique to solve this problem is Gaussian elimination.
However, this algorithm runs in time $O(n^3)$, which can be prohibitive for
large~$n$.  Jacobi iteration provides an alternative technique, which can be
more efficient.  We start by describing this technique.

Let $D$ be a matrix containing the diagonal entries of~$A$, and let $R$ be a
matrix containing the rest of the entries, so $A = D+R$.  We require that each
diagonal entry is non-zero; this means that $D\inverse$ exists: it is also
diagonal, with entries $1/a_{ii}$.

Then we have
\[
\begin{array}{cl}
& Ax = b \\
\iff & Dx + Rx = b \\
\iff & x = D\inverse (b - Rx).
\end{array}
\]
This suggests calculating a sequence of approximations $x^{0}, x^{(1)},
x^{(2)}, \ldots$, where $x^{(0)}$ is arbitrary (say all $0$s), and
\begin{eqnarray}
x^{(k+1)} & = & D\inverse \left( b-R x^{(k)} \right), \label{eqn:Jacobi}
\end{eqnarray}
for $k = 0,1,\ldots$.
If this iteration converges to a solution $x$, then $D\inverse (b - Rx)$, so
$Ax = b$, as required.  In fact, it can be shown that the iteration does
indeed converge if
\[\mstyle
 \left \| a_{ii} \right \| > \sum_{j \ne i} \left \| a_{ij} \right \| 
  \qquad \mbox{for all $i$}, 
\]
i.e., each row is dominated by its entry on the diagonal. 

It is convenient to rewrite equation~(\ref{eqn:Jacobi}) in component form:
\begin{eqnarray}
x^{(k+1)}_i & = & \frac{1}{a_{ii}} \left(b_i -\sum_{j\ne i} a_{ij} x^{(k)}_j\right),
\label{eqn:Jacobi-component}
\end{eqnarray}
for each~$k$, and for $i = 0,\ldots,n-1$.  This follows from the fact that
$D\inverse$ is diagonal with entries  $1/a_{ii}$.

We will produce two concurrent implementations of Jacobi iteration, using
shared variables and message passing, respectively.  We start by considering a
sequential implementation, given in Figure~\ref{fig:seq-Jacobi}.

%%%%%

\begin{figure}
\begin{scala}
object Jacobi{
  type Vector = Array[Double]
  type Matrix = Array[Array[Double]]
  val Epsilon = 0.000001 // Tolerance for convergence.
}

import Jacobi._

trait Jacobi{
  /** Find £x£ such that £a x£ is approximately £b£, performing Jacobi iteration until
    * successive iterations vary by at most £Epsilon£.
    * Precondition: £a£ is of size £n£ by £n£, and £b£ is of size £n£, for some £n£. */
  def solve(a: Matrix, b: Vector): Vector

  /** Calculate new value for £x(i)£ based on the old value. */ 
  protected def update(a: Matrix, b: Vector, x: Vector, i: Int, n: Int): Double = {
    var sum = 0.0
    for(j <- 0 until n; if j != i) sum += a(i)(j)*x(j)
    (b(i)-sum) / a(i)(i)
  }
}

/** A sequential implementation of Jacob iteration. */
object SeqJacobi extends Jacobi{
  def solve(a: Matrix, b: Vector): Vector = {
    val n = a.length; require(a.forall(_.length == n) && b.length == n)
    var oldX, newX = new Vector(n); var done = false

    while(!done){
      done = true
      for(i <- 0 until n){
        newX(i) = update(a, b, oldX, i, n)
	done &&= Math.abs(oldX(i)-newX(i)) < Epsilon
      }
      if(!done){ val t = oldX; oldX = newX; newX = t } // Swap arrays.
    }
    newX
  }
}
\end{scala}
\caption{A sequential implementation of Jacobi iteration.}
\label{fig:seq-Jacobi}
\end{figure}

%%%%%

The companion object |Jacobi| defines type synonyms for vectors and matrices:
one- and two-dimensional arrays, respectively.  Our implementations will
iterate until all entries in the new approximation are within |Epsilon| of the
previous approximation.

Our various implementations will extend the trait |Jacobi|, in particular
implementing a function |solve| to perform the iteration.  To avoid repeated
code, we include here a  function |update| that calculates the new value for a
component of the approximation, based on
equation~(\ref{eqn:Jacobi-component}). 

The sequential solution, in |SeqJacobi| keeps track of two approximations, the
previous one and the new one, in variables |oldX| and |newX|.  At each
iteration, it updates |newX|, following equation~(\ref{eqn:Jacobi}).  It swaps
|oldX| and |newX| at the end of each iteration.  It also maintains a variable
|done| that records whether each component of the new approximation is within
|Epsilon| of the previous.  The main loop terminates when this is true of all
components, corresponding to our convergence criterion mentioned earlier. 

\begin{instruction}
Make sure you understand the sequential implementation.
\end{instruction}

%%%%%%%%%%%%%%%%%%%%%%%%%%%%%%%%%%%%%%%%%%%%%%%%%%%%%%%%%%%%

\subsection{A concurrent implementation using shared variables}

We now consider the concurrent implementation.  We will use |p| workers, and
split the calculation of |newX| between the workers.  The implementation
proceeds in rounds, each round corresponding to one iteration of the
sequential solution.  In each round, all workers can read all of \SCALA{oldX}
and each worker can write its own segment of \SCALA{newX}.  The roles of the
arrays swaps between rounds.  We use a barrier synchronisation at the end of
each round, to keep the workers synchronised, and to avoid races. 

On each iteration, each thread can test whether its segment of \SCALA{x} has
converged, say using a thread-local variable \SCALA{myDone}.  The iteration
should terminate when \emph{all} the threads have $\sm{myDone} = \sm{true}$.
We can test this as part of the barrier synchronisation at the end of each
round; however, this requires a slightly different form of barrier object.  

A \emph{combining barrier} is associated with a function \SCALA{f: (A,A) =>
  A}, which is assumed to be associative.  Each thread contributes some piece
of data~$x_i: \sm{A}$ to each synchronisation.  All the threads then receive
back the result of combining all the $x_i$ values together using~|f|, i.e.,
\[\mstyle
\sm f(x_0, \sm f(x_1, \sm f(x_2, \ldots, \sm f(x_{p-2}, x_{p-1})\ldots),
\]
where $x_0, \ldots, x_{p-1}$ are the data provided, in some order.  In most
applications, |f| is commutative, and so the order doesn't matter.

In SCL, the expression |new CombiningBarrier(p, f)| creates a combining
barrier for |p| threads, associated with function~|f|.  In the Jacobi
iteration example, we want the combining barrier to calculate the conjunction
of the workers' |myDone| variables, so we can define the combining barrier
by\footnote{The notation {\scalashape \_ \&\& \_} uses placeholder notation;
  see Scala box~\ref{sb:anon-function}.  It represents a function that takes
  two arguments, \SCALA{x} and~{\scalashape y}, and returns {\scalashape x
    \&\& y}.}:
%
\begin{scala}
  val combBarrier = new CombiningBarrier[Boolean](p, _ && _)
\end{scala}
%
Each thread can then execute
\begin{scala}
  done = combBarrier.sync(me, myDone)
\end{scala}
%
Each thread receives back |true| if  all threads pass in |true|.

This particular form of combining barrier---returning the conjunction of the
inputs--is very common, so SCL contains a built-in equivalent version, created
by |new AndBarrier(p)|.  Likewise, an |OrBarrier| returns the disjunction of
the inputs.

%%%%%

\begin{figure}
\begin{scala}
/** A concurrent implementation of Jacobi iteration, using shared variables. */
class ConcJacobi(p: Int) extends Jacobi{
  private val combBarrier = new AndBarrier(p)

  def solve(a: Matrix, b: Vector): Vector = {
    val n = a.length
    require(a.forall(_.length == n) && b.length == n)
    /* The start of the segment calculated by thread £id£. */
    @inline def startFor(id: Int) = id*n/p
    val x0, x1 = new Vector(n)
    var result: Vector = null // Ends up storing the final result.

    // Worker to handle rows £[startFor(me) .. startFor(me+1))£.
    def worker(me: Int) = thread(s"worker($me)"){
      val start = startFor(me); val end = startFor(me+1)
      var oldX = x0; var newX = x1; var done = false
      while(!done){
        var myDone = true
        for(i <- start until end){
          newX(i) = update(a, b, oldX, i, n)
	  myDone &&= Math.abs(oldX(i)-newX(i)) < Epsilon
        }
        done = combBarrier.sync(me, myDone)
        if(!done){ val t = oldX; oldX = newX; newX = t } // Swap arrays.
      }
      // Worker £0£ sets £result£ to the final result.
      if(start == 0) result = newX
    }

    // Run system.
    run(|| (for (i <- 0 until p) yield worker(i)))
    result
  }
}
\end{scala}
\caption{A concurrent implementation of Jacobi iteration.}
\label{fig:ConcJacobi} 
\end{figure}

%%%%%

The concurrent implementation is in Figure~\ref{fig:ConcJacobi}.  The worker
with identity~|me| is responsible for calculating the entries $\sm x
\interval{\sm{me} \times \sm n / \sm p}{(\sm{me}+1) \times \sm n / \sm p}$.

The implementation uses two vectors, |x0| and |x1| to represent two successive
approximations.  Each worker has thread-local variables, |oldX| and |newX|,
which reference these vectors, representing the previous and new
approximations, respectively.  The worker swaps these variables at the end of
each round.  On each iteration, all threads have the same values for their
|oldX| and |newX|.

Termination is decided as described above.  At the end, worker~|0| sets the
variable |result| to hold the final result.

We can test the concurrent version by comparing its results against those for
the sequential version, in the normal way.  For small values of |n|, the
sequential version is faster: the benefits of parallelisation are outweighed
by the overheads of the synchronisation and keeping threads' caches
consistent.  But for larger values of |n|, the concurrent version is faster.

%%%%%

%% \begin{selfnote}
%% The computation time for each round is $\Theta(n^2)$ for the sequential
%% version, or $\Theta(n^2/p)$ for the concurrent version.  But there is a
%% communication time of $\Theta(n)$ for the concurrent version, to keep the
%% values in the caches up to date, and this is overwhelming for small values
%% of~$n$.
%% \end{selfnote}

%%%%%

An alternative approach is to use a \emph{single} vector, and to split each
round into two sub-rounds.  In the first sub-round, all threads read from the
shared array into thread-local variables (but write no shared variables).  In
the second sub-round, all threads write to their part of the shared array.
This requires an extra barrier synchronisation, between the two sub-rounds.
This approach is somewhat similar to our implementation of the simulation of
astronomical bodies. 

%%%%%%%%%%%%%%%%%%%%%%%%%%%%%%%%%%%%%%%%%%%%%%%%%%%%%%%%%%%%


\subsection{A concurrent implementation using message-passing}

We can convert the shared-memory program into a message-passing program, with
no shared variables.  Each thread has its own copy of~\SCALA{x}: all threads
should have the same value for this.  

On each iteration, each worker calculates the next value of its share
of~\SCALA{x}, and then distributes it to all other workers.  More precisely,
each thread sends a triple consisting of its own identity, the part of the
next value of~\SCALA{x} that it has just calculated, and a boolean that
indicates whether it is willing to terminate.  

Figure~\ref{fig:JacobiMP} gives the outline of the implementation.  The type
|Msg| represents messages sent between workers, as just described.  These are
sent on buffered channels, indexed by the recipient's identity.  We use a
|Barrier| object to provide synchronisation.

%%%%%

\begin{figure}
\begin{scala}
/** A concurrent implementation of Jacobi iteration, using message
  * passing. */
class JacobiMP0(p: Int) extends Jacobi{
  private val barrier = new Barrier(p)
  private type Msg = (Int, Vector, Boolean)
  // Channels to send messages to workers: indexed by receiver's identity.
  private val toWorker = Array.fill(p)(new BuffChan[Msg](p-1))

  def solve(a: Matrix, b: Vector): Vector = {
    val n = a.length; require(a.forall(_.length == n) && b.length == n)
    /* The start of the segment calculated by thread id. */
    @inline def startFor(id: Int) = id*n/p
    var result: Vector = null // will hold final result

    /* Worker to handle rows £[startFor(me) .. startFor(me+1))£. */
    def worker(me: Int) = // See Figure £\ref{fig:JacobiMP-worker}£.

    // Run system
    run(|| (for(i <- 0 until p) yield worker(i)))
    result
  }
}
\end{scala}
\caption{Jacobi iteration using message passing.}
\label{fig:JacobiMP}
\end{figure}

%%%%%

A worker is defined in Figure~\ref{fig:JacobiMP-worker}.  On each iteration,
the worker calculates its share of the next iteration.  This is similar to
previously, except it uses a thread-local variable~|newX| to store its
results, with a suitable offset.

%%%%%

\begin{figure}
\begin{scala}
    /* Worker to handle rows £[startFor(me) .. startFor(me+1))£. */
    def worker(me: Int) = thread(s"worker($me)"){
      val start = startFor(me); val end = startFor(me+1); val height = end-start
      var done = false; val x = new Vector(n)

      while(!done){
        done = true
        // £newX(i)£ holds the new value of £x(i+start)£.
        val newX = new Array[Double](height)
        // Update this section of £x£, storing results in £newX£.
        for(i <- start until end){
          newX(i-start) = update(a, b, x, i, n)
          done &&= Math.abs(x(i)-newX(i-start)) < Epsilon
        }
        // Send this section to all other workers.
        for(w <- 1 until p) toWorker((me+w)%p)!(me, newX, done)
        // Copy newX into x.
        for(i <- 0 until height) x(start+i) = newX(i)
        // Receive from others.
        for(k <- 0 until p-1){
          val (him, hisX, hisDone) = toWorker(me)?(); val offset = startFor(him)
          for(i <- 0 until hisX.length) x(offset+i) = hisX(i)
          done &&= hisDone
        }
        // Synchronise for end of round.
        barrier.sync(me)
      }
      if(me == 0) result = x // Copy final result.
    } // end of worker
\end{scala}
\caption{A worker for Jacobi iteration using message passing.}
\label{fig:JacobiMP-worker}
\end{figure}

The worker then sends its share to each of the other workers, together with
its identity, and a boolean indicating whether its share has converged; this
sending starts with the thread with identity one larger than its own, using
the technique from Section~\ref{sec:fully-connected} to avoid congestion on
the channels.  

Each worker then receives from the other workers, copying the data into its
own copy of~|x|, using an offset based on the sender's identity.  It performs
a barrier synchronisation at the end of each round.  It terminates if all
workers' shares have converged.

The barrier synchronisation is necessary for the following reason.  Suppose
one thread is very fast.  It could complete one round, and---if we didn't
use a barrier synchronisation---do its calculations for the next round, and
send its value of \SCALA{newX} while some slow thread is still doing some
sends from the previous round.  Hence a third thread could receive the fast
thread's value for the next round before it receives the slow thread's value
for the current round.  Thus this third thread will not have a complete or
consistent version of the current approximation. 

The previous version is a bit inefficient, because it involves copying the
data that is received into~\SCALA{x}; this copying takes time $O(\sm n)$ per iteration.  We
can replace \SCALA{x}, in each worker, by a two-dimensional array \SCALA{xs}.
Each row of~|xs| corresponds to the segment operated on by a particular
worker.  The array~|xs| represents the vector |x| formed by concatenating the
rows of |xs|. 
%%  That is, we have the abstraction:
%% \[
%% \sm{x} = \sm{xs.flatten}.
%% \]
Each worker then receives a block of data from another worker, and inserts it
into its~|xs| with a single update.  This copying takes time $O(\sm{p})$ per
round.  The code is available on the book website.  This change makes the
program about 45\% faster.
 % Jacobi iteration

\framebox{BFS example here}

%%%%%%%%%%%%%%%%%%%%%%%%%%%%%%%%%%%%%%%%%%%%%%%%%%%%%%%%%%%%

%%%%%

\begin{slide}
\heading{Communication with neighbours}

Several applications work on a rectangular grid, where the state of a cell on
one round depends only on the state of its neighbouring cells at the previous
round.  Examples:
%
\begin{itemize}
\item
Cellular automata;

\item
Solutions to differential equations, e.g.~in weather forecasting or fluid
dynamics;

\item Image processing, e.g.~smoothing.
\end{itemize}

In such cases, it is natural to allocate a horizontal strip of cells to each
thread.  At the end of each round, each thread communicates the state of its
top row to the thread above it, and communicates the state of its bottom row
to the thread below it.  

%% These rows can be passed by reference, but it might be necessary to make a
%% copy to avoid sharing references, e.g. \SCALA{up ! myA(start).toArray}.
\end{slide}

%%%%%

\begin{slide}
\heading{Passing arrays in Scala}

Arrays in Scala are reference objects, and so passed by reference.  

Suppose a particular thread has a two-dimensional array \SCALA{myA}, and it
is responsible for rows \SCALA{[start..end)}, then it can send its top row to
the thread above it (say on channel \SCALA{up}) by
%
\begin{scala}
up ! myA(start);
\end{scala}
%
The thread above it can receive it (say on channel \SCALA{receiveUp}) as:
%
\begin{scala}
myA(end) = receiveUp?()
\end{scala}
%
However, the two threads now share references to this array, so updates made
by one thread will have an effect on the other thread --- a race!

The solution is either for the first thread to re-initialise
\SCALA{myA(start)}, e.g., by \SCALA{myA(start) = new Array[Int](N)}, or to
make a \emph{copy} of the array, e.g., by \SCALA{up ! (myA(start).clone)}.
\end{slide}


%%%%%

\begin{slide}
\heading{Summary}

\begin{itemize}
\item 
Synchronous data parallel programming;
% Heart-beat algorithms;

\item
Barrier synchronisation; combining barrier synchronisation;

\item
Examples, using shared memory or message-passing.
\end{itemize}
\end{slide}

%%%%%%%%%%

\exercises

Log time barrier.  % \label{ex:log-barrier} 


Prefix sums with shared variables. % \label{ex:prefix-sums-shared}

Example of combining barrier with non-commutative function --- index of
largest item in array

 countDups.tex, using shared hashSet; maybe add contains operation to
 concurrent set first.
