\section{Jacobi iteration}

In this section, we study a problem from linear algebra, namely to find an
approximate solution to a large system of simultaneous linear equations.  We
will assume some familiarity with the topic. 

Given an $n$-by-$n$ matrix~$A$ with entries $a_{ij}$ for $i,j \in
\interval{0}{n}$, and a vector~$b$ of size~$n$, we want to find a vector~$x$
of size~$n$ such that $Ax \approx b$.
%
The standard technique to solve this problem is Gaussian elimination.
However, this algorithm runs in time $O(n^3)$, which can be prohibitive for
large~$n$.  Jacobi iteration provides an alternative technique, which can be
more efficient.  We start by describing this technique.

Let $D$ be a matrix containing the diagonal entries of~$A$, and let $R$ be a
matrix containing the rest of the entries, so $A = D+R$.  We require that each
diagonal entry is non-zero; this means that $D\inverse$ exists: it is also
diagonal, with entries $1/a_{ii}$.

Then we have
\[
\begin{array}{cl}
& Ax = b \\
\iff & Dx + Rx = b \\
\iff & x = D\inverse (b - Rx).
\end{array}
\]
This suggests calculating a sequence of approximations $x^{0}, x^{(1)},
x^{(2)}, \ldots$, where $x^{(0)}$ is arbitrary (say all $0$s), and
\begin{eqnarray}
x^{(k+1)} & = & D\inverse \left( b-R x^{(k)} \right), \label{eqn:Jacobi}
\end{eqnarray}
for $k = 0,1,\ldots$.
If this iteration converges to a solution $x$, then $D\inverse (b - Rx)$, so
$Ax = b$, as required.  In fact, it can be shown that the iteration does
indeed converge if
\[\mstyle
 \left \| a_{ii} \right \| > \sum_{j \ne i} \left \| a_{ij} \right \| 
  \qquad \mbox{for all $i$}, 
\]
i.e., each row is dominated by its entry on the diagonal. 

It is convenient to rewrite equation~(\ref{eqn:Jacobi}) in component form:
\begin{eqnarray}
x^{(k+1)}_i & = & \frac{1}{a_{ii}} \left(b_i -\sum_{j\ne i} a_{ij} x^{(k)}_j\right),
\label{eqn:Jacobi-component}
\end{eqnarray}
for each~$k$, and for $i = 0,\ldots,n-1$.  This follows from the fact that
$D\inverse$ is diagonal with entries  $1/a_{ii}$.

We will produce two concurrent implementations of Jacobi iteration, using
shared variables and message passing, respectively.  We start by considering a
sequential implementation, given in Figure~\ref{fig:seq-Jacobi}.

%%%%%

\begin{figure}
\begin{scala}
object Jacobi{
  type Vector = Array[Double]
  type Matrix = Array[Array[Double]]
  val Epsilon = 0.000001 // Tolerance for convergence.
}

import Jacobi._

trait Jacobi{
  /** Find £x£ such that £a x£ is approximately £b£, performing Jacobi iteration until
    * successive iterations vary by at most £Epsilon£.
    * Precondition: £a£ is of size £n£ by £n£, and £b£ is of size £n£, for some £n£. */
  def solve(a: Matrix, b: Vector): Vector

  /** Calculate new value for £x(i)£ based on the old value. */ 
  protected def update(a: Matrix, b: Vector, x: Vector, i: Int, n: Int): Double = {
    var sum = 0.0
    for(j <- 0 until n; if j != i) sum += a(i)(j)*x(j)
    (b(i)-sum) / a(i)(i)
  }
}

/** A sequential implementation of Jacob iteration. */
object SeqJacobi extends Jacobi{
  def solve(a: Matrix, b: Vector): Vector = {
    val n = a.length; require(a.forall(_.length == n) && b.length == n)
    var oldX, newX = new Vector(n); var done = false

    while(!done){
      done = true
      for(i <- 0 until n){
        newX(i) = update(a, b, oldX, i, n)
	done &&= Math.abs(oldX(i)-newX(i)) < Epsilon
      }
      if(!done){ val t = oldX; oldX = newX; newX = t } // Swap arrays.
    }
    newX
  }
}
\end{scala}
\caption{A sequential implementation of Jacobi iteration.}
\label{fig:seq-Jacobi}
\end{figure}

%%%%%

The companion object |Jacobi| defines type synonyms for vectors and matrices:
one- and two-dimensional arrays, respectively.  Our implementations will
iterate until all entries in the new approximation are within |Epsilon| of the
previous approximation.

Our various implementations will extend the trait |Jacobi|, in particular
implementing a function |solve| to perform the iteration.  To avoid repeated
code, we include here a  function |update| that calculates the new value for a
component of the approximation, based on
equation~(\ref{eqn:Jacobi-component}). 

The sequential solution, in |SeqJacobi| keeps track of two approximations, the
previous one and the new one, in variables |oldX| and |newX|.  At each
iteration, it updates |newX|, following equation~(\ref{eqn:Jacobi}).  It swaps
|oldX| and |newX| at the end of each iteration.  It also maintains a variable
|done| that records whether each component of the new approximation is within
|Epsilon| of the previous.  The main loop terminates when this is true of all
components, corresponding to our convergence criterion mentioned earlier. 

\begin{instruction}
Make sure you understand the sequential implementation.
\end{instruction}

%%%%%%%%%%%%%%%%%%%%%%%%%%%%%%%%%%%%%%%%%%%%%%%%%%%%%%%%%%%%

\subsection{A concurrent implementation using shared variables}

We now consider the concurrent implementation.  We will use |p| workers, and
split the calculation of |newX| between the workers.  The implementation
proceeds in rounds, each round corresponding to one iteration of the
sequential solution.  In each round, all workers can read all of \SCALA{oldX}
and each worker can write its own segment of \SCALA{newX}.  The roles of the
arrays swaps between rounds.  We use a barrier synchronisation at the end of
each round, to keep the workers synchronised, and to avoid races. 

On each iteration, each thread can test whether its segment of \SCALA{x} has
converged, say using a thread-local variable \SCALA{myDone}.  The iteration
should terminate when \emph{all} the threads have $\sm{myDone} = \sm{true}$.
We can test this as part of the barrier synchronisation at the end of each
round; however, this requires a slightly different form of barrier object.  

A \emph{combining barrier} is associated with a function \SCALA{f: (A,A) =>
  A}, which is assumed to be associative.  Each thread contributes some piece
of data~$x_i: \sm{A}$ to each synchronisation.  All the threads then receive
back the result of combining all the $x_i$ values together using~|f|, i.e.,
\[\mstyle
\sm f(x_0, \sm f(x_1, \sm f(x_2, \ldots, \sm f(x_{p-2}, x_{p-1})\ldots),
\]
where $x_0, \ldots, x_{p-1}$ are the data provided, in some order.  In most
applications, |f| is commutative, and so the order doesn't matter.

In SCL, the expression |new CombiningBarrier(p, f)| creates a combining
barrier for |p| threads, associated with function~|f|.  In the Jacobi
iteration example, we want the combining barrier to calculate the conjunction
of the workers' |myDone| variables, so we can define the combining barrier
by\footnote{The notation {\scalashape \_ \&\& \_} uses placeholder notation;
  see Scala box~\ref{sb:anon-function}.  It represents a function that takes
  two arguments, \SCALA{x} and~{\scalashape y}, and returns {\scalashape x
    \&\& y}.}:
%
\begin{scala}
  val combBarrier = new CombiningBarrier[Boolean](p, _ && _)
\end{scala}
%
Each thread can then execute
\begin{scala}
  done = combBarrier.sync(me, myDone)
\end{scala}
%
Each thread receives back |true| if  all threads pass in |true|.

This particular form of combining barrier---returning the conjunction of the
inputs--is very common, so SCL contains a built-in equivalent version, created
by |new AndBarrier(p)|.  Likewise, an |OrBarrier| returns the disjunction of
the inputs.

%%%%%

\begin{figure}
\begin{scala}
/** A concurrent implementation of Jacobi iteration, using shared variables. */
class ConcJacobi(p: Int) extends Jacobi{
  private val combBarrier = new AndBarrier(p)

  def solve(a: Matrix, b: Vector): Vector = {
    val n = a.length
    require(a.forall(_.length == n) && b.length == n)
    /* The start of the segment calculated by thread £id£. */
    @inline def startFor(id: Int) = id*n/p
    val x0, x1 = new Vector(n)
    var result: Vector = null // Ends up storing the final result.

    // Worker to handle rows £[startFor(me) .. startFor(me+1))£.
    def worker(me: Int) = thread(s"worker($me)"){
      val start = startFor(me); val end = startFor(me+1)
      var oldX = x0; var newX = x1; var done = false
      while(!done){
        var myDone = true
        for(i <- start until end){
          newX(i) = update(a, b, oldX, i, n)
	  myDone &&= Math.abs(oldX(i)-newX(i)) < Epsilon
        }
        done = combBarrier.sync(me, myDone)
        if(!done){ val t = oldX; oldX = newX; newX = t } // Swap arrays.
      }
      // Worker £0£ sets £result£ to the final result.
      if(start == 0) result = newX
    }

    // Run system.
    run(|| (for (i <- 0 until p) yield worker(i)))
    result
  }
}
\end{scala}
\caption{A concurrent implementation of Jacobi iteration.}
\label{fig:ConcJacobi} 
\end{figure}

%%%%%

The concurrent implementation is in Figure~\ref{fig:ConcJacobi}.  The worker
with identity~|me| is responsible for calculating the entries $\sm x
\interval{\sm{me} \times \sm n / \sm p}{(\sm{me}+1) \times \sm n / \sm p}$.

The implementation uses two vectors, |x0| and |x1| to represent two successive
approximations.  Each worker has thread-local variables, |oldX| and |newX|,
which reference these vectors, representing the previous and new
approximations, respectively.  The worker swaps these variables at the end of
each round.  On each iteration, all threads have the same values for their
|oldX| and |newX|.

Termination is decided as described above.  At the end, worker~|0| sets the
variable |result| to hold the final result.

We can test the concurrent version by comparing its results against those for
the sequential version, in the normal way.  For small values of |n|, the
sequential version is faster: the benefits of parallelisation are outweighed
by the overheads of the synchronisation and keeping threads' caches
consistent.  But for larger values of |n|, the concurrent version is faster.

%%%%%

%% \begin{selfnote}
%% The computation time for each round is $\Theta(n^2)$ for the sequential
%% version, or $\Theta(n^2/p)$ for the concurrent version.  But there is a
%% communication time of $\Theta(n)$ for the concurrent version, to keep the
%% values in the caches up to date, and this is overwhelming for small values
%% of~$n$.
%% \end{selfnote}

%%%%%

An alternative approach is to use a \emph{single} vector, and to split each
round into two sub-rounds.  In the first sub-round, all threads read from the
shared array into thread-local variables (but write no shared variables).  In
the second sub-round, all threads write to their part of the shared array.
This requires an extra barrier synchronisation, between the two sub-rounds.
This approach is somewhat similar to our implementation of the simulation of
astronomical bodies. 

%%%%%%%%%%%%%%%%%%%%%%%%%%%%%%%%%%%%%%%%%%%%%%%%%%%%%%%%%%%%


\subsection{A concurrent implementation using message-passing}

We can convert the shared-memory program into a message-passing program, with
no shared variables.  Each thread has its own copy of~\SCALA{x}: all threads
should have the same value for this.  

On each iteration, each worker calculates the next value of its share
of~\SCALA{x}, and then distributes it to all other workers.  More precisely,
each thread sends a triple consisting of its own identity, the part of the
next value of~\SCALA{x} that it has just calculated, and a boolean that
indicates whether it is willing to terminate.  

Figure~\ref{fig:JacobiMP} gives the outline of the implementation.  The type
|Msg| represents messages sent between workers, as just described.  These are
sent on buffered channels, indexed by the recipient's identity.  We use a
|Barrier| object to provide synchronisation.

%%%%%

\begin{figure}
\begin{scala}
/** A concurrent implementation of Jacobi iteration, using message
  * passing. */
class JacobiMP0(p: Int) extends Jacobi{
  private val barrier = new Barrier(p)
  private type Msg = (Int, Vector, Boolean)
  // Channels to send messages to workers: indexed by receiver's identity.
  private val toWorker = Array.fill(p)(new BuffChan[Msg](p-1))

  def solve(a: Matrix, b: Vector): Vector = {
    val n = a.length; require(a.forall(_.length == n) && b.length == n)
    /* The start of the segment calculated by thread id. */
    @inline def startFor(id: Int) = id*n/p
    var result: Vector = null // will hold final result

    /* Worker to handle rows £[startFor(me) .. startFor(me+1))£. */
    def worker(me: Int) = // See Figure £\ref{fig:JacobiMP-worker}£.

    // Run system
    run(|| (for(i <- 0 until p) yield worker(i)))
    result
  }
}
\end{scala}
\caption{Jacobi iteration using message passing.}
\label{fig:JacobiMP}
\end{figure}

%%%%%

A worker is defined in Figure~\ref{fig:JacobiMP-worker}.  On each iteration,
the worker calculates its share of the next iteration.  This is similar to
previously, except it uses a thread-local variable~|newX| to store its
results, with a suitable offset.

%%%%%

\begin{figure}
\begin{scala}
    /* Worker to handle rows £[startFor(me) .. startFor(me+1))£. */
    def worker(me: Int) = thread(s"worker($me)"){
      val start = startFor(me); val end = startFor(me+1); val height = end-start
      var done = false; val x = new Vector(n)

      while(!done){
        done = true
        // £newX(i)£ holds the new value of £x(i+start)£.
        val newX = new Array[Double](height)
        // Update this section of £x£, storing results in £newX£.
        for(i <- start until end){
          newX(i-start) = update(a, b, x, i, n)
          done &&= Math.abs(x(i)-newX(i-start)) < Epsilon
        }
        // Send this section to all other workers.
        for(w <- 1 until p) toWorker((me+w)%p)!(me, newX, done)
        // Copy newX into x.
        for(i <- 0 until height) x(start+i) = newX(i)
        // Receive from others.
        for(k <- 0 until p-1){
          val (him, hisX, hisDone) = toWorker(me)?(); val offset = startFor(him)
          for(i <- 0 until hisX.length) x(offset+i) = hisX(i)
          done &&= hisDone
        }
        // Synchronise for end of round.
        barrier.sync(me)
      }
      if(me == 0) result = x // Copy final result.
    } // end of worker
\end{scala}
\caption{A worker for Jacobi iteration using message passing.}
\label{fig:JacobiMP-worker}
\end{figure}

The worker then sends its share to each of the other workers, together with
its identity, and a boolean indicating whether its share has converged; this
sending starts with the thread with identity one larger than its own, using
the technique from Section~\ref{sec:fully-connected} to avoid congestion on
the channels.  

Each worker then receives from the other workers, copying the data into its
own copy of~|x|, using an offset based on the sender's identity.  It performs
a barrier synchronisation at the end of each round.  It terminates if all
workers' shares have converged.

The barrier synchronisation is necessary for the following reason.  Suppose
one thread is very fast.  It could complete one round, and---if we didn't
use a barrier synchronisation---do its calculations for the next round, and
send its value of \SCALA{newX} while some slow thread is still doing some
sends from the previous round.  Hence a third thread could receive the fast
thread's value for the next round before it receives the slow thread's value
for the current round.  Thus this third thread will not have a complete or
consistent version of the current approximation. 

The previous version is a bit inefficient, because it involves copying the
data that is received into~\SCALA{x}; this copying takes time $O(\sm n)$ per iteration.  We
can replace \SCALA{x}, in each worker, by a two-dimensional array \SCALA{xs}.
Each row of~|xs| corresponds to the segment operated on by a particular
worker.  The array~|xs| represents the vector |x| formed by concatenating the
rows of |xs|. 
%%  That is, we have the abstraction:
%% \[
%% \sm{x} = \sm{xs.flatten}.
%% \]
Each worker then receives a block of data from another worker, and inserts it
into its~|xs| with a single update.  This copying takes time $O(\sm{p})$ per
round.  The code is available on the book website.  This change makes the
program about 45\% faster.
