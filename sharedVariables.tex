\framebox{Include where shared variables first considered}


\heading{Caching and compiler optimisations}

Recall that multiprocessor machines may cache variables, and that Java does
not guarantee that the caches will be kept coherent, so two concurrent
threads may operate independently on their own cached copies of the same
variable!  And further, the compiler is allowed to optimise the code to
something semantically equivalent as a sequential program.  The Java Memory
Model
%% \footnote{%
  %% \url{http://www.cs.umd.edu/~pugh/java/memoryModel/jsr-133-faq.html}.}
defines this more formally.

However, when two threads synchronise on a channel communication, any writes
made by one thread before the synchronisation are guaranteed to be visible to
the other thread after the synchronisation.  
%  reads or write a channel, the updates in its cache are
% flushed to main memory, and cached values re-read from main
% memory. 
Further compiler optimisations may not reorder events to break this.


%%%%%

For example, in the code
\begin{scala}
val c = new SyncChan[Unit]
var x = -1
def p = thread{ x = 42; c!() }
def q = thread{ c?(); <use x> }
run(p || q)
\end{scala}
%
\SCALA{q} is guaranteed to use the value |42| for \SCALA{x} written
by~\SCALA{p} because:
%
\begin{itemize}
\item 
\SCALA{p} finishes writing before the synchronisation;

\item
The synchronisation ensures the caches are correctly updated;

\item
The compiler is not allowed to perform optimisations that reorder the accesses
to~\SCALA{x} with those to~\SCALA{c}.
\end{itemize}

%%%%%

\heading{Rule for disciplined interaction}

Earlier we said that parallel threads should use \emph{disjoint} sets of
variables.  We can weaken this slightly, to allow threads to share
variables, as long as they do so in a disciplined way. 

If two threads both access a variable (other than both reading it) then
there should be a (direct or indirect) synchronisation between the threads
after the first finishes accessing the variable, and before the second starts
accessing it.  In particular, this avoids race conditions: the second thread
``knows'' the first has finished with the variable.  The Java Memory Model
avoids problems with caching and compiler optimisations.

You should state clearly which threads may read or write variables in
different states, and this should be done to avoid race conditions.

%%%%%

\heading{Objects and message passing}

Reference objects can be passed across channels in the same way as simpler
types.  (The channel passes the reference rather than the object itself.)

However, this needs to be done with care: if two threads share an object,
there is the danger of race conditions, just as with standard shared
variables. 
 
Note, in particular, that reference objects are passed by reference rather
than by value: sometimes it is necessary to copy objects, rather than simply
passing them, to avoid unintended sharing.

Consider \SCALA{p(in, mid) || q(mid)} where
%
\begin{scala}
def p(in: ??[String], mid: ![Person]) = thread{
  val pers = new Person()
  while(true){ val n = in?(); pers.name = n; mid!pers }
}

def q(mid: ??[Person]) = thread{
  while(true){ val pers = mid?(); println(pers.name) }
} 
\end{scala}
%
What is wrong with this?

%%%%%
