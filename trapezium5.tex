\section{Encapsulation}
\label{sec:bag-of-tasks-encapsulation}

In the previous code, the design decisions concerning the use of
message-passing concurrency were interleaved with the computation itself.  If
we decided to implement the concurrency in a different way---for example, to
use one of the techniques we'll see later in the book---we would have to
change the code in several different places.  This is a very small program, so
it doesn't matter too much here; but it would be an issue in bigger programs.
%
It would be better to encapsulate each relevant design decision within a
single object, so that changes will involve just that object. 

Further, reasoning about correctness will (mostly) involve reasoning about a
single object at a time.  Each such object will be designed to allow
concurrent calls of its methods, and to avoid race conditions; we can persuade
ourselves that that is true by considering that object in isolation.  Code
outside that object can use it without worrying about how it is implemented.

We will see this idea of encapsulation a lot later in the book.  We will
encapsulate a datatype---such as a queue or a mapping---inside an object that
provides thread-safe operations on the datatype.  We can perform unit testing
on that datatype, considering its correctness in isolation.  Code outside the
object can mostly use the datatype in the same way as in a sequential setting.

%%%%%

For the example of this chapter, we will define concurrent objects with the
interfaces given in Figure~\ref{fig:BoT-interfaces}.  The |BagOfTasks| objects
have an operation to allow a worker thread to get a task.  We have made the
design decision to return the special value |null| in the case that there are
no more tasks.  An alternative would be to throw a |Stopped| exception, as
earlier; but using |null| seems cleaner, and is appropriate in this case
because |null| is easily distinguished from a real task.  The |Collector|
objects have an operation to allow a worker to submit its final subtotal, and
an operation to get the overall total.

%%%%%%

\begin{figure}
\begin{scala}
  /** The bag-of-tasks object. */
  trait BagOfTasks{
    /** Get a task.  Returns £null£ if there are no more tasks. */
    def getTask(): Task 
  }  

  /** A collector object that combines subresults from the workers. */
  trait Collector{
    /** Add £x£ to the result. */
    def add(x: Double) 

    /** Get the result, once the computation has finished. */
    def get: Double 
  }
\end{scala}
\caption{The interfaces for the bag-of-tasks and collector objects.}
\label{fig:BoT-interfaces}
\end{figure}

%%%%%

\begin{figure}
\begin{scala}
  private def worker(bag: BagOfTasks, collector: Collector) = thread("worker"){
    var myTotal = 0.0; var done = false
    while(!done) bag.getTask() match{
      case null => done = true
      case (left, right, taskSize, delta) =>
        myTotal += integral(left, right, taskSize, delta)
    }
    collector.add(myTotal)
  }

  def apply(): Double = {
    val bag = new BagOfTasksChannels(a, b, n, nTasks, nWorkers, buffering)
    val collector = new CollectorChannels(nWorkers, buffering)
    val workers = || (for (i <- 0 until nWorkers) yield worker(bag, collector))
    run(workers)
    collector.get
  }
\end{scala}
\caption{Adapting the code to use the bag-of-tasks and collector objects.}
\label{fig:BOT-using-objects}
\end{figure}
%\end{mysamepage}

It is straightforward to adapt the definition of a worker to use such objects:
see Figure~\ref{fig:BOT-using-objects}.  The definition is parameterised by
the |BagOfTasks| and |Collector|.  It calls the |getTask| operation on the
|BagOfTasks| to get a task, and pattern matches on it (see Scala
box~\ref{sb:pattern-matching}), exiting its main loop when it receives |null|.
At the end, it then uses the |add| operation on the |Collector| to return its
subtotal.

Below we will define an implementation |BagOfTasksChannels| of |BagOfTasks|,
and an implementation |CollectorChannels| of |Collector|.  The main |apply|
method (Figure~\ref{fig:BOT-using-objects}) creates the two objects, runs the
workers, and then gets the final result from the collector.

%%%%%

An implementation of |BagOfTasks| using message passing is in
Figure~\ref{fig:bagOfTasksObject}.  This contains a server thread that sends
|Task|s on the private |toWorkers| channel.  Once all proper tasks have been
sent, it sends |null| multiple times, once for each worker.  A worker can
obtain a task (either a proper task or |null|) by receiving on |toWorkers|.
Note that |toWorkers| is a private channel within the object: this ensures
that client code can interact with the object only as intended.

%%%%%

\begin{figure}
\begin{scala}
class BagOfTasksChannels(
  a: Double, b: Double, n: Long, nTasks: Int, numWorkers: Int, buffering: Int)
    extends BagOfTasks{
  /** Channel from the server to the workers, to distribute tasks. */
  private val toWorkers = mkChan[Task](buffering)

  /** Get a task.  Return £null£ if there are no more tasks. */
  def getTask(): Task = toWorkers?() 

  /** A server process that distributes tasks. */
  private def server = thread("bag of tasks"){
    val delta = (b-a)/n; var remainingIntervals = n; var left = a
    for(i <- 0 until nTasks){
      val taskSize = ((remainingIntervals-1) / (nTasks-i) + 1).toInt
      remainingIntervals -= taskSize; val right = left+taskSize*delta
      toWorkers!(left, right, taskSize, delta); left = right
    }
    for(i <- 0 until numWorkers) toWorkers!null
  }

  fork(server)  // Start the server running.
}
\end{scala}
\caption{An implementation of {\scalashape BagOfTasks} using message passing.}
\label{fig:bagOfTasksObject}
\end{figure}

%%%%%

The server is started by the command |fork(server)|, which is executed as part
of the construction of the |BagOfTasksChannels| object.
%
More generally, if |t| is a |ThreadGroup|, then |fork(t)| (or |t.fork()|)
starts a new thread (or threads) running which executes~|t|.  However---unlike
with |run(t)|---the new thread runs in parallel to the current thread.  In
Figure~\ref{fig:bagOfTasksObject}, the construction of the
|BagOfTasksChannels| object returns after the |fork(server)|, but leaves a
thread running the |server| code.

%%%%% \heading{Encapsulating the collector}

An implementation of |Collector| using message passing is in
Figure~\ref{fig:collector}.  This again contains a server thread, which is set
running as part of the construction of the object.  The |add(x)| method
sends~|x| to the server, which adds it to its running total.  The |get|
operation receives the final result from the server.

%%%%%

\begin{figure}
\begin{scala}
  private class CollectorChannels(buffering: Int) extends Collector{
    /** Channel from workers to the server. */
    private val toController = mkChan[Double](buffering)

    /** Channel that sends the final result. */
    private val resultChan = new SyncChan[Double]

    /** Add £x£ to the result. */
    def add(x: Double) = toController!x

    /** Get the result. */
    def get: Double = resultChan?()
    
    /** A collector, that accumulates the subresults. */
    private def server = thread("collector"){
      var result = 0.0
      for(i <- 0 until nTasks) result += (toController?())
      resultChan!result
    }

    fork(server)    // Start the server running.
  }
\end{scala}
\caption{An implementation of {\scalashape Collector} using message passing.}
\label{fig:collector}
\end{figure}

%%%%%


There is a slight subtlety in the implementation of the collector.  Recall
that the main |apply| operation calls |get| after all the workers have
terminated.  It would be incorrect to make |result| an object variable, and
for the |get| method to just read that variable, because that would constitute
a race.  It is possible that the |server| thread has received a subtotal
from the final worker thread, but not yet added it to |result|, so the |apply|
operation would receive the wrong value!  With the code
in~\ref{fig:collector}, we can be sure that the correct value is sent: the
|get| operation is blocked until the server has received all the subtotals.

