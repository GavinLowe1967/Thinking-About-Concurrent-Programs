\section{Using Locking to Protect an Array of Counters}

We now consider another example, namely an array of counters, with operations
to increment, decrement, or get the value of a particular counter, as captured
by the abstract class |CounterArray| in Figure~\ref{fig:CounterArray}.  In
particular, we consider how best to avoid this object acting as a bottleneck.

%%%%%

\begin{figure}[bhtp]
\begin{scala}
/** An array of £n£ thread£-£safe counters. */
abstract class CounterArray(n: Int){
  /** The counters. */
  protected val a = new Array[Int](n)

  /** Increment counter £i£. */
  def inc(i: Int): Unit

  /** Decrement counter £i£. */
  def dec(i: Int): Unit

  /** Get the value of counter £i£. */
  def get(i: Int): Int
}
\end{scala}
\caption{Abstract class for an array of counters.}
\label{fig:CounterArray}
\end{figure}

%%%%%

The class |CoarseGrainedCounterArray| in
Figure~\ref{fig:CoarseGrainedCounterArray} gives a simple implementation using
a single lock.  Each operation is performed under mutual exclusion using that
lock.  This approach is known as \emph{coarse-grained locking}.  The obvious
disadvantage of this approach is that the class might act as a bottleneck: a
single thread can perform an operation at a time, so other threads might be
blocked.

\begin{figure}
\begin{scala}
/** An implementation using coarse£-£grained locking. */
class CoarseGrainedCounterArray(n: Int) extends CounterArray(n){
  /** Lock that protects all the counters. */
  private val lock = new Lock

  def inc(i: Int) = lock.mutex{ a(i) += 1 }

  def dec(i: Int) = lock.mutex{ a(i) -= 1 }

  def get(i: Int) = lock.mutex{ a(i) }
}

/** An implementation using fine£-£grained locking. */
class FineGrainedCounterArray(n: Int) extends CounterArray(n){
  /** Locks.  £locks(i)£ protects £a(i)£. */
  private val locks = Array.fill(n)(new Lock)

  def inc(i: Int) = locks(i).mutex{ a(i) += 1 }

  def dec(i: Int) = locks(i).mutex{ a(i) -= 1 }

  def get(i: Int) = locks(i).mutex{ a(i) }
}

\end{scala}
\caption{Implementations of an array of counters using coarse-grained and
  fine-grained locking.}
%% \caption{An implementation of an array of counters using coarse-grained
%%   locking.}
\label{fig:CoarseGrainedCounterArray}
\label{fig:FineGrainedCounterArray}
\end{figure}

%%%%%

The class |FineGrainedCounterArray| in
Figure~\ref{fig:FineGrainedCounterArray} gives better performance.  The
crucial observation is that operations on \emph{different} counters can be
executed concurrently, because they will not interfere.  To achieve this, we
use an array of |n| |Lock|s, where each lock |locks(i)| protects the
entry~|a(i)|.  Each operation uses the relevant lock to ensure that two
executions do not act on the same counter concurrently, which would be a race.
However, operations on different counters use different locks, so can proceed
concurrently.  This approach is known as \emph{fine-grained locking}.

%%%%%

\begin{figure}
\begin{scala}
/** An implementation using striped locking. */
class StripedCounterArray(n: Int, stripes: Int) extends CounterArray(n){
  require(stripes > 0)

  /** Locks.  £locks(j)£ protects all £a(i)£ such that £i\%stripes = j£. */
  private val locks = Array.fill(stripes)(new Lock)

  def inc(i: Int) = locks(i%stripes).mutex{ a(i) += 1 }

  def dec(i: Int) = locks(i%stripes).mutex{ a(i) -= 1 }

  def get(i: Int) = locks(i%stripes).mutex{ a(i) }
}
\end{scala}
\caption{Implementation of an array of counters using  striped locking.}
%% \caption{Implementations of an array of counters using fine-grained locking
%%   and striped locking.}
\label{fig:StripedCounterArray}
\end{figure}

%%%%%

A potential disadvantage of fine-grained locking is that it might require many
locks, and so consume a lot of memory.  Each SCL |Lock| uses 24 bytes, which
on its own is fairly trivial; but with many locks, this can mount up.  

The class |StripedCounterArray| in Figure~\ref{fig:StripedCounterArray} uses
an in-between approach.  It uses some number |stripes| of locks, where
typically |stripes| will be much smaller than~|n|.  Each lock protects
multiple counters.  More precisely, |locks(j)| protects each |a(i)| such that
$\sm i \% \sm{stripes} = \sm{j}$; equivalently, each |a(i)| is protected by
\SCALA{locks(i\%stripes)}.

%%%%%

%% \begin{figure}
%% \begin{scala}
%% /** An implementation using striped locking. */
%% class StripedCounterArray(n: Int, stripes: Int) extends CounterArray(n){
%%   require(stripes > 0)

%%   /** Locks.  £locks(j)£ protects all £a(i)£ such that £i\%stripes = j£. */
%%   private val locks = Array.fill(stripes)(new Lock)

%%   def inc(i: Int) = locks(i%stripes).mutex{ a(i) += 1 }

%%   def dec(i: Int) = locks(i%stripes).mutex{ a(i) -= 1 }

%%   def get(i: Int) = locks(i%stripes).mutex{ a(i) }
%% }
%% \end{scala}
%% \caption{Implementations of an array of counters using striped locking.}
%% \label{fig:StripedCounterArray}
%% \end{figure}

%%%%%

This style is known as \emph{striped locking}.  Imagine allocating a different
colour to each lock, and then drawing a picture of the array where each entry
is coloured according to its lock.  Then this picture would have a regular
striped pattern. 

Suppose we take |stripes| to be somewhat larger than the number of threads,
say 10 or 100 times larger.  In most applications, it is reasonable to suppose
that different threads access counters independently from one another.  In
these circumstances, most times that a thread tries to obtain a lock, no other
thread will be holding that lock, and so the thread will not be blocked.  Thus
striped locking will be nearly as fast as fine-grained locking, but without
such a large memory overhead.  

\begin{instruction}
Study the three implementations, and make sure you understand the differences.
\end{instruction}

%%%%%%%%%%%%%%%%%%%%%%%%%%%%%%%%%%%%%%%%%%%%%%%%%%%%%%%%%%%%

\section{Sharding}
\label{sec:sharding}

In this section we  consider how to implement a concurrent set, say
containing elements of type~|A|.  We will give an implementation that extends
the trait in Figure~\ref{fig:Set-trait}, which includes the three main
operations provided by sets in the Scala API{}.  However, it would be
straightforward to add other operations.

%%%%%

\begin{figure}
\begin{scala}
/** Trait for a set containing elements of type A. */
trait Set[A]{
  /** Does this contain x? */
  def contains(x: A): Boolean

  /** Add x to this.  Return true if x was not already in the set. */
  def add(x: A): Boolean 

  /** Remove x from this.  Return true if x was previously in the set. */
  def remove(x: A): Boolean
}
\end{scala}
\caption{The interface for a set.}
\label{fig:Set-trait}
\end{figure}

%%%%%

One possible approach is to encapsulate a sequential set, say a |HashSet[A]|
from the Scala API{}.  (We will assume some familiarity with hashsets below.)
We could then use coarse-grained locking, with a single lock, where every
operation is performed under mutual exclusion using that lock.  This is a
straightforward approach, and would be adequate in many applications.
However, in other applications it might prove to be a bottleneck.  A pragmatic
approach might be to use this approach in a prototype, and to perform
profiling to see how well it works in practice.  In this section, we study a
different approach, using a finer granularity of locking, analagous in some
ways to striped locking.

Our approach is to split the set up into some number |shards| of disjoint
sets, known as \emph{shards}, held in an array.  The object represents the
union of these sets.  Each set can be protected using its own lock, so as to
provide thread safety.  We will use a function
%
\begin{scala}
  /** The shard in which £x£ is stored. */ 
  private def shardFor(x: A): Int = ...
\end{scala}
%
that gives the index in $\interval{\sm 0}{\sm{shards}}$ of the shard in
which~|x| is stored.  Thus all operations regarding a particular~|x| need only
access this single shard.  

%%%%%

\begin{figure}
\begin{scala}
/** A sharded set containing elements of type A, using `shards` shards. */
class ShardedSet[A](shards: Int) extends Set[A]{
  require(shards > 1)

  /** The amount to shift hash codes to obtain the index of the relevant
    * shard. */
  private val shift = 32-Sharding.logShards(shards)

  /** The shard in which £x£ is stored. */ 
  private def shardFor(x: A) = Sharding.improve(x.hashCode) >>> shift

  /** The shards.  This £ShardedSet£ object represents the union of £sets£. */ 
  private val sets = Array.fill(shards)(new scala.collection.mutable.HashSet[A])

  /** Locks to protect £sets£: £locks(i)£ protects £sets(i)£. */
  private val locks = Array.fill(shards)(new Lock)

  /** Does this contain £x£? */
  def contains(x: A): Boolean = {
    val s = shardFor(x); locks(s).mutex{ sets(s).contains(x) }
  }

  /** Add £x£ to this.  Return £true£ if £x£ was not already in the set. */
  def add(x: A): Boolean = {
    val s = shardFor(x); locks(s).mutex{ sets(s).add(x) }
  }

  /** Remove £x£ from this.  Return £true£ if £x£ was previously in the set. */
  def remove(x: A): Boolean = {
    val s = shardFor(x); locks(s).mutex{ sets(s).remove(x) }
  }
}
\end{scala}
\caption{A sharded set.}
\label{fig:ShardedSet}
\end{figure}

%%%%%

Figure~\ref{fig:ShardedSet} gives an implementation using this idea, using a
|HashSet| for each shard.  Our implementation abstracts away from how
operations are performed on the individual shards, for example issues such as
resizing a hash table, delegating these to the |HashSet| objects.
Figure~\ref{fig:Sharding} gives some supporting code
(Exercise~\ref{ex:sharded-map} asks you to implement a mapping using sharding;
some of this supporting code will also be useful there).

Each of the three public operations acts in a similar way: it calculates the
index for the shard, obtains the relevant lock, and then performs the
corresponding operation on the relevant shard.  

%%%%%

\begin{figure}
\begin{scala}
/** Useful functions supporting sharding. */
object Sharding{
  /** Improve a hash code. */
  def improve(hcode: Int): Int = {
    var h = hcode + ~(hcode << 9)
    h = h ^ (h >>> 14); h = h + (h << 4); h ^ (h >>> 10)
  }

  /** The log of shards.  Check this is a power of 2. */
  def logShards(shards: Int) = {
    var s = shards; var i = 0
    while(s > 1){ s = s >> 1; i += 1 }
    require(shards == 1 << i, 
      s"The number of shards should be a power of 2, received $shards.")
    i
  }
}
\end{scala}
\caption{Some functions supporting sharding.}
\label{fig:Sharding}
\end{figure}

%%%%%

The definition of |shardFor(x)| needs some explanation.  An obvious approach
is to base the definition upon the hashcode for~|x|.  However, as we are using
|HashSet|s to implement the shards, we want to try to ensure that the way we
use hashcodes in |shardFor| is independent from the way the |HashSet|s use the
hashcodes: otherwise the values in each |HashSet| might be unevenly
distributed, giving poor behaviour.

The Scala API |HashSet|s use the least significant bits of the hashcode.  We
therefore choose to use the \emph{most} significant bits of the hashcode
within the |shardFor| function.  We require that |shards| is equal to $2^n$
for some~$n$: the function |Sharding.logShards| checks this, and finds $n$.
The $n$ most-significant bits can be extracted by right-shifting $\sm{shift} =
32-n$ places.  However, most standard recipes for constructing hashcodes tend
to concentrate the entropy in the least-significant bits: this risks
distributing values unevenly between different shards.  Instead, we use the
function |Sharding.improve| to improve the hashcode, distributing the entropy
between different bits (this uses a standard technique; there is no need to
understand it).

\begin{instruction}
Study the details of the implementation.
\end{instruction}

As with striped locking, if we choose the number of shards to be somewhat
larger than the number of threads, then under reasonable assumptions we can
expect threads normally to operate on different shards, and so rarely have to
wait to obtain a lock.

We can test the implementation using linearizability testing, using an
immutable set as the sequential specification object.  The values added to the
set should be from a fairly small range so that the same value is used in
multiple operation calls.  The code is on the book website.

