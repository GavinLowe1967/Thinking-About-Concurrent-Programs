\section{Types}
\label{sec:scala-types}

All Scala types are either value types (Scala box~\ref{sb:types}) or reference
types (Scala box~\ref{sb:reference-types}).  Scala supports parametric
polymorphism (Scala box~\ref{sb:polymorphic}) and function types (Scala
box~\ref{sb:function-types}).

%%%%%

\begin{scalaBox}{Reference types}
\label{sb:reference-types}
Each class defines a type with the same name, which is a \emph{reference
  type}.  Each variable of that type actually holds a reference to the object,
i.e.~the address in memory where the object is stored.  
%% References are sometimes called \emph{pointers}.
Likewise, when a function takes a parameter
of a reference type, the value that is actually passed is the object's
reference.
\end{scalaBox}

%%%%%

\begin{scalaBox}{Parametric polymorphism}
\label{sb:polymorphic}
Scala classes can be defined to be polymorphic, i.e.~their definition takes a
type parameter~|A|, and the class is defined for all values of~|A|.  For
example, the type |Array[A]| is the type of arrays that hold data of type~|A|:
we can instantiate~|A| with an arbitrary type.

Functions can also be defined in a polymorphic way, for example:
\begin{scala}
  /** Does £a£ contain the value £x£? */
  def contains[A](a: Array[A], x: A): Boolean = ...
\end{scala}
%
Here, the initial ``|[A]|'' introduces~|A| as a polymorphic type parameter.
The function is defined for all types~|A|. 
\end{scalaBox}

%%%%%

\begin{scalaBox}{Function types}
\label{sb:function-types}
If |A| and |B| are types, then \SCALA{A => B} is the type of functions
from~|A| to~|B|, i.e.~that take an argument of type~|A| and return a result of
type~|B|.  For example, \SCALA{Int => Boolean} is the type of functions from
|Int| to |Boolean|.
\end{scalaBox}

%%%%%

\pagebreak[3]

Scala provides for the definition of type synonyms (Scala
box~\ref{sb:type-synonym}). 

\begin{scalaBox}{Type synonyms}
\label{sb:type-synonym}
Type synonyms can be defined using the keyword |type|, for example
\begin{scala}
  type Price = Int
  type Log1 = Log[LogEvent]
\end{scala}
Such type synonyms can help to distinguish between different uses of a common
type, or to provide brevity. 
\end{scalaBox}

\pagebreak[3]

The Application Programming Interface (API) contains a large number of useful
types, including arrays (Scala box~\ref{sb:arrays}), strings (Scala
box~\ref{sb:strings}), lists (Scala box~\ref{sb:lists}), and |Option| types
(Scala box~\ref{sb:option-type}).  For other API classes, you should refer to
the online documentation (e.g.~by searching for ``Scala
API''; recall that this book uses Scala~2.13).


\begin{scalaBox}{Lists}
\label{sb:lists}
The type |List[A]| represents the type of immutable lists containing data of
type~|A|.

A list containing values~|x|, |y| and~|z| can be defined as \SCALA{List[A](x,
  y, z)}; the type parameter~|A| can normally be omitted, as the compiler can
infer it.

If |xs| has type |List[A]|, and |x| has type~|A|, then |x :: xs| is a new
|List[A]| containing~|x| followed by the elements of~|xs|.  The expression
|xs.isEmpty| returns a |Boolean| indicating whether |xs| is empty, and
|xs.length| gives the length of~|xs|.  The element at index~|k| (counting
from~0) can be obtained as |xs(k)|; this traverses the list, so takes
time~$O(\sm k)$.  If |xs| is not empty, then |xs.head| gives its first
element, and |xs.tail| gives the list containing all of~|xs| except the first
element.  The expression |xs ++ ys| gives the concatenation of~|xs| and~|ys|. 
\end{scalaBox}

%%%%%

\begin{scalaBox}{The {\scalashape Option} type}
\label{sb:option-type}
The type |Option[A]| contains values of two subtypes:
%
\begin{itemize}
\item values of the form |Some(x)| where |x| has type~|A|;

\item the value~|None|.
\end{itemize}

The |Option| type is useful in cases where a function might or might not be
able to return a proper result: it can return a result |Some(x)| to indicate
that it succeeded, with result~|x|; or it can return |None| to indicate that
it was unsuccessful.

|Some(x)| and |None| can be used as patterns in pattern matching (Scala
box~\ref{sb:pattern-matching}).  The pattern ``|Some(x)|'' (where |x| is a
variable) matches any value of the form |Some(v)|, and binds the variable~|x|
to the value~|v|.  The pattern ``|None|'' matches just the value~|None|.
\end{scalaBox}

%%%%%%%%%%%%%%%%%%%%%%%%%%%%%%%%%%%%%%%%%%%%%%%%%%%%%%%



%% \section{{\scalashape for} expressions}
%% \label{app:for-expressions}

%% \section{Lists}
%% \label{sec:scala-lists}

%% The type |List[A]| represents the type of immutable lists containing data of
%% type~|A|.

%% A list containing values~|x|, |y| and~|z| can be defined as \SCALA{List[A](x,
%%   y, z)}; the type parameter~|A| can normally be omitted, as the compiler can
%% infer it.

%% If |xs| has type |List[A]|, and |x| has type~|A|, then |x :: xs| is a new
%% |List[A]| containing~|x| followed by the elements of~|xs|.  The expression
%% |xs.isEmpty| returns a |Boolean| indicating whether |xs| is empty.  If |xs| is
%% not empty, then |xs.head| gives its first element, and |xs.tail| gives the
%% list containing all of~|xs| except the first element.

%\section{Option types}
