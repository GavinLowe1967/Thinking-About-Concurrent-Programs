
\begin{slide}
\heading{Example: a monitor to enforce mutual exclusion}

Often client processes need to access a shared resource under \emph{mutual
  exclusion}: only one client may use the resource at a time.  We will
implement a monitor to enforce mutual exclusion, but also to allow clients
to access the resource in first-come-first-served order.  

When a client wishes to access the resource it calls a \SCALA{Request}
procedure (known as the \emph{entry protocol}), passing in its identity; this
will block until this client is allowed access.  After using the resource, the
client calls a \SCALA{Release} procedure (known as the \emph{exit protocol}).
So the code for a client will look like
%
\begin{scala}
...
Monitor.Request(me) // entry protocol
...                  // use resource
Monitor.Release     // exit protocol
...
\end{scala}
\end{slide}

%%%%%

\begin{slide}
\heading{First attempt}

The following version enforces mutual exclusion, but not FIFO behaviour.
%
\begin{scala}
object Monitor{
  private var busy = false // Is the resource in use?

  def Request(id: ProcId) = synchronized{
    while(busy) wait() // wait turn
    busy = true
  }

  def Release = synchronized{
      busy = false
      notify()
  }
}
\end{scala}
\end{slide}

%%%%%

\begin{slide}
\heading{Enforcing FIFO}

We use a queue to store the identities of waiting processes:
%
\begin{scala}
val queue = new scala.collection.mutable.Queue[ProcId];
\end{scala}

If a process cannot access the resource immediately, its identity is put on
the queue and it waits.  When it is woken up, it checks whether it has got to
the front of the queue.
%
\begin{scala}
def Request(id: ProcId) = synchronized{
  if(busy || !queue.isEmpty){
    queue.enqueue(id);
    while(busy || queue.first!=id) wait(); // wait turn
    val first = queue.dequeue;
    assert(first==id);
  }
  busy = true;
}
\end{scala}
\end{slide}

%%%%%

\begin{slide}
\heading{Enforcing FIFO}

The \SCALA{Release} method is as before.

If processes do not have identities, the monitor can give each process a
unique sequence number to act as its identity:
%
\begin{scala}
private var next = 0; // next sequence number to use

def Request = synchronized{
  val id = next; next += 1;
  ... // as before
}
\end{scala}
\end{slide}
