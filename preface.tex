\chapter*{Preface}
\addcontentsline{toc}{section}{Preface}
\markboth{Preface}{} % Else it says "list of Scala boxes!"

\framebox{Needs writing!}


Prerequisites.
%
\begin{itemize}
\item Good programming skills using Scala. 

%% \item Basic understanding of object-oriented programming (objects, classes,
%% interfaces, inheritance, abstract classes, polymorphism), and comfort with
%% programming in Scala.

\item Familiarity with standard techniques for reasoning about (sequential)
programs:  pre- and post-conditions, invariants, and abstraction functions.
\end{itemize}

A major theme of the book will be modularisation.  Often, we will aim to
encapsulate design decisions concerning concurrency within an object.
Operations on that object will be thread-safe, i.e.~different threads will not
interfere with one another.  Client code can use that object without knowing
how the concurrency is implemented.  A common example will be that the object
implements a datatype, such as a queue or a mapping.  Client code can use the
concurrent datatype much as they would a corresponding sequential datatype. 

Another major theme will be testing.  How can we test concurrent programs or
concurrent modules?  Often this requires different techniques from those for
sequential programs or modules. 

\framebox{Code availability}

\begin{instruction}
I will sometimes leave you to study some details of programs for yourself.  A
paragraph labelled with a triangle on the left, like this, is a nudge for you
to do that.
\end{instruction}

%%%%%%%%%%%%%%%%%%%%%%%%%%%%%%%%%%%%%%%%%%%%%%%%%%%%%%%

\subsection*{Acknowledgements}



SCL heavily based on CSO (Communicating Scala Objects), produced by Bernard
Sufrin.

Karel.
