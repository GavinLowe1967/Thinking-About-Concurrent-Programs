\section{Example: Bag of Tasks with Replacement}
\label{sec:quicksort-bag}

In Chapter~\ref{chap:trapezium} we say the bag-of-tasks pattern, where workers
repeatedly obtain a task, and perform it.  We now consider a variant, where
workers can return subtasks to the bag.  Thus, if a worker obtains a
task~$t$, it will perform some work itself, and (maybe) return some subtasks to
the bag; these together must equate to completing~$t$.

In particular, we will use the bag-of-tasks with replacement pattern to
produce a concurrent implementation of the Quicksort sorting algorithm.
Figure~\ref{fig:quicksort-seq} gives a sequential implementation of the
algorithm.  

%%%%%

\begin{figure}
\begin{scala}
/** Trait implemented by different implementations of Quicksort. */
trait QuicksortT{
  protected val a: Array[Int]

  /** Partition £$\sm a\interval{\sm l}{\sm r}$£, returning the index of the pivot. */
  protected def partition(l: Int, r: Int): Int = {
    assert(r-l > 1)
    val x = a(l); var i = l+1; var j = r
    while(i < j){
      while(i < j && a(i) <= x) i += 1
      while(i < j && a(j-1) > x) j -= 1
      if(i < j){ // £$\sm a(\sm i) > \sm x \land \sm a(j\sm -1) \le \sm x$£, so swap.
        val t = a(i); a(i) = a(j-1); a(j-1) = t 
      }
    }
    // Swap pivot into position £i-1£, and return that index.
    a(l) = a(i-1); a(i-1) = x; i-1
  }
}

/** Sequential implementation of Quicksort. */
class Quicksort(protected val a: Array[Int]) extends QuicksortT{
  /** Sort a. */
  def sort() = qsort(0, a.length)

  /** Sort the segment a[l..r). */
  private def qsort(l: Int, r: Int): Unit =
    if(r-l > 1){ val m = partition(l, r); qsort(l,m); qsort(m+1,r) }
}
\end{scala}
\caption{A sequential implementation of Quicksort.}
\label{fig:quicksort-seq}
\end{figure}

%%%%%

The function |partition(l, r)| partitions the non-empty segment $\sm
a\interval{\sm l}{\sm r}$.  It picks some element~|x| from the segment, and
permutes the elements of the segment so that it looks like this:
\[\mstyle
\begin{array}{rllclcll}
   & 0 & \ms{l} & & \ms{k} &  & \ms{r} & \ms{N} \\ \cline{2-7}
\multicolumn{1}{r\|}{\ms{a}:\;} & \qquad\qquad & 
  \multicolumn{2}{\|c\|}{\quad \le \ms{x} \quad} & 
  \multicolumn{1}{c\|}{\ms{x}} &
  \multicolumn{1}{c\|}{\quad > \ms{x} \quad} &
  \multicolumn{1}{c\|}{\qquad\qquad}  \\ \cline{2-7}
\end{array}
\]
More precisely, it returns an index~|k| such that
\[\mstyle
\sm a\interval{\sm l}{\sm k} \le \sm a(\sm k) < 
  \sm a\interval{\sm k+1}{\sm r} 
\land \sm l \le \sm k < \sm r.
\]

The implementation of~|partition| chooses the first element |a(l)| as the
pivot~|x|.  The main loop keeps |x| in position~|l|; it is swapped into the
correct position only at the end.  The loop maintains the invariant
\[\mstyle
\sm a\interval{\sm l+1}{\sm i} \le \sm x = \sm a(\sm l) < 
  \sm a\interval{\sm j}{\sm r} 
\land \sm l < \sm i \le \sm j \le \sm r,
\]
which we can picture as follows.
\[\mstyle
\begin{array}{rllclcllll}
   & 0 & \ms{l} & \ms{l+1} & & \ms{i} &  & \ms{j} & \ms{r} & \ms{N} 
\\ \cline{2-9}
\multicolumn{1}{r\|}{\ms{a}:\;} & \qquad\qquad & 
  \multicolumn{1}{\|c\|}{\ms{x}} &
  \multicolumn{2}{c\|}{\quad \le \ms{x} \quad} & 
  \multicolumn{2}{c\|}{\quad?\quad} &
  \multicolumn{1}{c\|}{\quad > \ms{x} \quad} &
  \multicolumn{1}{c\|}{\qquad\qquad}  \\ \cline{2-9}
\end{array}
\]
(We include the |partition| function in the |QuicksortT| trait, to avoid
repeating code; each of our implementations will extend this trait.)

Most of the work is done by the function |qsort(l, r)|, which sorts the
segment $\sm a\interval{\sm l}{\sm r}$.  If the interval contains at most one
element, then it is already sorted.  Otherwise, the function calls partition,
and then recursively sorts the two subintervals.  The main |sort| function
calls |qsort| on the whole array.

\begin{instruction}
Make sure you understand the code in Figure~\ref{fig:quicksort-seq}.
\end{instruction}

%%%%%%%%%%

\subsection{Basic Concurrent Implementation}

We will create two concurrent implementations of the Quicksort algorithm, each
based upon the bag-of-tasks pattern.  The implementation in this section will
be straightforward, but rather inefficient; we will produce a more efficient
implementation in the next section.

Each task will be represented by a pair |(l,r)|, and will represent the task
of sorting the interval $\sm a\interval{\sm l}{\sm r}$, rather like the
earlier function |qsort(l, r)|.  We define
\begin{scala}
  type Task = (Int, Int)
\end{scala}
%
If the interval $\sm a\interval{\sm l}{\sm r}$ contains at least two elements,
the worker will partition it, and then return to the bag two subtasks
corresponding to the two subintervals.  Once these subtasks are completed, the
interval $\sm a\interval{\sm l}{\sm r}$ will be sorted.

We need to think about termination.  The whole array will be sorted when the
bag is holding no tasks, and no worker is working on a task.  Thus the server
implementing the bag needs to keep track of how many workers are working on
tasks.  It can increment this count when it issues a task, and can decrement
the count when a worker returns two subtasks.  However, when a worker receives
a task containing at most one element, the worker will return no subtasks, so
it needs a different way to indicate that the task is complete.

We encapsulate the bag of tasks in an object~|Bag|, as in
Figure~\ref{fig:quicksort-bag-1}.   The operation |get| allows a worker to get
a task.  The operation |add| allows a worker to return two subtasks to the
bag.  The operation |done| allows a worker to signal that it has completed a
task (without returning subtasks).  

%%%%%%%%%%

\begin{figure}
\begin{scala}
  object Bag{
    private val getC = new SyncChan[Task]
    private val addC = new SyncChan[(Task,Task)]
    private val doneC = new SyncChan[Unit]

    /** Get a £Task£. */
    def get(): Task = getC?()

    /** Add £Task£s £t1£ and £t2£ to the bag. */  
    def add(t1: Task, t2: Task): Unit = addC!(t1,t2)

    /** Indicate that the current thread has completed its last £Task£. */
    def done(): Unit = doneC!()

    private def server = thread("server"){
      val queue = new scala.collection.mutable.Queue[Task]
      queue.enqueue((0, a.length)); var busyWorkers = 0
      serve(
        queue.nonEmpty && getC =!=> { busyWorkers += 1; queue.dequeue() }
        | busyWorkers > 0 && addC =?=> { case (t1,t2) => 
            queue.enqueue(t1); queue.enqueue(t2); busyWorkers -= 1 }
        | busyWorkers > 0 && doneC =?=> { _ => busyWorkers -= 1 }
      )
      assert(queue.isEmpty && busyWorkers == 0)
      getC.endOfStream()
    }

    fork(server)
  } 
\end{scala}
\caption{The bag of tasks in the basic concurrent implementation of Quicksort.}
\label{fig:quicksort-bag-1}
\end{figure}

%%%%%%%%%%

The |Bag| object encapsulates a server thread.  Each operation is implemented
by a single communication between the worker and the server.  Note, in
particular, that the |add| operation sends both subtasks in a single message,
rather than two separate messages: this roughly halves the cost of the
operation. 

The server itself stores the pending tasks in a queue (although we could have
used a different collection class, such as a stack).  It also uses a variable
|busyWorkers| to record the number of workers currently working on a task.
Most of the implementation is then fairly straightforward.  The main |serve|
loop terminates when all the guards are false, namely when the queue is empty
and $\sm{busyWorkers} = 0$: this is the termination condition we identified
earlier.  Note in particular how that second and third branches each has a
guard |busyWorkers > 0| in order to achieve this termination condition: and we
should certainly expect this condition to hold whenever a worker tries to send
on |addC| or |doneC|.  When the |serve| loop terminates the server performs
|endOfStream| on the |getC| channel, to signal to the workers that there are
no more tasks to complete.
%
\begin{instruction}
Make sure you understand the implementation of |Bag|.
\end{instruction}

The main implementation is in Figure~\ref{fig:quicksort-bag-of-tasks-1}.  Each
worker repeatedly gets a task from the bag.  If the task contains more than
one element, the worker partitions it, and returns the two subtasks to the
bag.  Otherwise, it informs the bag that the task is complete via the |done|
operation.  The worker terminates when the |get| operation throws a |Stopped|
exception, which corresponds to the termination condition described earlier.
The main |sort| function simply runs |numWorkers| workers.  

%%%%%%%%%%

\begin{figure}
\begin{scala}
/** Basic concurrent implementation. */
class ConcurrentQuicksort1(protected val a: Array[Int], numWorkers: Int)
    extends QuicksortT{
  type Task = (Int, Int)

  object Bag{
    £\ldots£ // as in Figure £\ref{fig:quicksort-bag-1}£.
  }

  /** A single worker. */
  private def worker = thread("worker"){
    repeat{
      val (l,r) = Bag.get()
      if(r-l > 1){ val m = partition(l, r); Bag.add((l,m), (m+1,r)) }
      else Bag.done()
    }
  }

  /** Sort £a£. */
  def sort() = run(|| (for(i <- 0 until numWorkers) yield worker))
}
\end{scala}
\caption{The basic implementation of Quicksort using a bag of tasks.}
\label{fig:quicksort-bag-of-tasks-1}
\end{figure}

%%%%%

The implementation can be tested by checking that it produces the same results
as the sequential implementation from Figure~\ref{fig:quicksort-seq}.  We can
create a random array of |Int|s, clone it, run the two algorithms on the two
copies, and check the two results contain the same elements.  This can be
repeated many times.

Unfortunately, this implementation performs poorly.  On an array containing a
million values, it is about 100 times slower than the sequential algorithm.
The main reason for this is that there are simply too many messages sent over
channels, and this creates a very large overhead.  For example, consider a
subtask containing just two elements; this will lead to one message issuing
that task, two messages returning the two subtasks, two messages issuing those
subtasks, and two messages indicating that those subtasks are done, for a
total of seven messages.  The following subsection seeks to improve this.


%%%%%%%%%%%%%%%%%%%%%%%%%%%%%%%%%%%%%%%%%%%%%%%%%%%%%%%

\subsection{Improved Concurrent Implementation} 

We can improve the implementation in several ways.
%
\begin{enumerate}
\item \label{obs:quicksort-1}
  Rather than returning two subtasks to the bag, a worker could return
  just one, and complete the second task itself.  This means that we save a
  message in each such case.

\item There is no point in returning subtasks to the bag if they contain at
  most one element, since those subtasks require no additional work.  This
  greatly reduces the number of messages corresponding to such subtasks.  We
  call subtasks with at most one element \emph{trivial}. 

\item\label{obs:quicksort-3} If a worker receives a task that is fairly small,
  it is more efficient for the worker to complete the whole task itself, using
  the sequential algorithm, rather than creating new subtasks.
\end{enumerate}

Figure~\ref{fig:quicksort-bag-2} gives a new definition of the bag of tasks.
The only change corresponds to observation~\ref{obs:quicksort-1} above.  Now
the |add| operation returns a single task; the server does not decrement
|busyWorkers| as a result, as the relevant worker is still working on the
other subtask. 

%%%%%%%%%%%%%%%%

\begin{figure}
\begin{scala}
  object Bag{
    private val getC = new SyncChan[Task]
    private val addC = new SyncChan[Task]
    private val doneC = new SyncChan[Unit]

    def get(): Task = getC?()

    def add(t: Task): Unit = addC!t

    def done(): Unit = doneC!()

    private def server = thread("server"){
      val queue = new scala.collection.mutable.Queue[Task]
      queue.enqueue((0, a.length)); var busyWorkers = 0
      serve(
        queue.nonEmpty && getC =!=> { busyWorkers += 1; queue.dequeue() }
        | busyWorkers > 0 && addC =?=> { t => queue.enqueue(t) }
        | busyWorkers > 0 && doneC =?=> { _ => busyWorkers -= 1 }
      )
      assert(queue.isEmpty && busyWorkers == 0)
      getC.endOfStream()
    }

    fork(server)
  } 
\end{scala}
\caption{The bag of tasks in the improved concurrent algorithm.}
\label{fig:quicksort-bag-2}
\end{figure}

%%%%%

The main implementation is in Figure~\ref{fig:quicksort-bag-of-tasks-2}.  For
the purposes of observation~\ref{obs:quicksort-3}, above, we consider a task
to be small if it contains fewer than |Limit| elements, where |Limit| is
defined as the minimum of $20,000$ and $\sm{a.size} / (10 \times
\sm{numWorkers})$.  My informal experiments suggested that for tasks smaller
than about $20,000$, it is faster for the worker to complete the whole task
itself than to create subtasks.  The latter term means that for very large
arrays, each worker can sort larger segments sequentially; the value ensures
that on average each worker sorts slightly fewer than ten segments
sequentially, which seems enough to give reasonable load balancing.  If
out-and-out performance was important, we could run further experiments to
tune these figures.

%%%%%

\begin{figure}
\begin{scala}
class ConcurrentQuicksort2(protected val a: Array[Int], numWorkers: Int)
    extends QuicksortT{
  type Task = (Int, Int)

  object Bag{
    £\ldots£ // As in Figure £\ref{fig:quicksort-bag-2}£.
  }

  /** Upper limit on the size of tasks sorted sequentially. */
  val Limit = 20_000 max (a.size / (10*numWorkers))

  /** Sequential implementation. */
  private def qsort(l: Int, r: Int): Unit =
    if(r-l > 1){ val m = partition(l, r); qsort(l,m); qsort(m+1,r) }

  /** A single worker. */
  private def worker = thread("worker"){
    var l = -1; var r = -1
    repeat{
      if(l == r){ val task = Bag.get(); l = task._1; r = task._2; assert(r-l > 1)}
      }  
      if(r-l < Limit){ qsort(l,r); r = l; Bag.done() }
      else{
        val m = partition(l, r)
        if(m-l > 1){ // Continue with £(l,m)£.
          if(r-(m+1) > 1) Bag.add((m+1,r))
          r = m
        }
        else if(r-(m+1) > 1) l = m+1 // Continue with £(m+1,r)£.
        else{ r = l; Bag.done() } // This segment finished.
      }
    }
  }

  def sort() = if(a.size > 1) run(|| (for(i <- 0 until numWorkers) yield worker))
}
\end{scala}
\caption{The improved implementation of Quicksort using a bag of tasks.}
\label{fig:quicksort-bag-of-tasks-2}
\end{figure}

%%%%%

The worker maintains two variables |l| and~|r|, with $\sm l \le \sm r$.  If
$\sm l < \sm r$, then this worker is still responsible for the segment $\sm
a\interval{\sm l}{\sm r}$.  But if $\sm l = \sm r$, the worker has no current
task.  In the latter case, the worker obtains a new task from the bag, and
unpacks it.  In either case, if the task is smaller than |Limit|, it sorts it
sequentially, sets~|r| to~|l| to record that the task is done, and signals
that fact to the bag.  Otherwise, it partitions the segment, to obtain a pivot
index~|m|.  If the subsegment $\sm a\interval{\sm l}{\sm m}$ is not trivial,
it continues with that subsegment, and returns the task for the other
subsegment to the bag if it is not trivial.  Otherwise, if the subsegment $\sm
a\interval{\sm m+1}{\sm r}$ is not trivial, it continues with that subsegment.
Otherwise, the task is complete, so it updates |r| to record that, and signals
that fact to the bag.

%This implementation is far faster than the previous concurrent implementation.
For arrays of size 1,000,000, using eight workers, this implementation is
slightly faster than the sequential implementation, so about 100 times
faster than the previous concurrent implementation.  For arrays of size
10,000,000, it is about four times faster than the sequential
implementation; and for arrays of size 100,000,000, it is about five times
faster.
