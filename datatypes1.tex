A \emph{concurrent datatype} is a datatype, for example, a queue, stack, set
or mapping, that can be safely accessed concurrently by multiple threads.
Operation executions should appear to take place in a one-at-a-time order,
without interference.

Threads that use the concurrent datatype can act much as they would with a
sequential datatype: the implementer of the threads does not need to think
much about concurrency.  

The implementer of the concurrent datatype \emph{does} have to think about
concurrency, and ensure different operation calls do not interfere with one
another.  But that concurrency is local, often inside a single object: local
reasoning is much easier than global reasoning.  And there's a lot of scope
for re-using concurrent datatypes.

Datatype-based concurrent programming encapsulates all, or nearly all, the
concurrency within a small number of concurrent datatypes.

As mentioned in Chapter~\ref{chap:clientServer}, operations on concurrent
objects can be either \emph{total}, i.e.~able to be performed in an arbirary
state, or \emph{partial}, i.e.~where the operation blocks if a precondition
does not hold.  The terms total and partial also apply to concurrent
datatypes.

We investigate these ideas in this chapter.  Most of the examples are based
around concurrent queues; exercises at the end of the chapter ask you to adapt
the techniques to concurrent stacks.

We start with a straightforward implementation of a total queue using a server
thread: a dequeue operation returns a special value to indicate that the queue
is empty.  We then ask what it means for such a concurrent datatype to be
correct.  The standard answer is a property called \emph{linearizability}: we
define this property, and show how it can be tested for.  We then implement a
partial queue using a server: a dequeue operation blocks when the queue is
empty; we show how to adapt the linearizability test to this case.  An issue
with this implementation is that if the queue is empty and all the client
threads are attempting to dequeue, the system deadlocks.  We adapt the
implementation to detect this case, and for all the clients to return an
appropriate value.  Finally, we use such a concurrent queue to implement a
concurrent graph search in approximate breadth-first order: much of the
definition of a worker thread is the same as for a sequential search, because
the concurrent queue is responsible for managing much of the concurrency.

%%%%%%%%%%%%%%%%%%%%%%%%%%%%%%%%%%%%%%%%%%%%%%%%%%%%%%%%%%%%

\section{Example: a concurrent queue}
\label{sec:total-queue}

To illustrate some of the ideas, we will implement a concurrent queue.  First,
we should ask: what interface should a concurrent queue have?  A sequential
queue would typically have an interface like:
%
%\begin{mysamepage}
\begin{scala}
trait Queue[T]{
  /** Enqueue £x£. */
  def enqueue(x: T): Unit

  /** Dequeue a value.  Precondition: the queue is not empty. */
  def dequeue(): T

  /** Is the queue empty? */
  def isEmpty: Boolean
}
\end{scala}
%\end{mysamepage}
%
The |dequeue| operation has a non-trivial precondition, so client code that
uses the queue would be expected to check this first, for example, via code
such as:
\begin{scala}
  if(queue.isEmpty){ ... /* Handle the empty queue. */ } 
  else{ val x = queue.dequeue(); ... /* Do something with £x£. */ }
\end{scala}

However, this won't work with multi-threaded code: thread~$t$ might check that
the queue is non-empty; but then other threads might empty the queue before
$t$ attempts the |dequeue|, at which point the precondition is violated.  This
is a time-of-check to time-of-use (TOCTTOU) problem: there is a delay between
the thread checking the relevant precondition and performing the operation that
depends upon it, during  which the precondition becomes false; several
exercises in Chapter~\ref{chap:intro} considered the same issue.

We saw in Chapter~\ref{chap:clientServer} that there are two main ways of
dealing with operations that have a non-trivial precondition.  One way is to
return a special value to indicate that the precondition does not hold.  In
Scala, it is natural to use an |Option| value, with |None| indicating that the
precondition does not hold.  In some circumstances, it might be appropriate
(and more efficient) to use |null| for this, if |null| can never be returned
when the precondition does hold.  An alternative is for the operation to throw
an exception, and expect the thread to catch it.  Recall that operations that
treat preconditions in this way are called \emph{total}: they can take effect
in any state.

%% The other way to deal with the case that the precondition does not hold is to
%% block the thread until the precondition becomes true.  Such operations are
%% called \emph{partial}.

We implement a total concurrent queue, with the following interface,
where the |dequeue| operation returns an |Option| type, with |None| used to
indicate am empty queue. 
%
%\begin{mysamepage}
\begin{scala}
/** A total queue. */
trait TotalQueue[T]{
  /** Enqueue £x£. */
  def enqueue(x: T): Unit

  /** Dequeue a value.  Returns £None£ if the queue is empty. */
  def dequeue(): Option[T]

  /** Shut down the queue. */
  def shutdown: Unit
}
\end{scala}
%\end{mysamepage}

A thread that performs a |dequeue| should handle both cases, for example, via
code such as:
\begin{scala}
  queue.dequeue() match{
    case Some(x) => ... // Do something with £x£.
    case None => ... // Handle the empty queue.
  }
\end{scala}

Figure~\ref{fig:total-queue-server} gives a straightforward implementation of
a total queue that encapsulates a server.  But we could also implement the
queue using one of the techniques we'll see later in the course.  The server
stores the queue itself (using a |Queue| from the Scala API).  Clients use
channels to cause the server to enqueue and dequeue values.  Note that the
server handles operations in a one-at-a-time way, preventing operations from
interfering with one another.  We also include a |shutdown| operation, to
provide a way to terminate the server thread, and so allow garbage collection.

%%%%%

\begin{figure}
\begin{scala}
class ServerTotalQueue[T] extends TotalQueue[T]{
  // Channels used for enqueueing and dequeueing.
  private val enqueueChan = new SyncChan[T]
  private val dequeueChan = new SyncChan[Option[T]]

  def enqueue(x: T) = enqueueChan!x

  def dequeue(): Option[T] = dequeueChan?()

  private def server = thread("ServerTotalQueue"){
    val queue = new scala.collection.mutable.Queue[T]
    serve(
      enqueueChan =?=> { x => queue.enqueue(x) }
      | dequeueChan =!=> { 
          if(queue.nonEmpty) Some(queue.dequeue()) else None 
        }
    )
  }

  fork(server)

  def shutdown() = { enqueueChan.close; dequeueChan.close }
}
\end{scala}
\caption{A total queue implemented using a server.}
\label{fig:total-queue-server}
\end{figure}

%%%%%

\begin{instruction}
Study the details of the implementation.
\end{instruction}


