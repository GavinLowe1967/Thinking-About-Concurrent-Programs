
\section{Disciplined interaction}

The main challenge of concurrent programming is ensuring disciplined
interaction between threads.  Without this, even very simple concurrent
programs can act in unexpected ways.  In particular, it is necessary to ensure
disciplined access to shared variables.

Figure~\ref{fig:race} gives an example that shows what can happen if you allow
undisciplined access to variables.  Such programs are very hard to reason
about.
Appendix~\ref{app:scala} contains a brief introduction to the Scala
programming language.  We will concentrate below on the concurrency aspects of
the program. 

%%%%%

\begin{figure}
\begin{scala}[numbers = left]
import ox.scl._  £\label{line:input}£

/** Program to show the dangers of undisciplined shared variables. */
object Race{
  var x = 0

  /** Thread to increment £x£ 1000 times. */
  def t1 = thread{ for(i <- 0 until 1000) x = x+1 }£\label{line:p}£

  /** Thread to decrement £x£ 1000 times. */
  def t2 = thread{ for(i <- 0 until 1000) x = x-1 }£\label{line:q}£

  /** Parallel composition. */
  def system = t1 || t2£\label{line:system}£

  def main(args: Array[String]) = { run(system); println(x) }£\label{line:main}£
}
\end{scala}% File Race/Race.scala
\caption{A simple program exhibiting a memory race.}
\label{fig:race}
\end{figure}

%%%%%

We use the Scala Concurrency Library (SCL) in this book.  It provides a number
of convenient concurrency primitives.  Line~\ref{line:input} of
Figure~\ref{fig:race} imports this library, so the main primitives can be used
directly; for example, ``|thread|'' is shorthand for ``|ox.scl.thread|''.  (In
later programs, we will tend to omit this |import| statement, for brevity.)

A definition of the form \SCALA{thread\{<code>\}} defines a thread that when
executed will execute \SCALA{<code>}.  For example, line~\ref{line:p} defines
|t1| to be a thread that when executed will increment the shared variable~|x|
1000 times, and line~\ref{line:q} defines |t2| to be a thread that when
executed will decrement |x| 1000 times.  Each thread is of type
|ThreadGroup|\footnote{This type is not the same as the type {\scalashape
    java.lang.ThreadGroup}.}.

If |p| and |q| are |ThreadGroup|s, then \SCALA{p || q} represents their
parallel composition, also of type |ThreadGroup|.  For example,
line~\ref{line:system} defines |system| to be the parallel composition of~|t1|
and~|t2|. 

If |p| is a |ThreadGroup|, then |run(p)| or |p.run| runs~|p|.  A parallel
composition terminates when all components have terminated.  Thus the |main|
function at line~\ref{line:main} runs |system|, i.e.,~it runs |t1| and~|t2| in
parallel, and when both have terminated, prints the final value of~|x|.

You might expect that the increments and decrements cancel out, so that the
program always prints~|0|.  In fact, the program can give any result between
$-1000$ and $+1000$.
%
To understand why, we need to understand how a concurrent program is
executed. 
 
Certain actions can be considered \emph{atomic}; for example a single machine
instruction, or, in the case of a Scala program, a single instruction of the
Java Virtual Machine.  Typical actions include loading the value of a variable
into a register, incrementing or decrementing it, or storing a register into a
variable.

A sequential program, or a single thread, executes a sequence of atomic
actions.  
%
A concurrent program runs two or more threads: the atomic actions of the
threads are interleaved; but there might be multiple ways in which those
actions can be interleaved.

Let's consider a slightly simpler program, where |t1| increments~|x| just
once, and |t2| decrements~|x| just once.  Let's suppose each does this via three
instructions, loading~|x| into a register, incrementing or decrementing it,
and storing the result back in~|x|.  We could write this as
\begin{quote}
\SCALA{LD x; INC; ST x} \qquad and \qquad \SCALA{LD x; DEC; ST x}
\end{quote}
respectively (in fact, the JVM uses slightly longer sequences).

These six instructions can be interleaved in multiple ways.  The table below
considers three interleavings, starting from $\sm x = 0$.
%
\begin{center}
\begin{tabular}{lccccccl}
|x = x+1| & \SCALA{LD x} & \SCALA{INC} & \SCALA{ST x} & \\
|x = x-1| &      &      &     & \SCALA{LD x} & \SCALA{DEC} & \SCALA{ST x}
& ($\sm x = 0$) 
\\
\hline
\SCALA{x = x+1} & \SCALA{LD x} &      & \SCALA{INC} & \SCALA{ST x} & \\
\SCALA{x = x-1} &      & \SCALA{LD x} &     &      & \SCALA{DEC} & \SCALA{ST x}
& (\SCALA{x = -1}) 
\\
\hline
\SCALA{x = x+1} & \SCALA{LD x} &      &     &      & \SCALA{INC} & \SCALA{ST x} & \\
\SCALA{x = x-1} &      & \SCALA{LD x} & \SCALA{DEC} & \SCALA{ST x} &     &
& (\SCALA{x = 1})
\end{tabular}
\end{center}
%
In the first interleaving, all the atomic instructions of the increment are
executed before all the atomic instructions of the decrement, and so we end up
with $\sm x = 0$.  In the second interleaving, both threads read the initial
value of~|x|, then the first thread writes the result of the increment ($1$),
and then the second thread writes the result of the decrement ($-1$),
overwriting the earlier write; hence we end up with $\sm x = -1$.  In the
third interleaving, the writes happen in the opposite order, and so we end up
with $\sm x = 1$.  All other interleavings give one of these three values.  

Returning to the program in Figure~\ref{fig:race}, it should now be clear how
different interleavings can produce any result between $-1000$ and~$1000$.

More generally, two actions are \emph{independent} if their order may be
reversed without changing the overall effect.  For example, two actions
concerning distinct variables are independent, as are two reads of the same
variable.  Likewise, the |INC| and |DEC| actions by the above threads affect
only thread-local registers, so are independent of any other actions.
Interleaving of independent atomic actions is not problematic.

However, some actions are non-independent, for example two writes of the same
variable, or a read and a write of the same variable.  Interleaving of
non-independent atomic actions can be problematic, as it means that the
program can produce different results on different runs.

We say that a program contains a \emph{race} if two non-independent actions
can be interleaved, and one of the interleavings leads to an incorrect result.
In particular, where those actions are reads or writes of shared variables, we
say that it is a \emph{memory race}.  The program in Figure~\ref{fig:race}
contains memory races: the writes by the two threads can happen in different
orders, leading to different results, as in the second and third interleavings
above.

Different threads can run at different rates: threads can be descheduled and
only later rescheduled; threads may be delayed in memory actions because of
congestion on the memory bus.  It is simply not feasible to predict which
interleaving will occur, and so the behaviour of the program becomes
unpredictable. 

%%%%% \heading{Caching}

However, there are other reasons why programs with memory races can be
unpredictable.  
%
As described earlier, multiprocessor machines may cache variables.  Caching
can make a program much faster; but it can create problems with concurrent
programs.

When a thread~$t$ first needs the value of a variable~|x|, it will read it
from shared memory, and store a copy in its cache.  If it later needs the
value of~|x|, it may read it from its cache.  However, if some other thread
has updated~|x| in the mean time, thread~$t$ will not see the result of this
update!

Similarly, when a thread~$t$ updates a variable, that update is initially made
only in its cache.  If another thread reads the variable, it will not see the
result of the update.

In the program of Figure~\ref{fig:race}, each thread might read the initial
value of~|x|, perform updates only within its cache, and write the final
value, $1000$ or $-1000$, only at the end.  (However, they are guaranteed to
write their final value when they terminate.)  Thus the final value will be
$1000$ or $-1000$, depending upon which thread writes its value last---in
fact, on some architectures this seems to happen on most runs of the program.

%%%%% \heading{Compiler optimisations}

Another reason why programs with memory optimisations can be unpredictable
concerns memory optimisations.  The compiler may optimise code, according to
certain rules, to something that is equivalent when run sequentially.
Compiler optimisations are certainly useful, and often produce faster code.
However, the resulting code might not be equivalent when run as part of a
concurrent program.

Consider the code in Figure~\ref{fig:busyWait}.  This uses a variant of the
|thread| function: the construct |thread(name){<code>}| acts like
|thread{<code>}|  but gives the name |name| to the thread; we will see how
this is useful shortly.

In Figure~\ref{fig:busyWait}, it appears that |t1| sets |answer| to |42|, and
then sets |done| to |true|, while |t2| waits for |done| to become true before
reading |answer|: it appears that |t1| is using the variable |done| to signal
to |t2| after it has written to |answer|.

\begin{figure}
\begin{scala}
object BusyWait{
  var answer = 0

  var done = false

  /** Thread to set £answer£ to 42, and then signal to £t2£. */
  def t1 = thread("t1"){ answer = 42; done = true }

  /** Thread to wait for a signal, and read £answer£. */
  def t2 = thread("t2"){
    while(!done){ } // Busy wait.  Don't do this!!!
    val myAnswer = answer; assert(myAnswer == 42)
  }

  /** Parallel composition. */
  def system = t1 || t2

  def main(args: Array[String]) = { 
    for(i <- 0 until 10000){ answer = 0; done = false; run(system); print(".") }
  }
}
\end{scala}
\caption{A program that shows an incorrect signalling technique.}
\label{fig:busyWait}
\end{figure}

However, the code might not act as expected because of compiler
optimisations.  For example, it would be perfectly valid for the compiler to
rewrite the body of~|t1| to
\begin{scala}
  done = true; answer = 42
\end{scala}
%
This is equivalent to the original code as a sequential program, and is a
valid rewrite.  However, the logic around the signal no longer holds.

Likewise, it would be valid to rewrite the body of~|t2| to
\begin{scala}
  val myAnswer = answer; assert(myAnswer == 42)
  while(!done){ } 
\end{scala}
and again the logic around the signal doesn't hold.  

Alternatively, it would
be valid for the compiler to rewrite the body of~|t2| to
% 
\begin{scala}
  if(!done){ while(true){ } }
  val myAnswer = answer; assert(myAnswer == 42)
\end{scala}
This is a fairly standard form of rewrite: it avoids rereading the variable
|done|.  Now, this thread can read |done = false| and loop forever, so the
program gets stuck.  In fact, on my machine, the program runs for a few
hundred iterations and then gets stuck: I think the just-in-time compiler
(which optimises Java bytecode to native machine code, while the program is
running) performs an optimisation equivalent to the one above.

When a program does get stuck in this way, typing \texttt{Ctrl}+$\backslash$
(i.e.~holding down the \texttt{Ctrl} key, and pressing the backslash key) in
the terminal produces a thread dump.  This lists information about all the
running threads, such as giving their current line number in the code.  (This
normally includes several threads that are part of the runtime, rather than
the program in question; these can normally be ignored.)  In particular, this
thread dump uses names given to threads by the |thread(name){...}| construct.
In the example of the previous paragraph, |t1| is not listed, because it has
terminated; however, |t2| is still at the line corresponding to the |while|
loop.

The style of the loop in~|t2|, where the thread spins, waiting for a condition
to become true, is known as \emph{busy waiting}.  This is normally considered
bad style.  As the above example shows, it often doesn't work!  And even if it
did work, the spinning thread would be consuming computational resources that
might be better used elsewhere.  It would be better for the thread to suspend
until the condition becomes true: we will see in later chapters how to do
this.  

We have seen that programs that contain race conditions are likely to be very
hard to reason about, unpredictable, and simply wrong!  However, we want the
behaviour of our programs to be predictable, and correct.  We therefore need
to design programs to avoid race conditions.  We will see ways to do this
later in the book.  In particular, in order to avoid memory races, we will
design our programs so that two threads may not perform interleaved actions on
a shared variable, except if both threads are performing reads.  We will see
that this has the additional benefit of removing the types of errors
introduced by caching or compiler optimisations.

%% NSA have issued a report saying that race conditions are the eighth most
%% dangerous programming error.

%%%%%%%%%%%%%%%%%%%%%%%%%%%%%%%%%%%%%%%%%%%%%%%%%%%%%%%%%%%%

\heading{Correctness properties} 

\begin{description}
\item[Safety/correctness:] The results produced by the program are correct.
  Typically this requires that the system state will satisfy an (intended)
  invariant property after every ``action''.

\item[Liveness/progress:] The system as a whole does something useful.
\end{description}

\bigskip

\heading{Performance properties}

\begin{description}
\item[Latency:]
Requests get serviced reasonably quickly.

\item[Throughput:]
The system deals with a high number of requests.
\end{description}
