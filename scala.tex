
\renewcommand{\textfraction}{.1}
\renewcommand{\topfraction}{.9}
\renewcommand{\floatpagefraction}{.75} % min proportion for [p] floats
\renewcommand{\bottomfraction}{0.9}
\def\floatsep{\medskipamount} % between [tb] floats and space
\def\intextsep{\medskipamount} % around [h] floats
\def\textfloatsep{\medskipamount} % between [tb] floats
\setcounter{topnumber}{2}

\chapter{A Brief Introduction to Scala}
\label{app:scala}

In this appendix, we give a brief introduction to Scala.  We present different
aspects of the language in separate boxes, to aid referencing; see list on
page~\pageref{listofsib}.

For those wanting to learn more about Scala, we would recommend the excellent
\emph{Programming in Scala}, by Martin Odersky, Lex Spoon, Bill Venners and
Frank Sommers~\cite{odersky+}.  Alternatively, there are good online tutorials.

In Section~\ref{sec:scala-core}, we describe core language features: the body
of the book assumes that you are familiar with these.  For other language
features, when we first use the feature, we include a reference to the
relevant description in this appendix.  Section~\ref{sec:scala-encapsulation}
describes encapsulation.  Section~\ref{sec:scala-types} describes types.
Section~\ref{sec:scala-misc} describes further language features.



%% \framebox{TO BE WRITTEN}

%% Scala built on top of Java; can use Java libraries; compiled into bytecode for
%% execution on the Java Virtual Machine (JVM).  Influenced by Haskell.

%%%%%

\section{Core language features}
\label{sec:scala-core}

As a first example, below is a simple program for
calculating factorials. 
%
%\begin{figure}
\begin{scala}
/** A program for calculating factorials. */
object Factorial{
  /** The factorial of £n£.  Pre: £n $\ge 0$£. */
  def fact(n: Int): Int = {
    require(n >= 0, s"Illegal argument $n for fact")
    var x = 1 // Invariant: £$\sm x = \sm i !$£ at the end of each iteration.
    for(i <- 1 to n) x = x*i
    x
  }

  def main(args: Array[String]) = {
    val n = args(0).toInt; println(fact(n))
  } 
}
\end{scala}
%% \caption{A simple factorial program.}
%% \label{fig:factorial}
%% \end{figure}

%%%%%

Most of the work is done by the function |fact| (Scala
box~\ref{sb:functions}): this takes an argument~|n| which is an |Int| (Scala
box~\ref{sb:types}), and returns the factorial of~|n|, also an |Int|.

%\afterpage{\clearpage}

%%%%%%%%%%%%%%%%%%%%%%%%%%%%%%%%%%%%%%%%%%%%%%%%%%%%%%%%%%%%

\begin{scalaBox}{Functions}
\label{sb:functions}
In Scala, functions are introduced using the keyword \SCALA{def} (short for
\emph{definition}).  Each parameter (or argument) of the function is given a
type using a single colon, e.g.~\SCALA{a: A}.  Likewise the type of the result
of the function can be given using a colon.  For example
\begin{scala}
  def func(a: A, b: B, c: C): T = ...
\end{scala}
introduces a function called |func| that takes three parameters |a|, |b|,
and~|c|, of types~|A|, |B|, and |C|, respectively, and returns a result of
type~|T|.  The body of the function appears on the right-hand side of the
``|=|''.  If the body consists of more than a single command or expression,
then it should be surrounded by curly brackets.

The return type of a function is often optional.  However, it is necessary for
a recursive function.  Even when it is not necessary, it is often a good idea
to include it, to help document the function.

If the body of the function ends with an expression, the result of the
function is the value of that expression: no explicit |return| is necessary.
\end{scalaBox}

%%%%%

\begin{scalaBox}{Value types}
\label{sb:types}
Scala has several primitive value types.  In this book we use the following.
\begin{description}
\item[{\scalashape Int}:] 32-bit integers, holding values in the range
  $\interval{-2^{31}}{2^{31}}$; 

\item[{\scalashape Long}:] 64-bit integers, holding values in the range
  $\interval{-2^{63}}{2^{63}}$; 

\item[{\scalashape Double}:] 64-bit floating-point numbers;

\item[{\scalashape Boolean}:] booleans, with two values, |true| and |false|;

\item[{\scalashape Unit}:] the unit type, containing a single value~|()|
  called the unit value.
\end{description}
% I don't think we use Float or Char.
\end{scalaBox}

%%%%%


The function starts by checking the precondition~$\sm n \ge 0$ (Scala
box~\ref{sb:assertions}).  It then declares a variable~|x| and initialises it
to~1 (Scala box~\ref{sb:variables}).  It next uses a |for| loop (Scala
box~\ref{sb:for-loop}), where the variable~|i| takes each value in the range
from |1| to |n|, inclusive, in turn (Scala box~\ref{sb:ranges}).  For each
such~|i|, the code multiplies |x| by~|i| (Scala box~\ref{sb:assignment}).
Finally, the function returns the value of~|x| (Scala
box~\ref{sb:functions}). The body of the function consists of the sequential
composition of these four lines (Scala box~\ref{sb:seq-comp}).

%%%%%%%%%%%%%%%%%%%%%%%%%%%%%%%%%%%%%%%%%%%%%%%%%%%%%%%

\begin{scalaBox}{Assertions}
\label{sb:assertions}
The command |assert(|$test$|)| evaluates the boolean expression $test$, and
gives an error if it is false.  The command |assert(|$test$|, |$msg$|)| is
similar, except it also displays the string $msg$ if $test$ is false.
Assertions are useful to catch errors in your code.  They are also useful to
document code, illustrating a property that you believe to be true at a
particular point.

The commands |require(|$test$|)| and |require(|$test$|, |$msg$|)| are
similar, except conventionally they are used to check preconditions of
functions.  Such statements are useful to catch when a function has been
erroneously called with its precondition not satisfied.  They are also useful
to document the code, stating assumptions.
\end{scalaBox}


%%%%%

\begin{scalaBox}{Variables}
\label{sb:variables}
\emph{Variables} are declared in Scala using the keyword |var|.  For example
\begin{scala}
  var x = 1
\end{scala}
introduces a new variable |x| and initialises it to~|1|.

Every variable in Scala has a \emph{type}, defining the range of values that
it can take.  The Scala compiler normally infers the type of a variable from
the way that it is initialised.  For example, the variable~|x|, above, has
type |Int|, the normal type for integers (whole numbers).  The Scala compiler
infers this type because |x| is initialised to~|1|, which has this type.  %% In
%% Scala (and most other languages), these |Int|s can hold only numbers in a
%% limited range, from $-2^{31}$ (inclusive) to $+2^{31}$ (exclusive).
%
%Scala box~\ref{sb:assignment} describes how variables are updated. 
\end{scalaBox}

%%%%%

\begin{scalaBox}{\SCALAKW{for} loops}
\label{sb:for-loop}
The |for| loop \SCALA{for(x <- xs) body} executes \SCALA{body}, where \SCALA{x}
takes each value in the sequence~\SCALA{xs} in turn.
\end{scalaBox}

%%%%%

\begin{scalaBox}{Ranges}
\label{sb:ranges}
\SCALA{a until b} creates the sequence of numbers from~\SCALA{a}
to~\SCALA{b-1}, inclusive.
% , which we sometimes write as $\interval{\sm a}{\sm b}$.
Similarly, \SCALA{a to b} creates the sequence of numbers from~\SCALA{a}
to~\SCALA{b}, inclusive.
\end{scalaBox}

%%%%%

\begin{scalaBox}{Assignment}
\label{sb:assignment}
Variables are updated using |=|.  For example
\begin{scala}
  x = x*i
\end{scala}
sets |x| to its previous value times~|i|.  
%
The assignment could also have been written as |x *= i|.  Likewise, an
assignment |x = x+1| could be written as |x += 1|.  Similar assignment
operators can be used with other infix operators; however, we prefer not to
over-use them.
\end{scalaBox}

%%%%%

\begin{scalaBox}{Sequential composition}
\label{sb:seq-comp}
Two commands can be executed one after the other by putting a semicolon
(``|;|'') between them; for example
\begin{scala}
  i = i+1; f = f*i
\end{scala}
Alternatively, they can be written on separate lines: the compiler
\emph{infers} a semicolon at the end of the line:
\begin{scala}
  i = i+1
  f = f*i
\end{scala}
\end{scalaBox}

%%%%%%%%%%%%%%%%%%%%%%%%%%%%%%%%%%%%%%%%%%%%%%%%%%%%%%%%%%%%
 

%%%%%%%%%%%%%%%%%%%%%%%%%%%%%%%%%%%%%%%%%%%%%%%%%%%%%%%

\begin{scalaBox}{Comments}
\label{sb:comments}
A one-line comment may be written in code by preceding it with ``|//|'':
\begin{scala}
  // This is a comment.
\end{scala}

A longer comment may be written
by placing it between ``{\scalastyle\rm /*}'' and ``{\scalastyle\rm */}''; it is
conventional to start each subsequent line with another 
``{\scalastyle\rm *}'':
\begin{scala}
  /* This is a comment that is too
   * long to fit on one line. */
\end{scala}

Scaladoc is a tool for extracting comments from a Scala program, and producing
corresponding HTML documentation.  Scaladoc uses comments that are written
between ``{\scalastyle\rm /**}'' and ``{\scalastyle\rm */}''; it is
conventional to start each subsequent line with another ``{\scalastyle\rm
  *}''.
%
%% There are good online tutorials describing Scaladoc syntax.  However, we will
%% mostly just give straightforward comments. 
\end{scalaBox}

%%%%%%%%%%%%%%%%%%%%%%%%%%%%%%%%%%%%%%%%%%%%%%%%%%%%%%%

\bigskip

\pagebreak[3]

The function is preceded by a comment, using Scaladoc format (Scala
box~\ref{sb:comments}), describing what the function does.  Internally, it
uses a comment to document an invariant of the loop.

\pagebreak[3]

The program is encapsulated in an object called |Factorial| (Scala
box~\ref{sb:objects}).  This has a |main| function (Scala box~\ref{sb:main})
which expects to receive a single argument on the command line.  It sets the
name~|n| to refer to this argument, converted to an |Int| (Scala
boxes~\ref{sb:arrays}, \ref{sb:val} and~\ref{sb:strings}).  It then calls
|fact|, instantiating its parameter with~|n|, and prints the result (Scala
box~\ref{sb:print}).

%%%%%%%%%%%%%%%%%%%%%%%%%%%%%%%%%%%%%%%%%%%%%%%%%%%%%%%

\begin{scalaBox}{Objects}
\label{sb:objects}
An object is defined using syntax
\begin{scala}
  object MyObject{ ... }
\end{scala}
%
The body (denoted ``|...|'' above) contains definitions, for example of
values, functions, types, and nested objects.  Typically, these constitute
some data and some operations on that data.

The definition defines a single object |MyObject|: this is sometimes called
the singleton pattern.

A field of an object can be accessed using the ``|.|'' operator,
e.g.~|MyObject.x|. 
\end{scalaBox}

%%%%%

\begin{scalaBox}{{\scalashape main} functions}
\label{sb:main}
Each program should contain a function with declaration
\begin{scala}
  def main(args: Array[String]) = ...
\end{scala}
This is  executed when the program is run.  The parameter |args| is
instantiated with any additional arguments provided on the command line.
\end{scalaBox}

%%%%%

\begin{scalaBox}{Arrays}
\label{sb:arrays}
If \SCALA{T} is a type then \SCALA{Array[T]} represents the type of arrays
that hold data of type~\SCALA{T}.  (This is an instance of parametric
polymorphism: see Scala box~\ref{sb:polymorphic}.)

Each array has a fixed size. The size of |a| can be obtained using the
expression |a.length|.

A new array of size~|n| can be created with, e.g.,
\begin{scala}
  val a = new Array[T](n)
\end{scala}

The \SCALA{i}th entry of array \SCALA{a} can be obtained using \SCALA{a(i)}.
(Note the round parentheses!)  The indexing is zero-based, i.e., if \SCALA{a}
is of size \SCALA{n}, then the elements are \SCALA{a(0),...,a(n-1)}.  This
indexing operation takes constant time.  

Arrays are \emph{mutable}.  The individual entries can be updated, e.g.
\begin{scala}
  a(i) = a(i) + 1
\end{scala}
\end{scalaBox}

%%%%%

\begin{scalaBox}{Named values}
\label{sb:val}
The Scala keyword |val| introduces a named value.  For example
\begin{scala}
  val f = fact(n)
\end{scala}
evaluates |fact(n)| and sets |f| to store the result.

Unlike variables introduced with |var|, values introduced with |val| can not
subsequently be changed; we say that they are \emph{immutable}.  Using |val|
rather than |var|, when appropriate, can make your program clearer: the reader
will understand that the value will not change.
\end{scalaBox}

%%%%%

\begin{scalaBox}{Strings}
\label{sb:strings}
Strings in Scala have type |String|.  They can  be defined by typing them
between double inverted commas, for example
\begin{scala}
  val hello = "Hello!"
\end{scala}

Strings can be concatenated together using ``|+|''. 

The operation |v.toString| converts value~|v| to a |String|.  The Scala compiler
provides a default implementation of |toString| for all values.  The
|toString| function is applied automatically when a |String| representation of
a value is required, for example within |println|.

Conversely, if |String|~|s| represents an |Int|, then it can be converted to
that |Int| using |s.toInt|.
\end{scalaBox}

%%%%%

\begin{scalaBox}{Standard printing functions}
\label{sb:print}
The standard function |print| prints its argument on the screen.
The function |println| is similar, except it also ends the line.
\end{scalaBox}

%%%%%

\pagebreak[4]
\subsection*{Running and compiling programs}

This program can then be compiled (Scala box~\ref{sb:compile}) and run (Scala
box~\ref{sb:run}).

\begin{scalaBox}{Compiling programs}
\def\prog#1{{\codecolour\tt #1}}
\label{sb:compile}
Programs can be compiled using the Scala compiler {\codecolour\tt scalac},
e.g.  {\codecolour
\begin{verbatim}
> scalac Factorial.scala
\end{verbatim}
}% 
\noindent This will produce a corresponding \texttt{\codecolour .class} file,
e.g.~\texttt{\codecolour Factorial.class} (assuming the file contains an
object called |Factorial|), and sometimes some other \texttt{\codecolour
  .class} files.  These files contain Java bytecode.

{\codecolour\tt scalac} can be quite slow, because it has to load a lot of
files.  It can be faster to use the fast Scala compiler {\codecolour\tt fsc},
e.g.  {\codecolour
\begin{verbatim}
> fsc Factorial.scala
\end{verbatim}
}% 
This will need to load the files the first time it is run; but will then keep
a cache of those files so will be faster subsequently.  (The fast Scala
compiler can become confused, for example if you change directory; it can be
reset using \prog{fsc -shutdown}.)
\end{scalaBox}

%%%%%

\begin{scalaBox}{Running programs}
\label{sb:run}
Once an object has been compiled, if it has a suitable |main| function with a definition of the form%
% 
\begin{scala}
  def main(args: Array[String]) = ...
\end{scala}
%
then it can be executed using \texttt{\codecolour scala}, e.g.  {\codecolour
\begin{verbatim}
> scala Factorial 5
\end{verbatim}
}%
\noindent
This executes the bytecode in the \texttt{\codecolour .class} file on the
Java Virtual Machine.
The array~|args| is instantiated as an array holding any additional arguments
provided on the command line (as strings). 
\end{scalaBox}

%%%%%%%%%%%%%%%%%%%%%%%%%%%%%%%%%%%%%%%%%%%%%%%%%%%%%%%

\subsection*{Other core language features}

To present some other core language features, we write the factorial function
|fact| in two different ways.  Firstly, it could be defined recursively using
an |if| statement (Scala box~\ref{sb:if}) as follows.
%
\begin{scala}
  def fact(n: Int): Int = {
    require(n >= 0)
    if(n == 0) 1 else n*fact(n-1)
  }
\end{scala}

%%%%%

\begin{scalaBox}{\SCALAKW{if} expressions}  
\label{sb:if}
The command
\begin{scala}
  if (£$test$£) £$c_1$£ else £$c_2$£
\end{scala}
tests whether $test$ is true; if it is, it executes $c_1$; otherwise it
executes $c_2$.

Note that the test must be inside parentheses, and (unlike some languages)
there is no ``\SCALA{then}'' after the test.

The command
\begin{scala}
  if (£$test$£) £$c_1$£
\end{scala}
is similar, except it does nothing if $test$ is false.

If either $c_1$ or $c_2$ above comprises more than a single statement, then it
should be surrounded by curly brackets.
\end{scalaBox}

%%%%%

\begin{mysamepage}
Alternatively, the factorial function can be written using a |while| loop
(Scala box~\ref{sb:while}):
%
\begin{scala}
  def fact(n: Int): Int = {
    require(n >= 0)
    var f = 1; var i = 0 // Invariant: £$\sm f = \sm i! \land 0 \le \sm i \le n$£.
    while(i < n){ i += 1; f *= i }
    // £$\sm i = \sm n$£ so £$\sm f = \sm n!$£.
    f
  }
\end{scala}
\end{mysamepage}

%%%%%

\begin{scalaBox}{\SCALAKW{while} loops}
\label{sb:while}
The code 
\begin{scala}
  while(£$test$£) £$body$£
\end{scala} 
does the following
\begin{enumerate}
  \item evaluates $test$;
  \item if $test = true$, executes $body$, and returns to step~1;
  \item if $test = false$, finishes.
\end{enumerate}
Hence it repeatedly executes $body$ while $test$ is true. 

The body of the |while| loop is the \emph{single} statement following the
test.  If the intention is that the body comprises more than a single
statement, then those statements should be surrounded by curly brackets.
\end{scalaBox}

%%%%%%%%%%%%%%%%%%%%%%%%%%%%%%%%%%%%%%%%%%%%%%%%%%%%%%%%%%%% Encapsulation

\section{Encapsulation}
\label{sec:scala-encapsulation}

Code may be encapsulated in objects (Scala box~\ref{sb:objects}), classes
(Scala box~\ref{sb:classes}), traits (Scala box~\ref{sb:traits}), and abstract
classes~\ref{sb:abstract-class}.  Scala differentiates between classes (of
which there may be many instances) and objects (of which there is a single
instance).  A class and object may be companions of one another (Scala
box~\ref{sb:companion}).  A class or object may extend a trait or abstract
class (Scala box~\ref{sb:extends}).  

%%%%%

\begin{scalaBox}{Classes}
\label{sb:classes}
A class is a template for objects.  A class can be defined using syntax
\begin{scala}
  class MyClass[£$\sm A_1, \ldots, \sm A_k$£](£$\sm x_1: \sm T_1, \ldots, \sm x_n: \sm T_n$£){ ... } 
\end{scala}
Here $\sm A_1, \ldots, \sm A_k$ are polymorphic type parameters (Scala
box~\ref{sb:polymorphic}), $\sm x_1,\linebreak[1] \ldots,\linebreak[1] \sm
x_n$ are parameter names, and $\sm T_1, \ldots, \sm T_n$ are corresponding
types (possibly using the type parameters).  

An instance of~|MyClass| can be created with
\begin{scala}
  val myObj = new MyClass[£$\sm B_1, \ldots, \sm B_k$£](£$\sm e_1, \ldots, \sm e_n$£)
\end{scala}
where $\sm B_1, \ldots, \sm B_k$ are types with which the type parameters are
instantiated, and $\sm e_1, \ldots, \sm e_n$ are expressions with types $\sm
T_1, \ldots, \sm T_n$ when the type parameters are instantiated with $\sm B_1,
\ldots, \sm B_k$.  The object |myObj| then has type |MyClass|.
\end{scalaBox}

%%%%%

\begin{scalaBox}{Companion objects}
\label{sb:companion}
If a class and an object have the same name, then they are known as
\emph{companions}.  Typically, the companion object will include definitions
that relate generally to instantiations of the class, rather than being
operations on a particular instantiation.

Companions can access |private| and |protected| fields of one another. 
\end{scalaBox}

%%%%%

\begin{scalaBox}{Traits}
\label{sb:traits}
A trait can be defined using syntax 
\begin{scala}
  trait MyTrait[£$\sm A_1, \ldots, \sm A_k$£]{ ... }
\end{scala}
where $\sm A_1, \ldots, \sm A_k$ are polymorphic type parameters (Scala
box~\ref{sb:polymorphic}).  A trait typically defines the interface of one or
more subclasses, or defines some common code used in subclasses.  

A trait cannot be instantiated directly: instead subclasses can be defined and
instantiated.
\end{scalaBox}

%%%%%

\begin{scalaBox}{Abstract classes}
\label{sb:abstract-class}
An abstract class can be defined using syntax
\begin{scala}
  abstract class MyClass[£$\sm A_1, \ldots, \sm A_k$£](£$\sm x_1: \sm T_1, \ldots, \sm x_n: \sm T_n$£){ ... } 
\end{scala}
%
Some values or functions can be abstract: the abstract class contains their
types, but with no definition; instead, a subclass provides the definition
(Scala box~\ref{sb:extends}).

An abstract class cannot be instantiated directly: instead subclasses can be
defined and instantiated.

The main differences between traits and  abstract classes are:
\begin{itemize}
\item An abstract class can have construction parameters, whereas a trait
  cannot (both can have type parameters);

\item A class or object can inherit from \emph{multiple} traits, but from only
  a single abstract class.
\end{itemize}
\end{scalaBox}

%%%%%

\begin{scalaBox}{\protect\SCALAKW{extends}}
\label{sb:extends}
An object~|O| or class~|C| can be defined to extend a trait or abstract
class~|T| using syntax
\begin{scala}
  object O extends T{ ... }
  class C extends T{ ... }
\end{scala}
It must provide an implementation for each of the operations defined by that
trait or abstract class.
%% The object
%% of class can use any (non-private) value of function of~$t$.

The object~|O| can be used anywhere a value of type~|T| is expected.
Likewise, any object that is an instance of~|C| can be used anywhere a value
of type~|T| is expected: we say that |C| is a \emph{subclass} of~|T|.

The definition of~|O| or |C| can use any non-private value, type or function
defined in~|T|.
\end{scalaBox}

%%%%%

\pagebreak[3]

We use case classes (Scala box~\ref{sb:case-classes}) at various points in
this book.

\pagebreak[3]

Fields of an object, class or trait may be marked as either private (Scala
box~\ref{sb:private}) or protected (Scala box~\ref{sb:protected}).  Any field
not so marked is public.
%%%%%

\begin{scalaBox}{\protect\SCALAKW{case} classes}
\label{sb:case-classes}
A definition of the following form (or with any number of parameters)
\begin{scala}
  case class C(x: X, y: Y) { ... }
\end{scala}
%
defines |C| as a \emph{case class}.  This has a number of consequences.
\begin{itemize}
\item Objects of type |C| can be defined without using ``|new|'', for example
|val o = C(x1, y1)|

\item The parameters of |C| act like |val|s: if |o| is of type~|C|, then its
  parameters can be accessed as |o.x| and |o.y|.

\item The compiler provides a natural definition of the function |toString|
  that converts an object of type~|C| to a string.  For the class~|C|, the
  |toString| function will produce a string such as ``|C(6, 7)|'' (assuming
  the types |X| and~|Y| are both~|Int|).

\item The compiler provides a natural definition of equality: two objects of
  type~|C| will be equal if their |x| and |y| parameters are equal.  Normally,
  two objects are considered equal if they are precisely the same object,
  i.e.~stored at the same address in memory. 

%% \item The compiler provides a natural definition of the function |hashCode|;
%%   see Chapter~\ref{chap:hash-tables}.  This definition is compatible with the
%%   definition of equality: if two objects are equal then they have the came
%%   hash codes. 

\item Pattern matching (Scala box~\ref{sb:pattern-matching}) can be used.
  With the above definition, a pattern |C(pat|$_1$|, pat|$_2$|)| matches a
  value |C(x,y)| provided |x| and |y| match |pat|$_1$ and |pat|$_2$,
  respectively, and bind names according to those patterns.
\end{itemize}
\end{scalaBox}

%%%%%

\begin{scalaBox}{\protect\SCALAKW{private}}
\label{sb:private}
If a field of a class, object or trait is marked with the keyword |private|,
then it can be accessed only by code inside the same class, object or trait,
or within a companion object or class (Scala box~\ref{sb:companion}).
\end{scalaBox}

%%%%%

\begin{scalaBox}{\SCALAKW{protected}}
\label{sb:protected}
If a field of a class or trait is marked with the keyword |protected|, then it 
can be accessed only by code inside the same class or trait, or classes or
traits that extend it.
\end{scalaBox}

\section{Types}
\label{sec:scala-types}

All Scala types are either value types (Scala box~\ref{sb:types}) or reference
types (Scala box~\ref{sb:reference-types}).  Scala supports parametric
polymorphism (Scala box~\ref{sb:polymorphic}) and function types (Scala
box~\ref{sb:function-types}).

%%%%%

\begin{scalaBox}{Reference types}
\label{sb:reference-types}
Each class defines a type with the same name, which is a \emph{reference
  type}.  Each variable of that type actually holds a reference to the object,
i.e.~the address in memory where the object is stored.  
%% References are sometimes called \emph{pointers}.
Likewise, when a function takes a parameter
of a reference type, the value that is actually passed is the object's
reference.
\end{scalaBox}

%%%%%

\begin{scalaBox}{Parametric polymorphism}
\label{sb:polymorphic}
Scala classes can be defined to be polymorphic, i.e.~their definition takes a
type parameter~|A|, and the class is defined for all values of~|A|.  For
example, the type |Array[A]| is the type of arrays that hold data of type~|A|:
we can instantiate~|A| with an arbitrary type.

Functions can also be defined in a polymorphic way, for example:
\begin{scala}
  /** Does £a£ contain the value £x£? */
  def contains[A](a: Array[A], x: A): Boolean = ...
\end{scala}
%
Here, the initial ``|[A]|'' introduces~|A| as a polymorphic type parameter.
The function is defined for all types~|A|. 
\end{scalaBox}

%%%%%

\begin{scalaBox}{Function types}
\label{sb:function-types}
If |A| and |B| are types, then \SCALA{A => B} is the type of functions
from~|A| to~|B|, i.e.~that take an argument of type~|A| and return a result of
type~|B|.  For example, \SCALA{Int => Boolean} is the type of functions from
|Int| to |Boolean|.
\end{scalaBox}

%%%%%

\pagebreak[3]

Scala provides for the definition of type synonyms (Scala
box~\ref{sb:type-synonym}). 

\begin{scalaBox}{Type synonyms}
\label{sb:type-synonym}
Type synonyms can be defined using the keyword |type|, for example
\begin{scala}
  type Price = Int
  type Log1 = Log[LogEvent]
\end{scala}
Such type synonyms can help to distinguish between different uses of a common
type, or to provide brevity. 
\end{scalaBox}

\pagebreak[3]

The Application Programming Interface (API) contains a large number of useful
types, including arrays (Scala box~\ref{sb:arrays}), strings (Scala
box~\ref{sb:strings}), lists (Scala box~\ref{sb:lists}), and |Option| types
(Scala box~\ref{sb:option-type}).  For other API classes, you should refer to
the online documentation (e.g.~by searching for ``Scala
API''; recall that this book uses Scala~2.13).


\begin{scalaBox}{Lists}
\label{sb:lists}
The type |List[A]| represents the type of immutable lists containing data of
type~|A|.

A list containing values~|x|, |y| and~|z| can be defined as \SCALA{List[A](x,
  y, z)}; the type parameter~|A| can normally be omitted, as the compiler can
infer it.

If |xs| has type |List[A]|, and |x| has type~|A|, then |x :: xs| is a new
|List[A]| containing~|x| followed by the elements of~|xs|.  The expression
|xs.isEmpty| returns a |Boolean| indicating whether |xs| is empty, and
|xs.length| gives the length of~|xs|.  The element at index~|k| (counting
from~0) can be obtained as |xs(k)|; this traverses the list, so takes
time~$O(\sm k)$.  If |xs| is not empty, then |xs.head| gives its first
element, and |xs.tail| gives the list containing all of~|xs| except the first
element.
\end{scalaBox}

%%%%%

\begin{scalaBox}{The {\scalashape Option} type}
\label{sb:option-type}
The type |Option[A]| contains values of two subtypes:
%
\begin{itemize}
\item values of the form |Some(x)| where |x| has type~|A|;

\item the value~|None|.
\end{itemize}

The |Option| type is useful in cases where a function might or might not be
able to return a proper result: it can return a result |Some(x)| to indicate
that it succeeded, with result~|x|; or it can return |None| to indicate that
it was unsuccessful.

|Some(x)| and |None| can be used as patterns in pattern matching (Scala
box~\ref{sb:pattern-matching}).  The pattern ``|Some(x)|'' (where |x| is a
variable) matches any value of the form |Some(v)|, and binds the variable~|x|
to the value~|v|.  The pattern ``|None|'' matches just the value~|None|.
\end{scalaBox}

%%%%%%%%%%%%%%%%%%%%%%%%%%%%%%%%%%%%%%%%%%%%%%%%%%%%%%%



%% \section{{\scalashape for} expressions}
%% \label{app:for-expressions}

%% \section{Lists}
%% \label{sec:scala-lists}

%% The type |List[A]| represents the type of immutable lists containing data of
%% type~|A|.

%% A list containing values~|x|, |y| and~|z| can be defined as \SCALA{List[A](x,
%%   y, z)}; the type parameter~|A| can normally be omitted, as the compiler can
%% infer it.

%% If |xs| has type |List[A]|, and |x| has type~|A|, then |x :: xs| is a new
%% |List[A]| containing~|x| followed by the elements of~|xs|.  The expression
%% |xs.isEmpty| returns a |Boolean| indicating whether |xs| is empty.  If |xs| is
%% not empty, then |xs.head| gives its first element, and |xs.tail| gives the
%% list containing all of~|xs| except the first element.

%\section{Option types}
 % types


\pagebreak[3]

\section{Further Language Features}
\label{sec:scala-misc}

We describe pattern matching (Scala box~\ref{sb:pattern-matching}), anonymous
functions (Scala box~\ref{sb:anon-function}), |for| expressions (Scala
box~\ref{sb:for-expressions}), and how to catch
exceptions~\ref{sb:try-catch}). 

\begin{scalaBox}{Pattern matching}
\label{sb:pattern-matching}
The Scala operator |match| performs pattern matching.  The command
%
\begin{scala}
  £$v$£ match{
    case £$pat_1$£ => £$cmd_1$£
    ...
    case £$pat_n$£ => £$cmd_n$£
  }
\end{scala}
tries to match the value~$v$ against each of the patterns $pat_1, \ldots,
pat_n$.  When it finds a pattern that matches, it executes the corresponding
command~$cmd_i$; if several match, it executes the first such command.  If no
pattern matches, then it throws an exception.

Patterns include the following.
\begin{itemize}
\item The constant pattern, matches just that value; for example |"get"| or
  |36|.

\item A variable pattern such as ``|x|'' matches all values, and binds the
name~|x| to the value.  A typed pattern such as ``|x: A|'' matches all values
of type~|A|, and binds the name~|x| to the value.  The wildcard pattern
\SCALA{\_} matches all values.  

\item A pair pattern, e.g.~|(pat|$_1$|, pat|$_2$|)|, where |pat|$_1$ and
  |pat|$_2$ are patterns, matches a pair~|(v|$_1$\SCALA{, v}$_2$|)| provided
  $\sm v_1$ and $\sm v_2$ match |pat|$_1$ and |pat|$_2$, respectively.  Names
  are bound according to those patterns.  This extends to larger tuples in the
  obvious way.
\end{itemize}
%
Case classes (Scala box~\ref{sb:case-classes}), such as the |Option| type
(Scala box~\ref{sb:option-type}), also allow for pattern matching.
\end{scalaBox}

%%%%%

\begin{scalaBox}{Anonymous functions}
\label{sb:anon-function}
Anonymous functions can be defined in Scala in a couple of ways.  The notation
\SCALA{(x: A) => exp} represents a function that takes an argument~|x| of
type~|A|, and returns the value of~|exp|.  For example, \SCALA{(x: Int) =>
  x+1} is a function that increments an |Int|.  The typing ``\SCALA{: A}'' can
often be omitted, when the Scala compiler can deduce the type; for example, in
\SCALA{List(1,2,3).map(x => x+1)}, the compiler can deduce that |x| must be an
|Int|.

An anonymous functions can also be defined using an underscore, ``\SCALA{_}'' to
represent a ``hole'' into which the argument is put.  For example,
\SCALA{((_:Int)+1)} again increments an |Int|.  Again, the type of the
argument can often be omitted. 
\end{scalaBox}

%%%%%

\begin{scalaBox}{\protect\SCALAKW{for} expressions}
\label{sb:for-expressions}
|for| expressions provide a convenient way of building sequences.  The
simplest form of |for| expression is of the form
\begin{scala}
  for(x <- xs) yield f(x)
\end{scala}
where |xs| is a sequence, and |f| is a suitable function: this will generate a
sequence that contains the value |f(x)| for each |x| in~|xs|.  The ``\SCALA{x
  <- xs}'' is known as a \emph{generator}.  For example, 
\begin{scala}
  for(x <- List(1,2,3)) yield x*x
\end{scala}
will produce the list |List(1,4,9)|.

A |for| expression may contain more than one generator.  For example
\begin{scala}
  for(x <- List(1,2,3); y <- List(4,5)) yield x*y
\end{scala}
will produce |List(4,5,8,10,12,15)|.  Note that the |for| expression iterates
over the second generator for each value from the first generator, before
moving on to the next value from the first generator.

A |for| expression may also contain a \emph{filter}, restricting the range of
values from which results are produced.  For example
\begin{scala}
  for(x <- List(1,2,3,4); if x%2 == 0; y <- List(1,2,3)) yield x*y
\end{scala}
only calculates |x*y| for even values of~|x|, so produces
|List(2, 4, 6, 4, 8, 12)|.

%% Scala has several different types of sequence: we mostly use |List|s in this
%% book.  The type of the result produced by a |for| expression depends upon the
%% type of the sequences used in the generators.  An arbitrary sequence can be
%% converted into a |List| using the operation |toList|, for example
%% \begin{scala}
%%   (for(x <- 1 to 3) yield x*x).toList
%% \end{scala}
\end{scalaBox}


\begin{scalaBox}{Catching exceptions}
\label{sb:try-catch}
Exceptions can be caught using a |try ... catch| construct as follows. 
\begin{scala}
try{ prog }catch{ handler }
\end{scala}
%
Here |prog| is a program fragment, and |handler| is a partial function
describing how to handle any exception thrown by |prog|.  The exception
handler is normally written using pattern matching syntax (Scala
box~\ref{sb:pattern-matching}), for example
\begin{scala}
  try{ prog } catch{
    case st: Stopped => ...
    case ae: AssertionError => ...
  }
\end{scala}
\end{scalaBox}


\begin{scalaBox}{{\scalashape apply} functions}
\label{sb:apply}
If an object~|o| has an operation called |apply| as follows:
\begin{scala}
  def apply(x: A) = ...
\end{scala}
then it can be invoked as |o(e)| where  |e| is an expression of type~|A|.
This generalises to different numbers of parameters, including none.  
\end{scalaBox}

\framebox{TODO:}  equality and hash codes
(from datatypes3).
