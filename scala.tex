\chapter{A brief introduction to Scala}
\label{app:scala}


Scala built on top of Java; can use Java libraries; compiled into bytecode for
execution on the Java Virtual Machine (JVM).  Influenced by Haskell.

An object is basically a class with a single instance, sometimes called a
\emph{singleton object}.  Most main modules in Scala are objects.

Most typing is optional.  Needed with arguments of procedures, and recursion.
Recommended when it improves clarity.

\SCALA{var} for variables.  \SCALA{val} for constant values.  \SCALA{def} for
definitions, e.g.~of procedures.  

\SCALA{0 until 1000} means $[0 .. 1000)$. 

|while|, |for|, |if|.  

def var val

\begin{scala}
object Factorial{

  def fact(n: Int): Int = {
    var x = 1
    for(i <- 1 to n) x = x*i
    i
  }

  def main(args: Array[String]) = {
    val n = args(0).toInt; println(fact(n))
  } 
}
\end{scala}


|assert|, |require| 
