\begin{slide}
\heading{Example: graph search}

Many Computer Science applications involve searching in a graph (whether
explicitly or implicitly).  Examples:
%
\begin{itemize}
\item Route planning;

\item Planning;

\item Puzzle solving;

\item Verification.
\end{itemize}
%
Searches might aim to minimise some cost, or might aim to find a node with a
particular property, using breadth-first search or depth-first search.

We will examine an algorithm for (approximate) concurrent breadth-first search.

The third practical asks you to implement (approximate) concurrent depth-first search.
\end{slide}

%%%%%

\begin{slide}
\heading{Graphs and graph search}

For our purposes, a graph can be defined by a function that gives all the
successors (or neighbours) of a given node.
%
\begin{scala}
/** A trait representing an unlabelled graph with nodes of type N. */
trait Graph[N]{
  /** The successors of node n.  All nodes n' such that there is an edge
    * from n to n' */
  def succs(n: N): List[N]
}
\end{scala}

We will search from some start node to try to find a path to a node with a
particular property |isTarget|.
\begin{scala}
/** Trait representing a graph search problem in graph g. */
abstract class GraphSearch[N](g: Graph[N]){
  /** Try to find a path in g from start to a node that satisfies isTarget. */
  def apply(start: N, isTarget: N => Boolean): Option[List[N]]
}
\end{scala}
\end{slide}

%%%%%

\begin{slide}
\heading{Example}

\def\word#1{\emph{#1}}
A popular word game, invented by Lewis Carroll, is to find a path of words,
each differing from the previous in a single letter, linking two given words.
Carroll used the example of finding such a path linking \word{grass} to
\word{green}, and gave the solution \word{grass}, \word{crass}, \word{cress},
\word{tress}, \word{trees}, \word{frees}, \word{freed}, \word{greed},
\word{green}, although in fact there are shorter solutions.

The course website contains a solution to this problem.  A |Graph| is built,
with words (from some dictionary file) as nodes, and with the successors of a
node being those that differ by one letter.  The |GraphSearch| then solves the
problem.
\end{slide}

%%%%%

\begin{slide}
\heading{Sequential breadth-first graph search}

We will search the graph in breadth-first order.  We will use a queue to store
nodes that still need to be expanded: this provides the breadth-first
behaviour.  Each node will be paired with the path that reached it, to allow a
successful path to be reproduced.

In addition, we will keep track of the set of nodes seen previously, to avoid
repeating work.
\begin{scala}
/** Sequential graph search implementation. */
class SeqGraphSearch[N](g: Graph[N]) extends GraphSearch[N](g){
  def apply(start: N, isTarget: N => Boolean): Option[List[N]] = {
    // Queue storing nodes and the path leading to that node
    val queue = new scala.collection.mutable.Queue[(N, List[N])]()
    queue += ((start, List(start)))
    // All nodes seen so far.
    val seen = scala.collection.mutable.Set[N](start)
    ...
} }
\end{scala}
\end{slide}

%%%%%

\begin{slide}
\heading{Sequential breadth-first graph search}

We repeatedly remove a node from the queue, and consider each of its
successors.  If a successor node is a target, we are done.  Otherwise, if the
successor node hasn't been seen previously, we add it to the queue.
%
\begin{scala}
  def apply(start: N, isTarget: N => Boolean): Option[List[N]] = {
    ...
    while(queue.nonEmpty){
      val (n, path) = queue.dequeue()
      for(n1 <- g.succs(n)){
        if(isTarget(n1)) return Some(path :+ n1)
        else if(seen.add(n1)) queue.enqueue((n1, path :+ n1))
      }
    }
    None
  }
\end{scala}
\end{slide}

%%%%%

\begin{slide}
\heading{Towards a concurrent algorithm}

We will implement a concurrent algorithm where each worker thread will execute
very much as in the main loop of the sequential algorithm.

The workers will share a concurrent queue.  We will use a
|TerminatingPartialQueue|. 

The workers will also share a concurrent set, with the following interface.
Exercise: implement such a set.
%
\begin{scala}
class ConcSet[A]{
  /** Add x to this set.  Return true if x was not previously in the set. */
  def add(x: A): Boolean
}
\end{scala}

We also assume that the |succs| method on the |Graph| object is thread-safe.
In most cases, this method will just perform reads, so there will be no race
conditions. 
\end{slide}

%%%%%

\begin{slide}
\heading{Towards a concurrent algorithm}

\begin{scala}
/** A class to search Graph g concurrently. */
class ConcGraphSearch[N](g: Graph[N]) extends GraphSearch[N](g){
  /**The number of workers. */
  val numWorkers = 8

  /** Try to find a path in g from start to a node that satisfies isTarget. */
  def apply(start: N, isTarget: N => Boolean): Option[List[N]] = {
    // Queue storing nodes and the path leading to that node
    val queue = new TerminatingPartialQueue[(N, List[N])](numWorkers)
    queue.enqueue((start, List(start)))
    // All nodes seen so far.
    val seen = new ConcSet[N]; seen.add(start)
    ...
  }
}
\end{scala}
\end{slide}

%%%%%

\begin{slide}
\heading{Termination}

There are a few complications concerning termination.
%
\begin{itemize}
\item If a path is found, we need to arrange for that path to be returned.  If
two threads each find a path, we need to return just one of them.  We need to
ensure the other threads terminate, and the queue is shut down.

We will use a |coordinator| thread to help with this.  If a thread finds a
path, it will send it to the |coordinator|.

\item If no path exists, we will get to a state where the queue is empty and
all threads are trying to dequeue a node; the |TerminatingPartialQueue| will
help us to terminate.
\end{itemize}
\end{slide}

%%%%%

\begin{slide}
\heading{A worker}

\begin{scala}
    // Channel on which a worker tells the coordinator that it has found a
    // solution.
    val pathFound = new SyncChan[List[N]]

    // A single worker
    def worker = thread("worker"){
      repeat{
        val (n, path) = queue.dequeue()
        for(n1 <- g.succs(n)){
          if(isTarget(n1)) pathFound!(path :+ n1) // done!
          else if(seen.add(n1)) queue.enqueue((n1, path :+ n1))
        }
      }
      pathFound.close // causes coordinator to close down
    }
\end{scala}
\end{slide}

%%%%%


\begin{slide}
\heading{The coordinator}

\begin{scala}
    // Variable that ends up holding the result; written by coordinator. 
    var result: Option[List[N]] = None

    def coordinator = thread("coordinator"){
      attempt{ result = Some(pathFound?()) }{ }
      queue.shutdown // close queue; this will cause most workers to terminate
      pathFound.close // in case another worker has found solution
    }
\end{scala}

If a thread finds a path, it sends it to the coordinator.  The coordinator
shuts down the queue and closes |pathFound|.  Any operation on the queue will
throw a |Stopped| exception.  Hence all the workers will terminate.  (It might
also be necessary to shut down the |seen| set.)

Alternatively, if there is no solution, the |queue| will eventually close
itself, throwing |Stopped| exceptions to the workers.  A worker will close
|pathFound|, so the coordinator will also terminate. 
\end{slide}

%%%%%

\begin{slide}
\heading{Putting it together}

\begin{scala}
  def apply(start: N, isTarget: N => Boolean): Option[List[N]] = {
    ... // initialise queue and seen set.
    val workers = || (for(_ <- 0 until numWorkers) yield worker)
    run(workers || coordinator)
    result
  }
\end{scala}
\end{slide}

%%%%%

\begin{slide}
\heading{Breadth-first search}

This doesn't quite achieve breadth-first search.  If a thread is delayed,
deeper nodes might be explored while a node from an earlier ply is being
expanded.  This might mean that the path that is found isn't the shortest.  In
many cases, this won't matter much.  If true breadth-first behaviour is
needed, the threads can performs a global synchronisation at the end of each
ply.
\end{slide}

%%%%%

\begin{slide}
\heading{Bag-of-tasks with replacement}

The implementation can be seen as a variant of the bag-of-tasks pattern, but
where additional tasks may be added to the bag.

Each node in the bag (the queue) represents the task of expanding that node.
This may lead to additional tasks being added to the bag.
\end{slide}
