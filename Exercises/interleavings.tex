\begin{questionS}
\begin{enumerate}
\item
Suppose two threads respectively perform $m$ and $n$ atomic actions
sequentially. How many interleavings of these atomic actions are there?

\item Use your answer to the previous part to give an estimate for the number of
  interleavings of the memory actions from Figure~\ref{fig:race}, where each
  thread performs 2000 memory actions.  \textbf{Hint:} use Stirling's
  approximation, $n! \approx \sqrt{2\pi n} \, (n/e)^n$. 
\end{enumerate}
\end{questionS}

%%%%%%%%%%%%%%%%%%%%%%%%%%%%%%%%%%%%%%%%%%%%%%%%%%%%%%%


\begin{answerS}
\begin{enumerate}
\item There are a total of $m+n$ actions, and the number of interleavings is
  the number of ways of choosing $m$ of those as the actions of the first
  thread.  There is a standard formula for this, namely
  \[\mstyle
  \frac{(m+n)!}{m! \, n!}
  \]
One way to see this is that if we ignore the orderings for each thread, there
are $(m+n)!$ ways of ordering these $m+n$ actions: we can choose the first
action in $m+n$ ways, the second in $m+n-1$ ways, and so on.  However, only a
proportion $1/m!$ of these will respect the order of the first thread, and
only a proportion $1/n!$ will respect the order of the second thread. 


\item
Using the above formula with $m = n = 2000$, and Stirling's approximation, we
get 
\[\mstyle
\frac{4000!}{(2000!)^2}
\approx \frac{\sqrt{2 \pi 4000} \, (4000/e)^{4000}}{
  2 \pi 2000 \, (2000/e)^{4000}}
= \frac{2^{4000}}{\sqrt{2000 \pi}}
\approx 1.7 \times 10^{1202}.
\]
That's a big number!
%% scala> Math.pow(2, 1000)
%% val res1: Double = 1.0715086071862673E301
%% scala> Math.pow(1.0715086071862673,4) /  Math.sqrt(Math.PI * 2000)
%% val res7: Double = 0.016630018093978575
%% So ~ 0.017E1204.
\end{enumerate}
\end{answerS}
