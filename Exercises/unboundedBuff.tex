\begin{question}
% \Programming\
Implement an unbounded buffer:
%
\begin{scala}
def buff[T](in: ??[T], out: !![T]) = thread{ ... }
\end{scala}
%
The buffer should always be willing to input on \SCALA{in}; it should output
data on \SCALA{out} in the order in which they were received, and should be
willing to output whenever it has input data that have not yet been output. 
%
You might want to use an instance of a \SCALA{scala.collection.mutable.Queue}
to store the current values held in the buffer.
\end{question}

%%%%%

\begin{answerI}
My code is below. 
\begin{scala}
  def buff[T](in: ??[T], out: !![T]) = thread{
    val queue = new scala.collection.mutable.Queue[T]
    serve(
      in =?=> { x => queue.enqueue(x) }
      | queue.nonEmpty && out =!=> { queue.dequeue() }
    )
    in.close(); out.endOfStream()
  }
\end{scala}
%
Note that in the second branch, it is important that the alternation evaluates
the expression |q.dequeue()| only \emph{after} another thread has committed to
communicating on~|out|: otherwise the dequeued value would be lost if no
communication takes place. 

A good way to test this is to arrange for one thread to pass in values, and
another thread to receive output values, and to check that the two agree.
\end{answerI}
