\begin{questionS}
Consider a web browser that supports multiple tabs (i.e.~different
tabs for different web pages).  Why might it be beneficial to use
concurrency in the implementation of such a web browser?  Would this
still make sense on a uni-processor computer?  What problems might
arise from such a design?
\end{questionS}

%%%%%

\begin{answerS}
There are two main reasons: ease of implementation, and efficiency.

Building a web browser this way is easier!  The different tabs are largely
independent.  It's far easier to write a thread that deals with a single tab,
and run $n$ such threads, than to write a single thread that deals with all
$n$ tabs.  Each tab thread will probably need to interact with some main
browser thread, but that's relatively straightforward.

In terms of efficiency, concurrency might be used to give better response to
user actions (i.e.\ lower latency), and better overall performance (i.e.\
higher throughput); of these, the former seems more important.

As an example of why a sequential implementation might be unsatisfactory,
consider a user who has two tabs open, one viewing a page that automatically
re-loads every few minutes, and another that the user is actively reading.
Suppose the user tries to scroll down in the second page just after the first
page starts a re-load; then the user has to wait until the re-load completes
before the second page scrolls; this could take several seconds, and so annoy
the user.  (Some browsers used to act in this way in the past.)  If the different
tabs used different threads, then they could run concurrently so the user
wouldn't see this delay.  

This would still be the case on a uni-processor machine.  However, the fact
that multiple threads are competing for the processor would slow things down a
bit.  Web browsers and web pages seem to have become more bloated over the
years, because it's assumed that they will run on a multi-processor machine,
so I suspect a uni-processor machine would struggle in some cases.

The problem is, of course, that two threads might try to access some shared
data (e.g.~the history list) simultaneously, leading to race conditions.
These race conditions can be avoided by careful programming: see the rest of
the book!
\end{answerS}
