\begin{question}
% \Programming\ 
Extend the ``slot'' example from the lectures on monitors so as
to hold a queue of data.  The length of the queue should be bounded by some
value $N$.
\end{question}

%%%%%

\begin{answer}
The adaptation is fairly straightforward.  The producer has to wait when the
queue has length $N$; the consumer has to wait when the queue is empty.
%
\begin{scala}
class Buffer[T](N:Int){
  private val queue = new scala.collection.mutable.Queue[T];

  def put(v:T) = synchronized{
    while(queue.length==N) wait();
    queue.enqueue(v); 
    notify();
  }

  def get : T = synchronized{
    while(queue.isEmpty) wait();
    val result = queue.dequeue;
    notify();
    return result;
  }
}
\end{scala}
%
%% [Students should include some testing code that causes the producer to be
%%   blocked sometimes; for example, the producer could produce data at a faster
%%   rate than the consumer consumes it.]
\end{answer}

