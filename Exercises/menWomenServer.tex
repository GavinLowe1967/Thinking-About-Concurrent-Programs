\begin{question}
Consider the following synchronisation problem.  There are two
types of client threads, which we shall call \emph{men} and \emph{women}.
These threads need to pair off for some purpose, with each pair containing
one thread of each type.
%
Design a server  to support this.  Each client should send its name to
the server, and receive back the name of its partner.  Encapsulate the server
within a class, with operations
\begin{scala}
  def manSync(me: String): String = ...
  def womanSync(me: String): String = ...
\end{scala}

Implement a test rig for your implementation: think carefully about the
correctness condition. 
\end{question}

%%%%%

\begin{answerI}
My solution is below.  The server has two request channels, one for each type
of client.  Each client sends its name and a reply channel.  The server waits
for a request from a man, and a request from a woman, and pairs them off.
%
\begin{scala}
class MenWomenServer{
  private type ReplyChan = Chan[String]

  /** Channels sending proposals from a man, resp., a woman. */
  private val manProp, womanProp = new SyncChan[(String, ReplyChan)]

  /** A man tries to find a partner. */
  def manSync(me: String): String = {
    val c = new OnePlaceBuffChan[String]; manProp!(me,c); c?()
  }

  /** A woman tries to find a partner. */
  def womanSync(me: String): String = {
    val c = new OnePlaceBuffChan[String]; womanProp!(me,c); c?()
  }

  /** The server. */
  private def server = thread{
    repeat{
      // Wait for a man and woman, and pair them off. 
      val (him,hisC) = manProp?(); val (her,herC) = womanProp?()
      hisC!her; herC!him
    }
    manProp.close(); womanProp.close()
  }

  fork(server)

  /** Shut down this object (so the server thread terminates). */
  def shutdown() = { manProp.close(); womanProp.close() }
}
\end{scala}


There are alternative solutions that involve storing requests that cannot be
paired immediately.  However, such approaches are more complex.  It is often
best to avoid communicating on a channel if you know the corresponding request
can't be served.


Most of my testing code is below.  We need to check that men and women receive
compatible results: if man~$m$ thinks he is paired with woman~$w$, then $w$
thinks she is paired with~$m$.  This implies the vice-versa direction
(assuming equal numbers of men and women).  It also implies that the two
operations synchronise, i.e.~overlap in time, as for the exchanger example.

Each man stores his result in |partnerOfMan|, and each woman stores her result
in |partnerOfWoman|; we then check the results agree.  (Each man and woman has
an integer identity; we take their name to be the corresponding string; we
cast back to an |Int| to store in the arrays.)
%
\begin{scala}
  /** Arrays that hold the identity of each man/woman's partner. */
  var partnerOfMan, partnerOfWoman: Array[Int] = null

  /** Thread for a man. */
  def man(me: Int, mw: MenWomen) = thread{
    partnerOfMan(me) = mw.manSync(me.toString).toInt
  }

  /** Thread for a woman. */
  def woman(me: Int, mw: MenWomen) = thread{
    partnerOfWoman(me) = mw.womanSync(me.toString).toInt
  }

  /** Do a single test. */
  def doTest = {
    val n = scala.util.Random.nextInt(10) // Number of men, women.
    partnerOfMan = new Array[Int](n); partnerOfWoman = new Array[Int](n)
    val mw = new MenWomenServer 
    val men = || (for(i <- 0 until n) yield man(i, mw))
    val women = || (for(i <- 0 until n) yield woman(i, mw))
    run(men || women)
    mw.shutdown
    for(m <- 0 until n)
      assert(partnerOfWoman(partnerOfMan(m)) == m,
             partnerOfMan.mkString(", ")+"\n"+
               partnerOfWoman.mkString(", ")+"\n"+m)
  }
\end{scala}

Alternatively, the implementation can be tested by adapting the
graph-theoretic approach described at the end of
Section~\ref{sec:filter-chan}. 
\end{answerI}
% 

%\begin{scala}
% // The men and women problem, using a server process

% import ox.CSO._

% class MenWomen{
%   type Name = String; // Names of men and women
%   private type ReplyChan = OneOne[Name]; // channels for replies

%   // Each request contains the name of the requester and a reply channel
%   private type Req = Tuple2[Name, ReplyChan];
%   private val manReq, womanReq = ManyOne[Req];

%   // Men and women send a request to the server on a fresh channel, and
%   // receive back a reply
%   def ManSync(me: Name) : Name = {
%     val c = OneOne[Name]; manReq ! (me, c);
%     val her = c?; return her;
%   }

%   def WomanSync(me: Name) : Name = {
%     val c = OneOne[Name]; womanReq ! (me, c);
%     val him = c?; return him;
%   }

%   private def Server = proc{
%     // Queues storing the men and women who have not yet been paired; at most
%     // one should be non-empty
%     val manQueue = new scala.collection.mutable.Queue[Req];
%     val womanQueue = new scala.collection.mutable.Queue[Req];
%     // The server stores unmatched requests in queues.  Each new request is
%     // matched immediately, if possible; otherwise it's places in the relevant
%     // queue.

%     serve(
%       manReq --> {
% 	val (man, manc) = manReq?; 
% 	if(womanQueue.isEmpty) manQueue += (man, manc);
% 	else{
% 	  val (woman, womanc) = womanQueue.dequeue;
% 	  manc ! woman; womanc ! man;
% 	} 
%       }
%       |
%       womanReq --> {
% 	val (woman, womanc) = womanReq?; 
% 	if(manQueue.isEmpty) womanQueue += (woman, womanc);
% 	else{
% 	  val (man, manc) = manQueue.dequeue;
% 	  manc ! woman; womanc ! man;
% 	} 
%       }
%     )
%   }

%   // Public method to close down the server
%   def Close = { manReq.close; womanReq.close; }

%   // Fork off server process
%   Server.fork;
% }

% // -------------------------------------------
% // Class to test the module

% object MenWomenTest{
%   val N = 10;
%   val random = new scala.util.Random;

%   // Create the server
%   val menWomenServer = new MenWomen;

%   type Name = menWomenServer.Name;

%   // Men and women call the appropriate methods, after a short delay
%   def Man(me: Name) = proc{
%     sleep(random.nextInt(2)); val her = menWomenServer.ManSync(me);
%     println("Man "+me+" pairs with woman "+her);
%   }

%   def Woman(me: Name) = proc{
%     sleep(random.nextInt(2)); val him = menWomenServer.WomanSync(me);
%     println("Woman "+me+" pairs with man "+him);
%   }

%   // Put the system together
%   def Men = || ( for (i <- 0 until N) yield Man("man"+i) );
%   def Women = || ( for (i <- 0 until N) yield Woman("woman"+i) );
  
%   def System = Men || Women;

%   def main(args : Array[String]) = { System() ; menWomenServer.Close; }
% }
% \end{scala}
% \end{answer}


%% \begin{scala}
%% // The men and women problem, using a server process

%% import ox.CSO._

%% object MenWomen{

%%   val N = 10;
%%   val random = new scala.util.Random;

%%   type Name = String; // Names of men and women
%%   type ReplyChan = OneOne[Name]; // channels for replies

%%   // Each request contains the name of the requester and a reply channel
%%   type Req = Tuple2[Name, ReplyChan];
%%   val manReq, womanReq = ManyOne[Req];

%%   // Men and women send a request and receive back a reply
%%   def Man(me: Name) = proc{
%%     val c = OneOne[Name];
%%     sleep(random.nextInt(2));
%%     manReq ! (me, c);
%%     val her = c?;
%%     println("Man "+me+" pairs with woman "+her);
%%   }

%%   def Woman(me: Name) = proc{
%%     val c = OneOne[Name];
%%     sleep(random.nextInt(2));
%%     womanReq ! (me, c);
%%     val him = c?;
%%     println("Woman "+me+" pairs with man "+him);
%%   }

%%   def Server = proc{
%%     // Queues storing the men and women who have not yet been paired; at most
%%     // one should be non-empty
%%     val manQueue = new scala.collection.mutable.Queue[Req];
%%     val womanQueue = new scala.collection.mutable.Queue[Req];
%%     // The server stores unmatched requests in queues.  Each new request is
%%     // matched immediately, if possible; otherwise it's placed in the relevant
%%     // queue.

%%     serve(
%%       manReq --> {
%% 	val (man, manc) = manReq?; 
%% 	if(womanQueue.isEmpty) manQueue += (man, manc);
%% 	else{
%% 	  val (woman, womanc) = womanQueue.dequeue;
%% 	  manc ! woman; womanc ! man;
%% 	} 
%%       }
%%       |
%%       womanReq --> {
%% 	val (woman, womanc) = womanReq?; 
%% 	if(manQueue.isEmpty) womanQueue += (woman, womanc);
%% 	else{
%% 	  val (man, manc) = manQueue.dequeue;
%% 	  manc ! woman; womanc ! man;
%% 	} 
%%       }
%%     )
%%   }

%%   // Put the system together
%%   def Men = || ( for (i <- 0 until N) yield Man("man"+i) );
%%   def Women = || ( for (i <- 0 until N) yield Woman("woman"+i) );
  
%%   def System = Men || Women || Server;

%%   def main(args : Array[String]) = System() 
%% }
%% \end{scala}
%\end{answer}

      
