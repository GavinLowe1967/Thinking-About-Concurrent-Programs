\begin{question}
\label{ex:diningPhils}
The aim of this question is to investigate some variants of the Dining
Philosophers example that aim to avoid deadlocks, as discussed in
Section~\ref{sec:avoid-deadlocks}.  You should implement and test each of the
three variants below.
%
\begin{enumerate}
\item \textbf{A right-handed philosopher.}
In the standard version of the Dining Philosophers, all the philosophers are
left-handed: they pick up their left fork first.  Implement a variant where
one of the philosophers is right-handed, i.e.\ they pick up their right fork
first.

\item \textbf{Using a butler.} Now consider a variant using an extra
  thread, which represents a butler.  The butler makes sure that no more than
  four philosophers are ever simultaneously seated.

\item \textbf{Using timeouts.} Now consider a variant where, if a philosopher
  is unable to obtain their second fork within a reasonable time, they put
  down their first fork and try again later.  In the version in the body of this
  chapter, all delays are a few hundred milliseconds, so the period spent
  attempting to pick up the second fork should be of a similar order.  You
  will need to use the |sendWithin| operation, described earlier.

If the system reaches a state where all philosophers are blocked, it's
possible that they all put down their forks at the same time, and then all
retry at the same time, leading to them being blocked again.  How can we avoid
this happening repeatedly?
\end{enumerate}
\end{question}

%%%%%%%%%%%%%%%%%%%%%%%%%%%%%%%%%%%%%%%%%%%%%%%%%%%%%%%

\begin{answerI}
In the answers below, I'll elide code that is identical to that in the body of
the chapter.

\begin{enumerate}
\item
The following code makes philosopher~$0$ right-handed.
%
\begin{scala}
  def phil(me: Int, left: !![Cmd], right: !![Cmd]) = thread("Phil"+me){
    repeat{
      ... // Thinks, and then sits.
      if(me == 0){
        right!Pick; println(s"$me picks up right fork"); pause()
        left!Pick; println(s"$me picks up left fork"); pause()
      }
      else{
        left!Pick; println(s"$me picks up left fork"); pause()
        right!Pick; println(s"$me picks up right fork"); pause()
      }
      ... // Drops forks, and leaves.
    }
  }
\end{scala}

%%%%%

\item We can use the following channels for a philosopher to signal to the
  butler that they want to sit down or are leaving.
\begin{scala}
  private val sit, leave = new SyncChan[Unit]
\end{scala}
%
The definition of a philosopher is easily adapted to use these channels.%
\begin{scala}
  def phil(me: Int, left: !![Cmd], right: !![Cmd]) = thread("Phil"+me){
    repeat{
      think()
      sit!(); println(s"$me sits"); pause()
      ... // Picks up forks, eats, drops forks.
      leave!(); println(s"$me leaves")
    }
  }
\end{scala}
%
The butler keeps track of how many philosophers are currently seated.  When
there are already $\sm N - 1$ seated, the butler prevents the final
philosopher from sitting down by refusing a communication on |sit|.  As a
sanity-check, the code below checks that, when a philosopher leaves, there was
at least one philosopher previously seated.
%
\begin{scala}
   def butler = thread("butler"){
    var seated = 0 // Number currently seated.
    serve(
      seated < N-1 && sit =?=> { _ => seated += 1 }
      | leave =?=> { _ => assert(seated > 0); seated -= 1 }
    )
  }
\end{scala}
%
The definition of |system| includes |butler| as an additional parallel
component. 

%%%%%

\item
If a philosopher is unable to obtain their second fork, they drop their first
fork, and then wait for a random amount of time before trying again.  This
randomness makes it more likely that the philosophers get out of sync, and so
succeed next time.  
%
\begin{scala}
  /** Time to wait for second fork. */
  private def waitTime = 300+Random.nextInt(200)

  /** Time to wait after failing to get second fork. */
  private def backoffTime = 200+Random.nextInt(800)
\end{scala}

\begin{scala}
  def phil(me: Int, left: !![Cmd], right: !![Cmd]) = thread("Phil"+me){
    repeat{
      ... // Thinks, and then sits. 
      var done = false
      while(!done){
        left!Pick; println(s"$me picks up left fork"); pause()
        if(right.sendWithin(waitTime)(Pick)){
          println(s"$me picks up right fork"); pause()
          println(s"$me eats"); eat()
          left!Drop; pause(); right!Drop; pause()
          println(s"$me leaves"); done = true
        }
        else{
          println(s"$me fails to get right fork"); pause()
          left!Drop; Thread.sleep(backoffTime)
        }
      } // End of £while£ loop.
    }
  }
\end{scala}
\end{enumerate}
\end{answerI}

