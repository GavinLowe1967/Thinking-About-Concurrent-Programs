\begin{questionS}
\label{ex:barrierLog}
Provide an implementation of barrier synchronisation that operates in
$\Theta(\log n)$ time, for $n$ threads.  Hint: base your implementation on one
of the patterns from Chapter~\ref{chap:interacting-peers}.  Test your code by
adapting the test framework on the book website.
\end{questionS}

%%%%%

\begin{answerS}
This implementation is based on the tree pattern from
Chapter~\ref{chap:interacting-peers}.
%
The threads are arranged in a binary heap.  Each waits until its children
are ready, signals to its parent that all of its subtree is ready, waits for a
``go'' signal from its parent, and passes that signal on to its children.
%
\begin{scala}
class BarrierLog(n: Int) extends BarrierT{
  /** Channels by which a thread signals to its parent that it is ready.
    * Indexed by the child's identity. */ 
  private val ready = Array.fill(n)(new SyncChan[Unit])

  /** Channels by which a thread signals to its children that it can
    * continue.  Indexed by the child's (receiver's) identity. */ 
  private val go = Array.fill(n)(new SyncChan[Unit])

  /** Barrier protocol for node with identity £me£, in a system of £n£ nodes. */
  def sync(me: Int) = {
    val child1 = 2*me+1; val child2 = 2*me+2
    // Wait for £ready£ signals from both children
    if(child1 < n) ready(child1)?()
    if(child2 < n) ready(child2)?()
    // Send £ready£ signal to parent, and wait for £go£ reply, unless this is the root.
    if(me != 0){ ready(me)!(); go(me)?() }
    // Send £go£ signals to children.
    if(child1 < n) go(child1)!()
    if(child2 < n) go(child2)!()
  }
}
\end{scala}

It is easy to adapt this to a combining barrier, by changing the channels to
pass data of type~|A|, and by having each thread combine together the values
from its subtree.
%
\begin{scala}
class CombiningBarrierLog[A](n: Int, f: (A,A) => A) 
    extends CombiningBarrierT[A](n,f){
  private val ready = Array.fill(n)(new SyncChan[A])
  private val go = Array.fill(n)(new SyncChan[A])

  def sync(me: Int, x: A): A = {
    val child1 = 2*me+1; val child2 = 2*me+2; var y = x
    if(child1 < n){ val x1 = ready(child1)?(); y = f(x1,y) }
    if(child2 < n){ val x2 = ready(child2)?(); y = f(x2,y) }
    if(me != 0){ ready(me)!y; y = go(me)?() }
    if(child1 < n) go(child1)!y
    if(child2 < n) go(child2)!y
    y
  }
}
\end{scala}
%
This combines together values following the shape of the heap.

By the way, the implementations within SCL also arrange threads into a heap,
but use a special-purpose synchronisation object between each parent and
child, in the interests of efficiency.
\end{answerS}
