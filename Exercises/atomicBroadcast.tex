\begin{questionS}
\label{ex:atomicBroadcast}
A mechanism is required to allow a single sender to broadcast a message to~|n|
receivers, repeatedly.  Each broadcast should represent a synchronisation of
all $\sm n + 1$ threads: no thread's operation should return until all $\sm
n+1$ have called their operation.  Thus we require a class with the following
signature.
%
\begin{scala}
class AtomicBroadcast[A](n: Int){
  /** Synchronously send £x£ to the £n£ receivers. */
  def send(x: A): Unit

  /** Synchronously receive the sender's value. */
  def receive(): A 
}
\end{scala}
%
Implement such a class, encapsulating a server.

Produce a testing harness for your code.
\end{questionS}

% =======================================================

\begin{answerS}
My code is below.  In order to implement the synchronisation, each client
sends a reply channel to the server.  Once the server has received from all
$\sm n+1$ clients, the synchronisation can occur, and the server can reply to
all the clients.
%
\begin{scala}
class AtomicBroadcast[A](n: Int){
  private val fromSender = new SyncChan[(A, Chan[Unit])]

  private val fromReceiver = new SyncChan[Chan[A]]

  def send(x: A): Unit = {
    val c = new OnePlaceBuffChan[Unit]; fromSender!(x,c); c?()
  }

  def receive(): A = {
    val c = new OnePlaceBuffChan[A]; fromReceiver!c; c?()
  }

  private def server = thread{
    val queue = new scala.collection.mutable.Queue[Chan[A]]
    repeat{
      val (x,cs) = fromSender?()
      for(_ <- 0 until n) queue.enqueue(fromReceiver?())
      // All channels received.
      cs!(); for(_ <- 0 until n) queue.dequeue()!x
    }
  }

  fork(server)

  def shutdown() = { fromSender.close(); fromReceiver.close() }
}
\end{scala}

To test the atomic broadcast, we arrange for clients to write into the log
before and after each operation.  

\begin{scala}
  trait LogEvent
  case class BeginSend(x: Int) extends LogEvent
  case object EndSend extends LogEvent
  case object BeginReceive extends LogEvent
  case class EndReceive(x: Int) extends LogEvent
  type Log1 = Log[LogEvent] 

  val iters = 10 // Number of broadcasts per run.
  val Max = 100 // Range of values sent.

  def sender(me: Int, ab: AtomicBroadcast[Int], log: Log1) = thread{
    for(i <- 0 until iters){
      val x = Random.nextInt(Max); log.add(me, BeginSend(x))
      ab.send(x); log.add(me, EndSend)
    }
  }

  def receiver(me: Int, ab: AtomicBroadcast[Int], log: Log1) = thread{
    for(i <- 0 until iters){
      log.add(me, BeginReceive); val x = ab.receive(); log.add(me, EndReceive(x))
    }
  }
\end{scala}

The main testing code is below.  For a correct implementation, it should be
possible to identify the place in the log where each synchronisation takes
place: after all $\sm n+1$ clients have called their operation, but before any
returns.  To this end, we traverse the log (in function |checkLog|), keeping
track of the following information:
%
\begin{itemize}
\item |currentSend| records information about the value currently being sent,
  if any: it holds |Some(x)| from the point at which |send(x)| is called up
  until the synchronisation, and subsequently holds |None|;

\item |receiversCalled| records the number of receivers that have called their
  operation but not yet synchronised;

\item |receiversToReturn| records the number of receivers that have
  synchronised but not yet returned;

\item |currentVal| records the value for the most recent synchronisation,
  which receivers are expected to return.
\end{itemize}
%
If |currentSend = Some(x)| and |receiversCalled = n|, the synchronisation can
happen.  
%
\begin{scala}
  def checkLog(n: Int, events: Array[LogEvent]) = {
    def mkError(i: Int) = 
      s"Error at index $i: ${events(i)}\n"+events.mkString("\n")
    var currentSend: Option[Int] = None; var currentVal = -1
    var receiversCalled = 0; var receiversToReturn = 0
    for(i <- 0 until events.length){
      events(i) match{
        case BeginSend(x) => 
          assert(currentSend == None, mkError(i)); currentSend = Some(x)
        case EndSend => assert(currentSend == None, mkError(i))
        case BeginReceive => receiversCalled += 1
        case EndReceive(x) => 
          assert(receiversToReturn > 0, mkError(i))
          assert(x == currentVal, mkError(i)); receiversToReturn -= 1
      }
      if(currentSend != None && receiversCalled == n){
        // The threads can now synchronise.
        currentVal = currentSend.get; currentSend = None
        receiversCalled = 0; receiversToReturn = n
      }
    }
  }

  def doTest = {
    val n = Random.nextInt(10); val ab = new AtomicBroadcast[Int](n)
    val log = new Log1(n+1)
    val receivers = || (for(i <- 0 until n) yield receiver(i, ab, log))
    run(sender(n, ab, log) || receivers)
    checkLog(n, log.get)
    ab.shutdown()
  }
\end{scala}
\end{answerS}
