\begin{question}
\def\integral#1#2{\int_{#1}^{#2} f(x)\,\mbox{d}x} 
%
Recall the trapezium rule, used for estimating integrals, from Chapter~\ref{chap:trapezium}.  \emph{Adaptive quadrature} is an
alternative approach that proceeds as follows.  In order to calculate
$\integral{a}{b}$, first compute the midpoint $mid = (a+b)/2$.  Then estimate
three integrals, $\integral{a}{mid}$, $\integral{mid}{b}$ and
$\integral{a}{b}$, each using the trapezium rule with a single interval.  If
the sum of the former two estimates is within some value $\epsilon$ of the
third, then we take that third estimate as being the result.  Otherwise,
recursively estimate the integrals over the two ranges $a$ to~$mid$ and $mid$
to~$b$, and sum the results.  The following sequential algorithm captures this
idea:
%
\begin{mysamepage}
\begin{scala}
  /** Estimate the integral of £f£ from £a£ to £b£ using adaptive quadrature. */
  def estimate(f: Double => Double, a: Double, b: Double) : Double = {
    val mid = (a+b)/2.0; val fa = f(a); val fb = f(b); val fmid = f(mid)
    val lArea = (fa+fmid)*(mid-a)/2; val rArea = (fmid+fb)*(b-mid)/2
    val area = (fa+fb)*(b-a)/2
    if (Math.abs(lArea+rArea-area) < Epsilon) area
    else estimate(f,a,mid) + estimate(f,mid,b)
  }
\end{scala}
\end{mysamepage}

Write a concurrent program to implement adaptive quadrature.  Give your
program the following signature:
\begin{scala}
class Adaptive(
    f: Double => Double, a: Double, b: Double, Epsilon: Double, nWorkers: Int){
  require(a <= b)

  def apply(): Double = ...
}
\end{scala}
%
The program should use a bag of tasks with replacement: one or both recursive
calls in the sequential version can be implemented by returning tasks to the
bag.  The bag of tasks needs to store the tasks still to be performed: use
either a |Queue| or a |Stack| for this.  You will have to think carefully
about when the system should terminate.

A test harness for this program is on the \framebox{book website}.  Use this
to test your code.
%%%%% Teaching/ConcProg/Exercises/TestingHarnesses/Trapezium.AdaptiveTest.scala

Suggest ways in which your program can be made more efficient; optional:
implement them.
\end{question}

%%%%%%%%%%%%%%%%%%%%%%%%%%%%%%%%%%%%%%%%%%%%%%%%%%%%%%%%%%%%

\begin{answerI}
My code is below.  The bag of tasks is encapsulated into an object |Bag|, and
the summation of subresults is done via an object |Adder|.  The former is very
like the bag of tasks in Section~\ref{sec:quicksort-bag}.  The latter is very
like the collector in Section~\ref{sec:bag-of-tasks-encapsulation}. 
%
\begin{scala}
class Adaptive(
    f: Double => Double, a: Double, b: Double, Epsilon: Double, nWorkers: Int){
  require(a <= b)

  // Interval on which to work
  private type Task = (Double, Double)

  /** The bag, that keeps track of jobs pending. */
  object Bag{
    /** Channel to the workers, to distribute tasks.  */
    private val toWorkers = new SyncChan[Task]

    /** Channel from the workers, to return subtasks. */
    private val toBag = new SyncChan[Task]

    /** Channel to indicate that a worker has completed a task. */
    private val doneC = new SyncChan[Unit]

    /** Get a task. */
    def get(): Task = toWorkers?()

    /** Add t to the bag. */
    def add(t: Task) = toBag!t

    /** Indicate that a task is done. */
    def done() = doneC!()

    /** Server controlling the bag. */
    private def server = thread("bag"){
      val stack = new scala.collection.mutable.Stack[Task]
      stack.push((a,b)); var busyWorkers = 0 // # workers with tasks
      serve(
        stack.nonEmpty && toWorkers =!=> { busyWorkers += 1; stack.pop() }
          | busyWorkers > 0 && toBag =?=> { t1 => stack.push(t1) }
          | busyWorkers > 0 && doneC =?=> { _ => busyWorkers -= 1 }
      )
      assert(busyWorkers == 0 && stack.isEmpty)
      toWorkers.endOfStream() 
    }

    fork(server)
  } // End of Bag.

  /** The adder object, responsible for adding workers' subresults. */
  object Adder{
    /** Channel from the workers to the adder thread, to add up subresults. */
    private val toAdder = new SyncChan[Double]

    /** Channel to get final result. */
    private val getC = new SyncChan[Double]

    /** Add x to the overall result. */
    def add(x: Double) = toAdder!x

    /** Get the final result. */
    def get = getC?()
    
    /** Server to receive results from workers and add up the results. */
    private def adder = thread("adder"){
      var result = 0.0
      for(_ <- 0 until nWorkers){ result += (toAdder?()) }
      getC!result
    }

    fork(adder)
  } // End of Adder.

  /** A worker, that receives arguments from the server, either estimates the
    * integral directly or returns new tasks to the bag. */
  private def worker = thread("worker"){
    // If £a < b£, then this worker is responsible for the task £(a,b)£.  If 
    // £a = b£, then the worker has no current task.
    var a = -1.0; var b = -1.0
    var mySum = 0.0 // Current sum this worker has calculated.
    repeat{
      if(a == b){ val p = Bag.get(); a = p._1; b = p._2; assert(a < b) }
      val mid = (a+b)/2.0; val fa = f(a); val fb = f(b); val fmid = f(mid)
      val larea = (fa+fmid)*(mid-a)/2; val rarea = (fmid+fb)*(b-mid)/2
      val area = (fa+fb)*(b-a)/2
      if (Math.abs(larea+rarea-area) < Epsilon){ 
        mySum += area; b = a; Bag.done()
      }
      else{ // Return £(a,mid)£ to the bag, and carry on with £(mid,b)£. 
        Bag.add((a,mid)); a = mid 
      }
    }
    Adder.add(mySum)
  }

  def apply(): Double = {
    run( || (for (i <- 0 until nWorkers) yield worker) )
    Adder.get
  }
}
\end{scala}

One inefficiency is that |f| is evaluated on most arguments multiple times.
We could avoid this by storing in the bag tuples |(a, b, fa, fb)| where
\SCALA{fa = f(a)} and \SCALA{fb = f(b)}.  Likewise, the above loop could be
rewritten to avoid recalculating these values on each iteration, adding a loop
invariant that if |a < b| then \SCALA{fa = f(a)} and \SCALA{fb = f(b)}.  We
could extend this to the result of each trapezium rule calculation, adding to
each tuple \SCALA{area = (fa+fb)*(b-a)/2}, and again avoiding recalculations
on each iteration of the above loop.
\end{answerI}


%%%%%



%% We will build a system as illustrated below.
%% %
%% \begin{center}
%% \begin{tikzpicture}
%% \draw(0,0) node[draw] (bag) {\scalashape bag};
%% \draw (3,0) node[draw] (worker) {\scalashape workers};
%% \draw 
%% %% \draw (worker.north east)++(0.2,0.2) node (control1) {};
%% %% \draw (worker.south east)++(0.2,0.2) node (control2) {};
%% %% \draw (worker.north west)++(0.2,0) -- ++ (0,0.2) -- (control1) --
%% %% (worker.south east)++(0.2,0.2); %(control2);
%% \end{tikzpicture}
%% \end{center}
%% My code is below.  To support termination, the bag keeps track of the
%% number of workers currently working on a task; a worker informs the
%% bag that it has completed a task via the channel |done|.  When the bag
%% is empty and there are no busy workers, the bag closes the channels to
%% terminate the system.
%% %
%% \begin{scala}
%% /** Calculating integral, using trapezium rule, adaptive quadrature, and bag
%%   * of tasks pattern. */
%% class Adaptive(
%%   f: Double => Double, a: Double, b: Double, Epsilon: Double, nWorkers: Int){
%%   require(a <= b)

%%   // Interval on which to work
%%   private type Task = (Double, Double)

%%   /** Channel from the controller to the workers, to distribute tasks.  We
%%     * create the channel freshly for each run of the system. */
%%   private val toWorkers = new BuffChan[Task] (nWorkers)

%%   /** Channel from the workers to the bag, to return subtasks. */
%%   private val toBag = new BuffChan[(Task,Task)](nWorkers)

%%   /** Channel from the workers to the adder thread, to add up subresults. */
%%   private val toAdder = new BuffChan[Double](nWorkers)

%%   /** Channel to indicate to the bag that a worker has completed a task. */
%%   private val done = new BuffChan[Unit](nWorkers)

%%   /** A client, who receives arguments from the server, either estimates the
%%     * integral directly or returns new tasks to the bag. */
%%   private def worker = thread("worker"){
%%     repeat{
%%       val (a,b) = toWorkers?()
%%       val mid = (a+b)/2.0; val fa = f(a); val fb = f(b); val fmid = f(mid)
%%       val larea = (fa+fmid)*(mid-a)/2; val rarea = (fmid+fb)*(b-mid)/2
%%       val area = (fa+fb)*(b-a)/2
%%       if (Math.abs(larea+rarea-area) < Epsilon) toAdder!area
%%       else toBag!((a,mid), (mid,b)) 
%%       done!(())
%%     }
%%   }

%%   /** The bag, that keeps track of jobs pending. */
%%   private def bag = thread("bag"){
%%     val stack = new scala.collection.mutable.Stack[Task]
%%     stack.push((a,b))
%%     var busyWorkers = 0 // # workers with tasks

%%     serve(
%%       stack.nonEmpty && toWorkers =!=> { busyWorkers += 1; stack.pop }
%%         | busyWorkers > 0 && toBag =?=> { case (t1,t2) => stack.push(t1); stack.push(t2) }
%%         | busyWorkers > 0 && done =?=> { _ => busyWorkers -= 1 }
%%     )
%%     // busyWorkers == 0 && stack.isEmpty
%%     toWorkers.endOfStream; toAdder.endOfStream
%%   }

%%   private var result = 0.0

%%   // Server to receive results from workers and add up the results
%%   private def adder = thread("adder"){ repeat{ result += (toAdder?()) } }

%%   def apply(): Double = {
%%     val workers = || (for (i <- 0 until nWorkers) yield worker)
%%     run(workers || bag || adder)
%%     result
%%   }
%% }
%% \end{scala}

%% Here are four ways this could be made more efficient.  Each aims to reduce
%% the number of communications (since communications are expensive).
%% %
%% \begin{itemize}
%% \item
%% Above the worker returns two tasks to the bag, and then (on the next
%% iteration) gets one back, possibly one of the tasks it just put there.  It
%% would be more efficient to just return one and to continue calculating with
%% the other.  In particular, the bag is likely to be a bottleneck, and this
%% reduces the load on the bag.

%% \item
%% The trapezium rule is applied to most intervals twice: once before that
%% interval is placed in the bag, and once when it is retrieved.  It would be
%% more efficient to put the calculated estimate into the bag along with the
%% interval, so it doesn't need to be recalculated.

%% \item
%% We could arrange for a |done| message to be sent only when the worker
%% does not return subtasks to the bag: the bag can infer that the worker
%% is done in the other case.

%% \item 
%% Each worker could accumulate sub-results in a thread-local variable, and send
%% to the adder only at the end. 
%% \end{itemize}
%% \end{answer}

