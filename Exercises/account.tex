\begin{questionS}
\label{exercise:account}
Consider an object representing a bank account, from the following class.
%
\begin{scala}
class Account{
  private var balance = 0

  def credit(value: Int) = atomically{ balance += value }

  def canDebit(value: Int): Boolean = atomically{ balance >= value }

  def debit(value: Int) = atomically{ balance -= value }
}
\end{scala}
%
The ``|atomically|'' pseudocode is intended to indicate that each
procedure is performed atomically (we will see how to do this in a
later chapter).

A thread that wants to perform a debit should first of all call
\SCALA{canDebit}, to avoid the account from going overdrawn; for example
%
\begin{scala}[showstringspaces=false]
  if(account.canDebit(value)) account.debit(value)
  else println("Debit not allowed!")
\end{scala}

What can go wrong if two threads execute the above code at the same
time?  Sketch a solution to this problem.
\end{questionS}

%%%%%

\begin{answerS}
Suppose the current balance is 100, and both threads want to debit 100;
clearly only one should succeed.  Both threads could call
\SCALA{canDebit(100)}, getting back the result \SCALA{true}.  They would then
both call \SCALA{debit(100)}, leading to the account balance becoming $-$100.
This is a \emph{time-of-check to time-of-use} (TOCTTOU) problem: the check
that there is enough money in the account is no longer valid when the debit is
performed.

The point is that the \SCALA{canDebit} action of one thread and the
\SCALA{debit} action of the other are not independent.  This leads to a race
condition.

The obvious way to avoid this problem is to combine the \SCALA{canDebit} and
\SCALA{debit} actions into a single atomic action within the \SCALA{Account}
class.  Something like
%
\begin{scala}
  /** Attempt to debit value.  Return true if successful. */
  def tryDebit(value: Int): Boolean = atomically{
    if(balance >= value){ balance -= value; true } else false
  }
\end{scala}
% 
The calling code (outside the |Account| class) can do the right thing with the
boolean result. 
\end{answerS}

