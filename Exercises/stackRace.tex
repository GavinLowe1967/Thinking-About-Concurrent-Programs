\begin{questionS}
\label{exercise:stackRace}
Consider the simple implementation of a stack of |Int|s in
Figure~\ref{fig:stackRace}.  This encapsulates a |List[Int]|, holding the
current contents of the stack.  Scala box~\ref{sb:Lists} describes the basics
of the |List| class in Scala. 

%%%%%

\begin{figure}
\begin{scala}
class IntStack{
  private var st = List[Int]()

  /** Push £x£ onto the stack. */
  def push(x: Int) = { st = x :: st }

  /** Pop a value off the stack.  Precondition: the stack is not empty. */
  def pop(): Int = {
    require(!st.isEmpty); val result = st.head; st = st.tail; result
  }

  /** Is the stack empty? */
  def isEmpty: Boolean = st.isEmpty
}
\end{scala}
\caption{A faulty implementation of a stack.}
\label{fig:stackRace}
\end{figure}

%%%%%

\begin{scalaBox}{Lists}
\label{sb:Lists}
The type |List[A]| represents the type of immutable lists containing data of
type~|A|.

A list containing values~|x|, |y| and~|z| can be defined as \SCALA{List[A](x,
  y, z)}; the type parameter~|A| can normally be omitted, as the compiler can
infer it.

If |xs| has type |List[A]|, and |x| has type~|A|, then |x :: xs| is a new
|List[A]| containing~|x| followed by the elements of~|xs|.  The expression
|xs.isEmpty| returns a |Boolean| indicating whether |xs| is empty.  If |xs| is
not empty, then |xs.head| gives its first element, and |xs.tail| gives the
list containing all of~|xs| except the first element.
\end{scalaBox}

%%%%%

The implementation works correctly when used sequentially, but not when it is
used by several threads concurrently.  List as many different things that can
go wrong as possible.  For the purpose of this question, you can ignore errors
caused by caching or compiler optimisations.

You might be wondering what ``correct'' means in this case.  The operations
should appear to take place in a one-at-a-time order, without interfering with
one another, and giving the results one would expect from that order.
Further, the point at which each operation takes place should be at some point
between when the operation is called and when it returns.  That means that if
one operation returns before another is called, they should take place in the
same order; but if two operations overlap in time, they can appear to take
place in either order.

For example, if there are concurrent calls of |push(2)| and |push(3)|,
starting in a state where the stack holds~|xs|, then the final state should
either be |2::3::xs| (the |push(3)| seems to take place before the
|push(2)|), or |3::2::xs| (the |push(2)| seems to take place before the
|push(3)|).  Similarly, if there are concurrent calls to |push(3)| and |pop|,
starting in a state where the stack holds |2::xs|, then either the |pop|
should return~|2| and the final state should be |3::xs| (the |pop| seems to
take place before the |push|), or the |pop| should return~|3| and the final
state should be |2::xs| (the |push| seems to take place before the |pop|).
This property is called \emph{linearization}: we will discuss it in more
detail later in the book.
\end{questionS}

%%%%%%%%%%%%%%%%%%%%%%%%%%%%%%%%%%%%%%%%%%%%%%%%%%%%%%%

\begin{answerS}
The implementation can go wrong in a numbre of ways, described below.  I~will
write ``|pop|-read-1'', ``|pop|-read-2'' and ``|pop|-read-3'' for the three
reads of~|st| within~|pop|.
%
\begin{enumerate}
\item First, the interface of the class does not seem suitable for a stack
  that is used concurrently.  A typical way in which a stack might be used in
  a sequential program is via code such as
\begin{scala}
  if(!stack.isEmpty){ 
    val x = stack.pop(); ... // Do something with £x£.
  }
  else ... // Handle the empty stack.
\end{scala}
However, if this code is used concurrently, the following is possible:
thread~$A$ finds that the stack is not empty; thread~$B$ performs |pop|s to
empty the stack; thread~$A$ attempts a |pop|, and the |require| statement
throws an exception.  Note that this can occur even if each individual
operation call acts atomically.  This is a time-of-check to time-of-use error,
as in the answer to Exercise~\ref{exercise:account}.

It would be better for |pop| to have a signature such as
\begin{scala}
  def pop(): Option[Int]
\end{scala}
either returning a result of the form~|Some(x)|, where |x| is the value
popped, or a result~|None| to indicate that the stack is empty.  (The type
|Option[Int]| contains the union of such values.)  This combines the |isEmpty|
and |pop| operations from the given class.  Client code can treat the result
appropriately in each case.  

\item (This item is somewhat related to the previous.)  Suppose a thread calls
  |pop| when the stack is non-empty, and the subsequent |require| statement
  passes.  However, if another thread then performs |pop|s to make the stack
  empty, |pop|-read-2 will return the empty list, and the expression |st.head|
  will throw an exception.  Alternatively, if the stack becomes empty between
  |pop|-read-2 and |pop|-read-3, then the expression |st.tail| will throw an
  exception.

\item Suppose two threads call |push(2)| and |push(3)| concurrently, and
  suppose both read the same value of~|st|.  Each will write to |st|,
  corresponding to its own operation.  But whichever thread writes first will
  have its value overwritten by the other thread, so the former thread's value
  will be lost.

\item Suppose two threads call |pop| concurrently.  If both read the same
  value of |st| for their |pop|-read-2,  then both will end up
  returning the same value.  Further, it's possible that either one or two
  values are removed from the stack, depending upon the relative orders of the
  reads and writes of |st| within |st = st.tail|.  

\item Now suppose two threads call |push(3)| and |pop|, respectively, and both
  the |push|-read and |pop|-read-2 obtain the same value of~|st|, of the
  form~|2::xs|.  Then the |pop| will return~|2|, but the final value of~|st|
  might be |3::2::xs| (if the |pop|-read-3 and |pop|-write precede the
  |push|-write), or |2::xs| (if the |push|-write precedes the |pop|-read-3 and
  |pop|-write), or |xs| (if~the |pop|-read-3 precedes the |push|-write, which
  precedes the |pop|-write).
\end{enumerate}

However, there is no race concerning the |isEmpty| operation and either a
|push| or a |pop|.  The |isEmpty| will see the result of a |push| if and only
if its read follows the |push|-write, in which case it seems to take place
after the |push|.  Similarly, the |isEmpty| will see the result of a |pop| if
and only if its read follows the |pop|-write, in which case it seems to take
place after the |pop|.
\end{answerS}
