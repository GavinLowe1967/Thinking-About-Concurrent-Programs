\chapter{Clients and Servers}

%%%%%

A common pattern in concurrent systems is that of clients and servers.  A
\emph{server} is a thread or process that repeatedly handles requests from
\emph{client}s.

Many modules, such as concurrent datatypes, can be easily implemented in
this way.

Also synchronisations.

This is a pattern that is often used in the implementation of operating
systems.

It's the pattern that is prevalent in networked systems.

\framebox{More here.}

%%%%%%%%%%%%%%%%%%%%%%%%%%%%%%%%%%%%%%%%%%%%%%%%%%%%%%%

\section{A resource allocation server}

We consider the example of a server that is responsible for managing and
allocating multiple resources of the same kind.  Such servers are common in
operating systems, where the resources might be memory or file blocks.
However, the pattern is more widely applicable.  Clients acquire resources for
use, and later return them to the server.

One problem we need to address is: what should happen if a client requests a
resource and there is none available?  We choose to use an |Option[Resource]|
value: the client receives a value |Some(r)| to indicate that resource |r| has
been acquired, or the value |None| to indicate that no resource was available.
%
In the latter case, it is up to the client to decide what to do: it might try
again later, or might throw an exception.  
%
An alternative approach would be for the server to queue the request until it
can be serviced; we will consider this approach later in the section.

%%%%%

Thus we will define a server with the interface in Figure~\ref{fig:RAServer}.
We represent resources by |Int|s, and we also assume each client has an |Int|
identity, which they include in operation calls.  There are operations for a
client to obtain a resource, or to return a resource.  There is also an
operation to shut down the server (although this could be considered
optional). 

%%%%%

\begin{figure}
\begin{scala}
object RAServer{
  /** Client identities. */
  type ClientId = Int

  /** Resource identities. */
  type Resource = Int
}

import RAServer._

/** A resource server. */
trait RAServer{
  /** Request a resource. */
  def requestResource(me: ClientId): Option[Resource]

  /** Return a resource. */
  def returnResource(me: ClientId, r: Resource) 

  /** Shut down the server. */
  def shutdown(): Unit
} 
\end{scala}
\caption{The interface for the resource allocation module.}
\label{fig:RAServer}
\end{figure}


%%%%% First implementation

The first implementation is in Figures~\ref{fig:RAServer1-1}
and~\ref{fig:RAServer1-2}.  We assume clients have identities in the range
$\interval0{\sm{clients}}$ and resources have identities in the range
$\interval0{\sm{numResources}}$, where |clients| and |numResources| are known
in advance.

%%%%%

\begin{figure}
\begin{scala}
/** A resource server. 
  * This version assumes the number of clients is known initially. 
  * @param £clients£ the number of clients.
  * @param £numResources£ the number of resources.  */
class RAServer1(clients: Int, numResources: Int) extends RAServer{

  /* Channel for requesting a resource. */
  private val acquireRequestChan = new SyncChan[ClientId]

  /* Channels for optionally returning a resouce, indexed by client identities. */
  private val acquireReplyChan = 
    Array.fill(clients)(new SyncChan[Option[Resource]])

  /* Channel for returning a resource. */
  private val returnChan = new SyncChan[Resource]

  /* Channel for shutting down the server. */
  private val shutdownChan = new SyncChan[Unit]

  /** Request a resource. */
  def requestResource(me: ClientId): Option[Resource] = {
    acquireRequestChan!me  // Send request.
    acquireReplyChan(me)?() // Wait for response.
  }

  /** Return a resource. */
  def returnResource(me: ClientId, r: Resource) = returnChan!r

  /** Shut down the server. */
  def shutdown() = shutdownChan!()

  private def server = thread("server"){ ...  } // See Figure £\ref{fig:RAServer1-2}£

  fork(server)
}
\end{scala}
\caption{The first implementation of the resource allocation module (part~1).}
\label{fig:RAServer1-1} 
\end{figure}

%%%%%

To request a resource, a client sends a message on channel
|acquire|\-|Request|\-|Chan|; this channel is shared between all clients.  It
then receives a reply from the server, which is the result of the operation.
In this version, we assume one channel per client, in array
|acquireReplyChan|; this ensures that the expected client receives the
resource.

To return a resource, the client simply sends the resource's identity on the
channel |returnChan|.  Likewise, to shut down the server, a message is sent on
the channel |shutDownChan|.

%%%%%

\begin{figure}
\begin{scala}
  private def server = thread("server"){
    // Record whether resource £i£ is available in £free(i)£.
    val free = Array.fill(numResources)(true); var done = false
    serve(!done)(
      acquireRequestChan =?=> { c => 
	// Find free resource.
	var r = 0
	while(r < numResources && !free(r)) r += 1
	if(r == numResources) acquireReplyChan(c)!None
        else{  // Pass resource £r£ back to client £c£.
	  free(r) = false; acquireReplyChan(c)!Some(r)
        }
      }
      | returnChan =?=> { r => free(r) = true }
      | shutdownChan =?=> { _ => done = true }
    )
    acquireRequestChan.close(); returnChan.close(); shutdownChan.close()
  }
\end{scala}
\caption{The first implementation of the resource allocation module: the
  server}
\label{fig:RAServer1-2}
\end{figure}

%%%%%

The server (Figure~\ref{fig:RAServer1-2}) keeps track of the resources that are
currently free in an array \SCALA{free}.  Note that |free| is declared as a
local variable within the definition of~|server|.  This ensures that only the
server has access to |free|, which avoids race conditions. This also clarifies
the intended use of~|free|: if we had declared it as an object variable, it
would have been less clear.

The main loop is defined using a |serve| with a boolean guard |!done|.  This
loop will continue to iterate while the guard is true; i.e.~it will terminate
when |done| is true.  On each iteration, it is willing to receive a message on
any of its input channels.  If it receives a request for a resource from
client~|c|, it performs a straightforward search through |free|, and returns a
suitable reply to~|c|, updating |free| if appropriate.  If it receives a
returned resource, is simply updates |free| to mark that resource as free.  If
it receives a shutdown message, it sets |done| to true to exit the loop, at
which point it closes its input channels to signal to clients.

Above, we chose to use different channels for different requests to the
server.  An alternative is to use a single channel, and multiple types of
request.  Such an approach seems more natural in a networked application.  In
the resource allocation module, we could define the following:
%
\begin{scala}
  private trait Request
  private case class Acquire(c: ClientId) extends Request
  private case class Return(r: Resource) extends Request
  private case object Shutdown extends Request

  private val requestChan = new SyncChan[Request]
\end{scala}
%
The server would then receive on |requestChan|, and pattern match on the value
received:
%
\begin{scala}
  requestChan?() match{
    case Acquire(c) => ...
    case Return(r) => ...
    case Shutdown => ...
  }
\end{scala}
 % intro, resource allocation example, asynchrony
\section{Testing the resource allocation server}

We now want to test the resource allocation server.  The property we want to
test is that two clients never hold the same resource: if a resource~$r$ is
allocated to client~$c_1$, it can't be allocated to another client~$c_2$ until
$c_1$ has returned it.

We can perform this test by running a number of clients that randomly request
and return resources, and log those actions.  We can then test whether the
resulting history of log events is valid.

%%% Note: we could use linearisation testing here; but there are a few
%%% wrinkles, so this probably isn't the best example for introducing the
%%% technique. 

%%%%%

We use a log of type |Log[LogEvent]|, where |LogEvent| is defined in
Figure~\ref{fig:RATest-1}.  
%% Subsequently, we can use the |get| method on the
%% log, to get the results of logging as an |Array[LogEvent]|.
%
A single client thread for testing is also defined in
Figure~\ref{fig:RATest-1}.   This uses a
queue |got| to keep track of the resources it currently holds.  On each
iteration it either tries to acquire another resource, or return a resource if
it has one.  In each case, it adds a suitable event to the log.
%%%%%

\begin{figure}
\begin{scala}
  trait LogEvent
  case class GotResource(c: ClientId, r: Resource) extends LogEvent
  case class ReturnedResource(c: ClientId, r: Resource) extends LogEvent

  /** A client */
  def client(me: ClientId, resourceServer: RAServer, log: Log[LogEvent]) = thread{
    var got = new scala.collection.mutable.Queue[Resource]()
    val random = new scala.util.Random
    for(_ <- 0 until iters){
      if(random.nextInt(2) == 0){ // Acquire new resource
	resourceServer.requestResource(me) match{
          case Some(r) =>  log.add(me, GotResource(me, r)); got.enqueue(r)
          case None => {}  // try again
        }
      }
      else if(!got.isEmpty){     // Return resource.
	val r = got.dequeue()
        log.add(me, ReturnedResource(me, r))
	resourceServer.returnResource(me, r)
      }
    }
  }
\end{scala}
\caption{Testing a resource allocation server (part~1).}
\label{fig:RATest-1}
\end{figure}

%%%%%

Note that we log acquiring resources \emph{after} the resource is acquired,
and log returning resources \emph{before} the resource is returned.  This
means that the period during which the log indicates that the thread holds a
resource is a \emph{subset} of the true time.
%
This is necessary to avoid false positives (i.e.~the testing signalling an
error, when in fact there is none).  If we were to return a resource before
logging the return, it is possible that a client would be slow in logging the
return, and that another client might have obtained the same resource in the
mean time, giving a false positive.
%
However, there is a small risk of false negatives (i.e.~the testing failing to
detect an incorrect behaviour).  But we would expect these to be rare: if
there is a bug, doing enough testing will almost certainly also produce true
positives.


%%%%%

\begin{figure}
\begin{scala}
  /** Check that events represents a valid log.  */
  def checkLog(events: Array[LogEvent]): Boolean = {
    val held = Array.fill(numResources)(false); var error = false; var i = 0
    while(i < events.size && !error){
      events(i) match{
        case GotResource(_, r) =>
          if(held(r)){ // Error!
            println("Error found:\n"+events.take(i+1).mkString("\n"))
            error = true
          }
          else held(r) = true
        case ReturnedResource(_, r) => held(r) = false
      }
      i += 1
    }
    !error
  }

  /** Run a single test. */
  def runTest(resourceServer: RAServer) = {
    val log = new Log[LogEvent](numClients)
    run(|| (for (i <- 0 until numClients) yield client(i, resourceServer, log)))
    if(!checkLog(log.get)) sys.exit()
    resourceServer.shutdown()
  }
\end{scala}
\caption{Testing a resource allocation server (part~2).}
\label{fig:RATest-2}
\end{figure}

%%%%%

The function |runTest(resourceServer)| (Figure~\ref{fig:RATest-2}) performs a
single test on |resourceServer|.  This runs |numClients| clients, sharing a
log.  It uses the |checkLog| function (described below) to check whether the
log represents a correct execution.  It shuts down |resourceServer|, to
terminate the server thread: this is necessary when we perform many tests on
different |RAServer| objects; if we did not do this, we would end up with many
server threads still running, which would consume system resources.

The |checkLog| function traverses the log, keeping track (in array~|held|) of
which resources are currently held by threads.  If it finds that a resource
that is currently held is allocated a second time, this constitutes an error:
it prints the log up to that point to help with debugging.

The |main| function calls |runTest| many times. 

\begin{instruction}
Study the details of the test program.
\end{instruction}

%%%%%%%%%%%%%%%%%%%%%%%%%%%%%%%%%%%%%%%%%%%%%%%%%%%%%%% 

\section{Supporting arbitrary many clients}

So far, we have assumed a fixed number of clients, with this number known when
the resource server is created.  This allowed us to create one reply channel
per client.  However, in many circumstances, this assumption doesn't hold.

If we do not know the number of clients, an alternative is for each client to
create a new reply channel for each request, and to send the reply channel
within the request.  The server can then reply on this reply channel.  
%
This technique is illustrated in Figure~\ref{fig:RAServer-replyChan} (eliding
code that is unchanged from earlier). 

%%%%%

\begin{figure}
\begin{scala}
  private type ReplyChan = Chan[Option[Resource]]

  private val acquireRequestChan = new SyncChan[ReplyChan]

  def requestResource(me: ClientId): Option[Resource] = {
    val replyChan = new OnePlaceBuffChan[Option[Resource]]
    acquireRequestChan!replyChan  // Send request.
    replyChan?() // Wait for response.
  }

  private def server = thread{
    ...
    serve(
      acquireRequestChan =?=> { replyChan => 
	var r = 0
	while(r < numResources && !free(r)) r += 1
	if(r == numResources) replyChan!None
        else{ free(r) = false; replyChan!Some(r) }
      }
      | ... // As previously. 
    )
  }
\end{scala}
\caption{The resource server, using reply channels.}
\label{fig:RAServer-replyChan}
\end{figure}

%%%%%

This implementation can be tested in the same way as the previous
implementation, since it provides the same interface and functionality: only
internal details have changed. 

\begin{instruction}
Study the details of the new implementation.
\end{instruction}

We will use reply channels like this in most subsequent client-server
examples. 

With this implementation, the client sends a request, and then stops and waits
for a response.  In some cases, the client could do some useful work while
waiting for the request to be serviced:
%
\begin{scala}
  <make request>
  <other useful work>
  <obtain response>
\end{scala}
%
We could change the interface of the object to support this, decoupling the
request for a resource from obtaining the resource.
The fact that we have used buffered reply channels means that the server can
send the response even if the client isn't yet ready for it.  

%%%%%%%%%%%%%%%%%%%%%%%%%%%%%%%%%%%%%%%%%%%%%%%%%%%%%%%

\section{Buffering requests}

Previously, if the server couldn't meet a request, it immediately replied with
a |None| value.  An alternative is to store such requests until they can be
met.  
%
Figure~\ref{fig:totalRAServer} gives code outlining this approach.  We omit
code that is identical to earlier.

It makes sense to change the |requestResource| operation to return a result of
type |Resource| (rather than |Option[Resource]|), since it can never return
the |None| value.  The types of the channels are adapted accordingly, but
otherwise the client-side code is unchanged.

%%%%%

\begin{figure}
\begin{scala}
  private type ReplyChan = Chan[Resource]

  /* Channel for requesting a resource. */
  private val acquireRequestChan = new SyncChan[ReplyChan]

  /** Request a resource. */
  def requestResource(me: ClientId): Resource = {
    val replyChan = new OnePlaceBuffChan[Resource]
    acquireRequestChan!replyChan  // Send request.
    replyChan?() // Wait for response.
  }

  private def server = thread{
    // Record whether resource £i£ is available in £free(i)£.
    val free = Array.fill(numResources)(true)
    // Reply channels for requests that cannot be served immediately.
    val pending = new scala.collection.mutable.Queue[ReplyChan]
    // Invariant: if £pending£ is non-empty, then all entries in £free£ are false.
    var done = false

    serve(!done)(
      acquireRequestChan =?=> { replyChan =>
	var r = 0
	while(r < numResources && !free(r)) r += 1
	if(r == numResources) 
          pending.enqueue(replyChan) // Client has to wait.
        else{  // Pass resource r back to client. 
	  free(r) = false; replyChan!r
        }
      }
      | returnChan =?=> { r =>
          if(pending.nonEmpty)
            pending.dequeue()!r // Allocate £r£ to blocked client.
          else free(r) = true
      }
      | shutdownChan =?=> { _ => done = true }
    )
    acquireRequestChan.close(); returnChan.close(); shutdownChan.close()
  }
\end{scala}
\caption{A total resource server (code similar to earlier is omitted).}
\label{fig:totalRAServer}
\end{figure}

The server maintains a queue |pending| that holds the reply channels
corresponding to blocked |requestResource| operations.  This queue is nonempty
only if all entries in |free| are false: otherwise there would be a free
resource that could be allocated to the blocked operation.  When the server
receives a request for a resource, but there is no free resource, it adds the
reply channel to |pending|.  When a resource is returned, if |pending| is
nonempty, the server dequeues the first reply channel, and uses it to send the
resource to the corresponding channel.

\begin{instruction}
Study the details of the implementation.
\end{instruction}

With this implementation, the |requestResource| operation can take effect only
when there is an unallocated resource: we say that it is a \emph{partial
  operation}.  By contrast, in the previous implementations, the operation was
a \emph{total operation}, that could take effect in any state.  (The terms
``partial'' and ``total'' are by analogy with partial and total functions,
which are defined on some or all potential arguments, respectively.)

This approach can lead to a deadlock if all resources are allocated, and all
clients are requesting more.  There is no good solution to this problem:
sometimes there are simply insufficient resources available.  

The possibility of deadlock also has implications for testing.  We want a
testing harness that can run unsupervised, maybe for a long time, without
deadlocking.  To achieve this, we need to ensure that the clients never
request so many resources that the system deadlocks; but we do need to ensure
that the tests cover cases where all the resources are allocated.

My approach is to arrange that each client in the testing harness holds at
most two resources, and when it does hold two, to return one on the next
iteration.  In addition, I arrange that there are slightly more resources than
clients.  This means that if the server is unable to immediately satisfy a
request, and so that request is queued, then there must be some other client
that holds two resources, and so will return one on the next iteration.  With
such a definition for clients, the testing harness is almost identical to
previously. 

It is reasonable to assume that the server is able to keep up with clients,
i.e.,~that it is able to deal with each message faster, on average, than
clients attempt to send new messages.  This implies that when a client
attempts to send a message to the server, the server eventually receives it.
Put another way, we assume that clients don't make requests that the server is
overwhelmed. 


The use of a queue then ensures a degree of fairness.  Assuming resources are
repeatedly returned, once a client attempts to obtain a resource, its message
on |acquireRequestChan| will eventually be received, so the reply channel will
be enqueued; and eventually the reply channel will reach the front of the
queue, and the request serviced.

%%%%%%%%%%%%%%%%%%%%%%%%%%%%%%%%%%%%%%%%%%%%%%%%%%%%%%%%%%%%


There is a simpler way to achieve the same effect of buffering requests that
cannot be served immediately, albeit it the cost of losing this fairness.
Rather than the server explicitly queueing pending requests, the clients can
be blocked on the |acquireRequestChan| channel until there is a resource free.
This can be achieved simply by adding a variable that keeps track of the
number of free resources, and a suitable guard on the |acquireRequestChan|
branch of the |serve|, to block communications when there are no free
resources.%
%
\begin{scala}
  private def server = thread{
    val free = Array.fill(numResources)(true)
    // Invariant: £numFree£ is the number of free resources, i.e. the number 
    // of £true£s in £free£.
    var numFree = numResources
    serve(
      numFree > 0 && acquireRequestChan =?=> { replyChan => 
	var r = 0; while(r < numResources && !free(r)) r += 1
	assert(r < numResources); free(r) = false; numFree -= 1
        replyChan!Some(r)
      }
      | returnChan =?=> { r => free(r) = true; numFree += 1 }
      | shutdownChan =?=> { ... } // As before.
    )
  }
\end{scala}
%
The assertion in the |acquireRequestChan| branch holds because of
the invariant concerning |numFree|.

When the server has no free resources, all client threads wanting to obtain a
resource will be blocked on |acquireRequestChan|.
%
When a resource is returned, those clients will compete to communicate on the
channel.  The one that is able to send first will obtain the resource.
%
This means that this version is probably less fair than the previous one: an SCL
channel is not guaranteed to be fair between the threads trying to use it.
Whether this is important depends on the use case.






 % testing; reply channels; buffering; 
\section{Example: an exchanger}

We now consider how to use a server to solve a synchronisation problem.  An
\emph{exchanger} is a concurrent object that allows pairs of threads to
exchange values: each thread should pass in a value, and receive the other
thread's value back, via an operation
\begin{scala}
  def exchange(x: A): A 
\end{scala}
(where |A| is a polymorphic type parameter of the object).  This exchange
represents a \emph{synchronisation}: the two executions of |exchange| must
necessarily overlap in time.

A straightforward implementation of an exchanger using a server is in
Figure~\ref{fig:exchanger}.  The client creates a reply channel, sends its
value and the reply channel to the server, and waits to receive the other
thread's value on its reply channel.  The server repeatedly receives two
requests, and passes each value to the other thread.

%%%%%

\begin{figure}
\begin{scala}
class Exchanger[A]{
  /** Reply channels, for the server to return results to clients. */
  private type ReplyChan = OnePlaceBuffChan[A]

  /** Channel from clients to the server. */
  private val toServer = new SyncChan[(A, ReplyChan)]

  /** Exchange x with another thread. */
  def exchange(x: A): A = {
    val c = new ReplyChan; toServer!(x, c); c?()
  }

  /** The server thread. */
  private def server = thread("Exchanger"){
    repeat{
      val (x1, c1) = toServer?(); val (x2, c2) = toServer?()
      c1!x2; c2!x1
    }
  }

  fork(server)

  /** Shut down the server. */
  def shutdown() = toServer.endOfStream()
}
\end{scala}
\caption{An exchanger, using a server.}
\label{fig:exchanger}
\end{figure}

%%%%%

\begin{instruction}
Make sure you understand the details of the exchanger.
\end{instruction}

%%%%%

We now consider testing.  The property we need to check is that if
thread~$t_1$ receives thread~$t_2$'s value, then $t_2$ receives $t_1$'s value.
Note that this necessarily implies that the two operation executions overlap
in time: if, say, $t_1$ returned before $t_2$ called the operation, there
would be no way for $t_1$ to obtain $t_2$'s value.  

We can test this property by arranging for each thread to store the value it
receives, and subsequently trying to pair up the threads that exchanged.  It
is not difficult to come up with an algorithm to attempt this pairing.
However, if two threads pass in the same value, there might be multiple
candidates for pairing, which complicates the algorithm.

We can make things easier by arranging for threads to pass in distinct values.
This approach is sound because the implementation is \emph{data independent}:
each data value is passed in as an input (i.e.,~a parameter of |exchange|),
passed around, stored, and output; but no operation is performed on it that
depends on its actual value.  This means that if there were an incorrect
behaviour that didn't satisfy our requirements, there would also be an
incorrect behaviour where all the inputs were replaced by distinct values.

A function to perform a single test is below; this can be executed many times.
Each test runs |n| threads with an exchanger, where |n| is a random even
number.  In fact, for convenience, we arrange that each thread submits its
identity to the exchanger.  It then stores the value it receives back in the
array |results|, indexed by its identity.  The correctness condition then is
that for each~|i|, if thread~|i| received~|x|, then thread~|x| received~|i|,
i.e.~those two threads exchanged with each other; this is easily checked.
%
\begin{scala}
  def doTest = {
    val n = 2*scala.util.Random.nextInt(10); val results = new Array[Int](n)
    val exchanger = new Exchanger[Int]
    def worker(me: Int) = thread(s"worker($me)"){ 
      val x = exchanger.exchange(me); results(me) = x 
    }
    run(|| (for(i <- 0 until n) yield worker(i)))
    for(i <- 0 until n){ val x = results(i);  assert(results(x) == i) }
    exchanger.shutdown()
  }
\end{scala}
%

One detail concerning this testing strategy is that each thread performed just
a single exchange.  It is tempting to arrange for a worker to perform multiple
exchanges, perhaps submitting a different value each time.  However, this would
lead to the possibility of the testing system deadlocking: it could reach a
state where one thread has two exchanges still to make, but all the other
threads have terminated.  When each thread performs just a single exchange, we
avoid this possibility of deadlock. 

\begin{instruction}
Study the details of the |doTest| function.
\end{instruction}

%%%%%%%%%%%%%%%%%%%%%%%%%%%%%%%%%%%%%%%%%%%%%%%%%%%%%%%

\section{Example: A Filter Channel}

We now consider another example of a synchronisation object.  A \emph{filter
  channel} acts much like a standard synchronous channel, except a receiver
can specify a property that it requires the value it receives to satisfy.
Thus it will implement the following trait.
%
\begin{scala}
trait FilterChanT[A]{
  def send(x: A): Unit
  def receive(p: A => Boolean): A 
}
\end{scala}
%
Executions of |send(x)| and |receive(p)| can synchronise with each other
if and only if |p(x)|; in this case, |receive(p)| returns~|x|.  Each operation
blocks until it is able to synchronise with a suitable operation of the other
kind. 

%%%%%

\begin{figure}
\begin{scala}
class FilterChan[A] extends FilterChanT[A]{
  /** Channel the sender uses to send to the server. */
  private val fromSender = new SyncChan[(A, Chan[Unit])]

  /** Synchronously send x. */
  def send(x: A) = { 
    val replyChan = new SyncChan[Unit]; fromSender!(x, replyChan); replyChan?()
  }

  /** Channel receivers use to send to the server. */
  private val fromReceiver = new SyncChan[(A => Boolean, Chan[A])]
  
  /** Synchronously receive a value that satisfied p. */
  def receive(p: A => Boolean): A = {
    val replyChan = new SyncChan[A]; fromReceiver!(p, replyChan); replyChan?()
  }

  private val shutdownChan = new SyncChan[Unit]

  /** Shut down the server. */
  def shutdown() = shutdownChan!() 

  /** The server process. */
  private def server = thread("ServerFilterChan"){ ... } // See Figure £\ref{fig:FilterChan-server}£.

  fork(server)
}
\end{scala}
\caption{Most of the {\scalashape FilterChan} class.}
\label{fig:FilterChan}
\end{figure}

%%%%%%%%%%

Most of the code for the |FilterChan| class is in
Figure~\ref{fig:FilterChan}.  The |send(x)| and |receive(p)| send,
respectively, |x| and~|p|, together with a reply channel to the server, and
then wait to receive on the reply channel.  As normal, the |shutdown|
operation sends a simple signal to the server. 

\pagebreak[2]

The server is in Figure~\ref{fig:FilterChan-server}.  This maintains queues
|pendingSends| and |pendingRecs| of pending requests from senders and
receivers.  An invariant is that no entries in these queues are compatible:
for each |(x,cs)| in |pendingSends| and for each |(p,cr)| in |pendingRecs|,\,
|p(x)| does not hold.  However, the code does not necessarily maintain the
order in the queues: a queue is simply a convenient collection class to use
here, allowing easy iteration. 

%%%%%

\begin{figure}
\begin{scala}
  private def server = thread("ServerFilterChan"){
    // Queue of pending requests from senders.
    val pendingSends = new Queue[(A, Chan[Unit])]
    // Queue of pending requests from receivers.
    val pendingRecs = new Queue[(A => Boolean, Chan[A])]
    var isShutdown = false
    serve(!isShutdown)(
      fromSender =?=> { case (x,cs) => 
        // Traverse £pendingRecs£ to see if a request matches.
        var i = 0; var done = false; val len = pendingRecs.length
        while(i < len && !done){
          val (p,cr) = pendingRecs.dequeue()
          if (p(x)){ cr!x; cs!(); done = true }
          else{ pendingRecs.enqueue((p, cr)); i += 1 }
        }
        if(!done) pendingSends.enqueue((x,cs))
      }
      | fromReceiver =?=> { case (p,cr) =>
        // Traverse £pendingSends£ to see if a value matches.
        var i = 0; var done = false; val len = pendingSends.length
        while(i < len && !done){
          val (x,cs) = pendingSends.dequeue()
          if(p(x)){ cr!x; cs!(); done = true }
          else{ pendingSends.enqueue((x, cs)); i += 1 }
        }
        if(!done) pendingRecs.enqueue((p,cr))
      }
      | shutdownChan =?=> { _ => isShutdown = true }
    )
    fromSender.close(); fromReceiver.close()
    for((_,c) <- pendingSends) c.endOfStream()
    for((_,c) <- pendingRecs) c.endOfStream()
  }
\end{scala}
\caption{The server for the {\scalashape FilterChan} class.}
\label{fig:FilterChan-server}
\end{figure}

When the server receives a pair |(x,cs)| from a sender, it traverses
|pendingRecs| to try to find a pair |(p,cr)| such that |p(x)|.  If is finds a
match, it sends |x| to the receiver on~|cr|, and signals to the sender
on~|cs|.  If it doesn't find a match, it adds |(x,cs)| to |pendingSends|.

The server acts very similarly when it receives a message from a receiver
(there is an argument for factoring out the common code into a separate
function).  When it receives a message to shut down, it sets the |isShutdown|
flag, exits the loops, and closes all the channels, including those in its
queues, to signal to any waiting clients.

\begin{instruction}
Make sure you understand the implementation of |FilterChan|.
\end{instruction}

%%%%%

We now consider how to test the implementation.  In particular, we need to
ensure that the filter channel is \emph{synchronous}, in particular that the
sender does not return before the corresponding receiver calls the operation:
in the implementation, the use of a reply channel was designed to avoid this.
We will therefore use the technique of logging, as we did with the resource
server.  We will run a number of senders and receivers that call the
operations, but that write into the log both before and after the operation.
This will allow us to identify which operation executions overlap, and then we
can try to pair up corresponding send and receives.  
%
This pairing up of corresponding operation executions will be easier if we
arrange for each |send| operation to send a different value.  

Most of the testing code is in Figure~\ref{fig:FilterChan-test1}.  (It is
trivial to adapt it to test other implementations of the |FilterChanT| trait.)
We log using events of type |LogEvent|, with subclasses corresponding to the
begin and end of |send|s and |receive|s; the events include the identity of
the thread in question (to allow pairing of corresponding |Begin| and |End|
events), and, in some cases, the value sent or received (to allow matching a
|send| to the corresponding |receive|.

%%%%%

\begin{figure}
\begin{scala}
  trait LogEvent // Events to include in the log. 
  case class BeginSend(id: Int, x: Int) extends LogEvent
  case class EndSend(id: Int) extends LogEvent
  case class BeginReceive(id: Int) extends LogEvent
  case class EndReceive(id: Int, x: Int) extends LogEvent
  type Log1 = Log[LogEvent]

  def sender(me: Int, chan: FilterChanT[Int], log: Log1) 
  = thread(s"sender($me)"){
    for(x <- me*iters until (me+1)*iters){
      log.add(me, BeginSend(me, x)); chan.send(x); log.add(me, EndSend(me))
    }
  }

  def receiver(me: Int, chan: FilterChanT[Int], log: Log1) 
  = thread(s"receiver($me)"){
    def p(x: Int) = x%n == me
    val logId = me+n // Identity for logging purposes.
    for(x <- 0 until iters){
      log.add(logId, BeginReceive(me)); val x = chan.receive(p)
      log.add(logId, EndReceive(me, x))
    }
  }

  def checkLog(events: Array[LogEvent]): Boolean = ... // See Figure £\ref{fig:FilterChan-test2}£.

  def doTest = {
    val chan: FilterChanT[Int] = new FilterChan[Int]; val log = new Log1(2*n)
    val senders = || (for(i <- 0 until n) yield(sender(i, chan, log)))
    val receivers = || (for(i <- 0 until n) yield (receiver(i, chan, log)))
    run(senders || receivers)
    assert(checkLog(log.get))
    chan.shutdown()
  }
\end{scala}
\caption{Most of the code for testing the {\scalashape FilterChan}.}
\label{fig:FilterChan-test1}
\end{figure}

%%%%%

We run |n| senders and |n| receivers.  The sender with identity |me| (with
$\sm{me} \in \interval{0}{\sm n}$) sends the values $\interval{\sm{me} \times
  \sm{iters}}{(\sm{me}+1) \times \sm{iters}}$, so all sent values are
distinct.  The receiver with identity~|me| (again with $\sm{me} \in
\interval{0}{\sm n}$) is willing to receive values~|x| such that
$\sm{x}\%\sm{n} = \sm{me}$, so the receivers collectively are willing to
receive all the values.  

Each thread writes suitable events into the log before calling the operation,
and after it returns.  Recall that each thread includes its identity in
operations to add events to the log.  In order to ensure that the senders and
receivers use different identities for this purpose, we add |n| to the
identity of each receiver.

The log slightly overestimates the time period during which a thread is
actually in the operation, because there is a slight delay between logging the
|Begin| event and calling the operation, and a slight delay between the
operation returning and logging the |End| event.  However, this discrepancy is
not a problem as it will not lead to false positives.  If the implementation
is faulty, this might lead to false negatives; but this is unlikely, and other
runs are likely to find the error.

The |doTest| operation performs a single test.  It creates the |FilterChan|
and log, and runs the senders and receivers.  It then uses the |checkLog|
function, given in Figure~\ref{fig:FilterChan-test2} and described below, to
test whether the log represents a valid execution, where senders and receivers
correctly synchronise and exchange values.


%%%%%

\begin{figure}
\begin{scala}
  def checkLog(events: Array[LogEvent]): Boolean = {
    def giveError(i: Int) = 
      println(s"\nUnmatched End event at index $i: "+events(i)+"\n"+
        events.mkString("\n"))
    val Out = 0; val Unmatched = 1; val Matched = 2
    val senderState = Array.fill(n)(Out); val receiverState = Array.fill(n)(Out)
    val senderValue = new Array[Int](n); val receiverValue = new Array[Int](n)
    for(i <- 0 until events.length) events(i) match{
      case BeginSend(s, x) => 
        assert(senderState(s) == Out)
        senderState(s) = Unmatched; senderValue(s) = x
      case BeginReceive(r) =>
        assert(receiverState(r) == Out); receiverState(r) = Unmatched
        var j = i+1; var done = false
        while(!done) events(j) match{
          case EndReceive(r1, x) if r == r1 => receiverValue(r) = x; done = true
          case _ => j += 1
        }
      case EndSend(s) =>
        if(senderState(s) == Matched) senderState(s) = Out
        else{
          assert(senderState(s)==Unmatched); var r = 0; val x = senderValue(s)
          while(r < n && senderState(s) == Unmatched)
            if(receiverState(r) == Unmatched && receiverValue(r) == x){
              senderState(s) = Out; receiverState(r) = Matched }
            else r += 1 // End of £while£.
          if(senderState(s) == Unmatched){ giveError(i); return false }
        }
      case EndReceive(r, x) =>
        if(receiverState(r) == Matched) receiverState(r) = Out
        else{
          assert(receiverState(r) == Unmatched); var s = 0
          while(s < n && receiverState(r) == Unmatched)
            if(senderState(s) == Unmatched && senderValue(s) == x){
              senderState(s) = Matched; receiverState(r) = Out }
            else s += 1 // End of £while£.
          if(receiverState(r) == Unmatched){ giveError(i); return false }
        }
    } // End of £for£/£match£.
    true
  }
\end{scala}
\caption{The {\scalashape checkLog} function for testing a {\scalashape
    FilterChan}.} 
\label{fig:FilterChan-test2}
\end{figure}

The |checkLog| function traverses the log, keeping track, in arrays
|senderState| and |receiverState|, whether each thread is currently outside an
operation call (value~|Out|), inside an operation call but has not yet been
matched with another operation (value~|Unmatched|), or inside an operation
call and matched with another operation call (value~|Matched|).  It also keeps
track, in arrays |senderValue| and |receiverValue| of the value being sent by
a sender, or the value that a |receive| operation will return.

When the traversal encounters a |Begin| operation, it updates values
appropriately.  In the case of a |BeginReceive|, it searches forward through
the log to find the corresponding |EndReceive| event, to record the value the
operation will return.

When the traversal encounters an |EndSend| event, if this operation has not
already been matched, it searches through the receivers to try to find one
that is also unmatched and that receives the corresponding value.  If so, it
records that |receive| operation as matched, and the |send| operation as
complete.  If it fails to find a match, this represents an error: it prints
appropriate information to help with debugging, and returns false.  When the
traversal encounters an |EndReceive| event, it acts in a very similar way.

\begin{instruction}
Make sure you understand how the testing program works, and, in particular,
the |checkLog| function.
\end{instruction}


%% The easiest way seems to be
%% to set up a testing system where it is possible to predict what values each
%% receiver should receive, and then to test whether those values are indeed
%% received.  To this end, the |doTest| function in
%% Figure~\ref{fig:FilterChan-testing} runs |n| senders and |n|~receivers.  The
%% sender with identity~|me| sends the values |me|, |me+n|, |me+2*n|, \ldots,
%% |me+(iters-1)*n|.  The receiver with identity~|me| uses a predicate to ensure
%% that it receives values~|x| such that \SCALA{x\%n == me}; thus we expect it to
%% receive the values |me|, |me+n|, |me+2*n|, \ldots, |me+(iters-1)*n|, in that
%% order; the thread throws an exception if this is not the case. 

%% \begin{figure}
%% \begin{scala}
%%   def sender(me: Int, chan: FilterChan[Int]) = thread(s"sender($me)"){
%%     for(i <- 0 until iters) chan.send(me+i*n)
%%   }

%%   def receiver(me: Int, chan: FilterChan[Int]) = thread(s"receiver($me)"){
%%     def p(x: Int) = x%n == me
%%     for(i <- 0 until iters){
%%       val x = chan.receive(p); assert(p(x)); assert(x == me+i*n)
%%     }
%%   }

%%   def doTest = {
%%     val chan = new FilterChan[Int]
%%     val senders = || (for(i <- 0 until n) yield(sender(i, chan)))
%%     val receivers = || (for(i <- 0 until n) yield (receiver(i, chan)))
%%     run(senders || receivers)
%%     chan.shutdown()
%%   }
%% \end{scala}
%% \caption{Testing the {\scalashape FilterChan} implementation.}
%% \label{fig:FilterChan-testing}
%% \end{figure}

%% \begin{instruction}
%% Make sure you understand this testing harness.
%% \end{instruction}
 % exchanger, filter channel

%\input{client-server4} % layering; conclusions

%%%%%

\section{Summary}

In this chapter we have studied the pattern of clients and servers: clients
make requests to a server, and the server responds.  

Typically, a server will allow different sorts of request.  These can be
supported by using a different channel for each sort, and using an alternation
in the server.  Alternatively, we could use a single channel, and use
different subtypes of the channel's data type to represent the different sorts
of request. 

Responses to requests (where needed) can be implemented in a couple of
different ways.  We could use a separate channel for each client.
Alternatively, we can arrange for the client to create a short-term reply
channel which it sends to the server, and for the server to respond on this
reply channel.  The latter approach is more flexible, so it is what we tend to
use. 

Sometimes a server is not immediately able to respond to a request; this is
the case in the resource allocation example, when no resource is available.
The normal approach in these circumstances is to store the request in a queue.
However, sometimes it is simpler for the server to refuse to communicate on
the relevant channel, so clients are blocked on that communication.  This
latter approach can also be useful in synchronisation objects, by restricting
communications with different sorts of clients to happen in a fixed order. 

Testing is an important consideration for concurrent objects.  We saw some
techniques in this chapter; and the same techniques are applicable regardless
of how the concurrent objects themselves are implemented.

Logging is often necessary for testing concurrent objects.  This is
particularly the case when the object maintains some state: we need to test
whether the results of operation executions are consistent with that state.
We will use this technique further when we consider concurrent datatypes in
the next chapter.

Logging is also often necessary when testing synchronisation objects.  We need
to identify which operation executions synchronise together, and for this we
need to know which overlap in time.  (The exchanger example was unusual in
this respect as we could identify which operations synchronised from their
parameters and results.)
 


%%%%%%%%%%%%%%%%%%%%%%%%%%%%%%%%%%%%%%%%%%%%%%%%%%%%%%%

\section*{Exercises}

\begin{question}
Consider the following synchronisation problem.  There are two
types of client threads, which we shall call \emph{men} and \emph{women}.
These threads need to pair off for some purpose, with each pair containing
one thread of each type.
%
Design a server  to support this.  Each client should send its name to
the server, and receive back the name of its partner.  Encapsulate the server
within a class, with operations
\begin{scala}
  def manSync(me: String): String = ...
  def womanSync(me: String): String = ...
\end{scala}

Implement a test rig for your implementation: think carefully about the
correctness condition. 
\end{question}

%%%%%

\begin{answerI}
My solution is below.  The server has two request channels, one for each type
of client.  Each client sends its name and a reply channel.  The server waits
for a request from a man, and a request from a woman, and pairs them off.
%
\begin{scala}
class MenWomenServer{
  private type ReplyChan = Chan[String]

  /** Channels sending proposals from a man, resp., a woman. */
  private val manProp, womanProp = new SyncChan[(String, ReplyChan)]

  /** A man tries to find a partner. */
  def manSync(me: String): String = {
    val c = new OnePlaceBuffChan[String]; manProp!(me,c); c?()
  }

  /** A woman tries to find a partner. */
  def womanSync(me: String): String = {
    val c = new OnePlaceBuffChan[String]; womanProp!(me,c); c?()
  }

  /** The server. */
  private def server = thread{
    repeat{
      // Wait for a man and woman, and pair them off. 
      val (him,hisC) = manProp?(); val (her,herC) = womanProp?()
      hisC!her; herC!him
    }
    manProp.close(); womanProp.close()
  }

  fork(server)

  /** Shut down this object (so the server thread terminates). */
  def shutdown() = { manProp.close(); womanProp.close() }
}
\end{scala}


There are alternative solutions that involve storing requests that cannot be
paired immediately.  However, such approaches are more complex.  It is often
best to avoid communicating on a channel if you know the corresponding request
can't be served.


Most of my testing code is below.  We need to check that men and women receive
compatible results: if man~$m$ thinks he is paired with woman~$w$, then $w$
thinks she is paired with~$m$.  This implies the vice-versa direction
(assuming equal numbers of men and women).  It also implies that the two
operations synchronise, i.e.~overlap in time, as for the exchanger example.

Each man stores his result in |partnerOfMan|, and each woman stores her result
in |partnerOfWoman|; we then check the results agree.  (Each man and woman has
an integer identity; we take their name to be the corresponding string; we
cast back to an |Int| to store in the arrays.)
%
\begin{scala}
  /** Arrays that hold the identity of each man/woman's partner. */
  var partnerOfMan, partnerOfWoman: Array[Int] = null

  /** Thread for a man. */
  def man(me: Int, mw: MenWomen) = thread{
    partnerOfMan(me) = mw.manSync(me.toString).toInt
  }

  /** Thread for a woman. */
  def woman(me: Int, mw: MenWomen) = thread{
    partnerOfWoman(me) = mw.womanSync(me.toString).toInt
  }

  /** Do a single test. */
  def doTest = {
    val n = scala.util.Random.nextInt(10) // Number of men, women.
    partnerOfMan = new Array[Int](n); partnerOfWoman = new Array[Int](n)
    val mw = new MenWomenServer 
    val men = || (for(i <- 0 until n) yield man(i, mw))
    val women = || (for(i <- 0 until n) yield woman(i, mw))
    run(men || women)
    mw.shutdown
    for(m <- 0 until n)
      assert(partnerOfWoman(partnerOfMan(m)) == m,
             partnerOfMan.mkString(", ")+"\n"+
               partnerOfWoman.mkString(", ")+"\n"+m)
  }
\end{scala}

Alternatively, the implementation can be tested by adapting the
graph-theoretic approach described at the end of
Section~\ref{sec:filter-chan}. 
\end{answerI}
% 

%\begin{scala}
% // The men and women problem, using a server process

% import ox.CSO._

% class MenWomen{
%   type Name = String; // Names of men and women
%   private type ReplyChan = OneOne[Name]; // channels for replies

%   // Each request contains the name of the requester and a reply channel
%   private type Req = Tuple2[Name, ReplyChan];
%   private val manReq, womanReq = ManyOne[Req];

%   // Men and women send a request to the server on a fresh channel, and
%   // receive back a reply
%   def ManSync(me: Name) : Name = {
%     val c = OneOne[Name]; manReq ! (me, c);
%     val her = c?; return her;
%   }

%   def WomanSync(me: Name) : Name = {
%     val c = OneOne[Name]; womanReq ! (me, c);
%     val him = c?; return him;
%   }

%   private def Server = proc{
%     // Queues storing the men and women who have not yet been paired; at most
%     // one should be non-empty
%     val manQueue = new scala.collection.mutable.Queue[Req];
%     val womanQueue = new scala.collection.mutable.Queue[Req];
%     // The server stores unmatched requests in queues.  Each new request is
%     // matched immediately, if possible; otherwise it's places in the relevant
%     // queue.

%     serve(
%       manReq --> {
% 	val (man, manc) = manReq?; 
% 	if(womanQueue.isEmpty) manQueue += (man, manc);
% 	else{
% 	  val (woman, womanc) = womanQueue.dequeue;
% 	  manc ! woman; womanc ! man;
% 	} 
%       }
%       |
%       womanReq --> {
% 	val (woman, womanc) = womanReq?; 
% 	if(manQueue.isEmpty) womanQueue += (woman, womanc);
% 	else{
% 	  val (man, manc) = manQueue.dequeue;
% 	  manc ! woman; womanc ! man;
% 	} 
%       }
%     )
%   }

%   // Public method to close down the server
%   def Close = { manReq.close; womanReq.close; }

%   // Fork off server process
%   Server.fork;
% }

% // -------------------------------------------
% // Class to test the module

% object MenWomenTest{
%   val N = 10;
%   val random = new scala.util.Random;

%   // Create the server
%   val menWomenServer = new MenWomen;

%   type Name = menWomenServer.Name;

%   // Men and women call the appropriate methods, after a short delay
%   def Man(me: Name) = proc{
%     sleep(random.nextInt(2)); val her = menWomenServer.ManSync(me);
%     println("Man "+me+" pairs with woman "+her);
%   }

%   def Woman(me: Name) = proc{
%     sleep(random.nextInt(2)); val him = menWomenServer.WomanSync(me);
%     println("Woman "+me+" pairs with man "+him);
%   }

%   // Put the system together
%   def Men = || ( for (i <- 0 until N) yield Man("man"+i) );
%   def Women = || ( for (i <- 0 until N) yield Woman("woman"+i) );
  
%   def System = Men || Women;

%   def main(args : Array[String]) = { System() ; menWomenServer.Close; }
% }
% \end{scala}
% \end{answer}


%% \begin{scala}
%% // The men and women problem, using a server process

%% import ox.CSO._

%% object MenWomen{

%%   val N = 10;
%%   val random = new scala.util.Random;

%%   type Name = String; // Names of men and women
%%   type ReplyChan = OneOne[Name]; // channels for replies

%%   // Each request contains the name of the requester and a reply channel
%%   type Req = Tuple2[Name, ReplyChan];
%%   val manReq, womanReq = ManyOne[Req];

%%   // Men and women send a request and receive back a reply
%%   def Man(me: Name) = proc{
%%     val c = OneOne[Name];
%%     sleep(random.nextInt(2));
%%     manReq ! (me, c);
%%     val her = c?;
%%     println("Man "+me+" pairs with woman "+her);
%%   }

%%   def Woman(me: Name) = proc{
%%     val c = OneOne[Name];
%%     sleep(random.nextInt(2));
%%     womanReq ! (me, c);
%%     val him = c?;
%%     println("Woman "+me+" pairs with man "+him);
%%   }

%%   def Server = proc{
%%     // Queues storing the men and women who have not yet been paired; at most
%%     // one should be non-empty
%%     val manQueue = new scala.collection.mutable.Queue[Req];
%%     val womanQueue = new scala.collection.mutable.Queue[Req];
%%     // The server stores unmatched requests in queues.  Each new request is
%%     // matched immediately, if possible; otherwise it's placed in the relevant
%%     // queue.

%%     serve(
%%       manReq --> {
%% 	val (man, manc) = manReq?; 
%% 	if(womanQueue.isEmpty) manQueue += (man, manc);
%% 	else{
%% 	  val (woman, womanc) = womanQueue.dequeue;
%% 	  manc ! woman; womanc ! man;
%% 	} 
%%       }
%%       |
%%       womanReq --> {
%% 	val (woman, womanc) = womanReq?; 
%% 	if(manQueue.isEmpty) womanQueue += (woman, womanc);
%% 	else{
%% 	  val (man, manc) = manQueue.dequeue;
%% 	  manc ! woman; womanc ! man;
%% 	} 
%%       }
%%     )
%%   }

%%   // Put the system together
%%   def Men = || ( for (i <- 0 until N) yield Man("man"+i) );
%%   def Women = || ( for (i <- 0 until N) yield Woman("woman"+i) );
  
%%   def System = Men || Women || Server;

%%   def main(args : Array[String]) = System() 
%% }
%% \end{scala}
%\end{answer}

      


\begin{questionS}
A mechanism is required to allow a single sender to broadcast a message to~|n|
receivers.  Each broadcast should represent a synchronisation of all $\sm n +
1$ threads, after each has called its operation, but before any has returned. 
Thus we require a class with the following signature.
%
\begin{scala}
class AtomicBroadcast[A](n: Int){
  /** Synchronously send £x£ to the £n£ receivers. */
  def send(x: A): Unit

  /** Synchronously receive the sender's value. */
  def receive(): A 
}
\end{scala}
%
Implement such a class, encapsulating a server.

Produce a testing harness for your code.
\end{questionS}

% =======================================================

\begin{answerS}
My code is below.  In order to implement the synchronisation, each client
sends a reply channel to the server.  One the server has received from all
$\sm n+1$ clients, the synchronisation can occur, and the server can reply to
all the clients.
%
\begin{scala}
class AtomicBroadcast[A](n: Int){
  private val fromSender = new SyncChan[(A, Chan[Unit])]

  private val fromReceiver = new SyncChan[Chan[A]]

  def send(x: A): Unit = {
    val c = new OnePlaceBuffChan[Unit]; fromSender!(x,c); c?()
  }

  def receive(): A = {
    val c = new OnePlaceBuffChan[A]; fromReceiver!c; c?()
  }

  private def server = thread{
    val queue = new scala.collection.mutable.Queue[Chan[A]]
    repeat{
      val (x,cs) = fromSender?()
      for(_ <- 0 until n) queue.enqueue(fromReceiver?())
      // All channels received.
      cs!(); for(_ <- 0 until n) queue.dequeue()!x
    }
  }

  fork(server)

  def shutdown() = { fromSender.close(); fromReceiver.close() }
}
\end{scala}

To test the atomic broadcast, we arrange for clients to write into the log
before and after each operation.  

\begin{scala}
  trait LogEvent
  case class BeginSend(x: Int) extends LogEvent
  case object EndSend extends LogEvent
  case object BeginReceive extends LogEvent
  case class EndReceive(x: Int) extends LogEvent
  type Log1 = Log[LogEvent] 

  val iters = 10 // Number of broadcasts per run.
  val Max = 100 // Range of values sent.

  def sender(me: Int, ab: AtomicBroadcast[Int], log: Log1) = thread{
    for(i <- 0 until iters){
      val x = Random.nextInt(Max); log.add(me, BeginSend(x))
      ab.send(x); log.add(me, EndSend)
    }
  }

  def receiver(me: Int, ab: AtomicBroadcast[Int], log: Log1) = thread{
    for(i <- 0 until iters){
      log.add(me, BeginReceive); val x = ab.receive(); log.add(me, EndReceive(x))
    }
  }
\end{scala}

The main testing code is blow.  For a correct implementation, it should be
possible to identify the place in the log where each synchronisation takes
place: after all $\sm n+1$ clients have called their operation, but before any
returns.  To this end, we traverse the log (in function |checkLog|), keeping
track of the following information:
%
\begin{itemize}
\item |currentSend| records information about the value currently being sent,
  if any: it holds |Some(x)| from a call of |send(x)| up until the
  synchronisation, and subsequently holds |None|;

\item |receiversCalled| records the number of receivers that have called their
  operation but not yet synchronised;

\item |receiversToReturn| is the number of receivers that have synchronised
  but not yet returned;

\item |currentVal| is the value for the most recent synchronisation, which
  receivers are expected to return.
\end{itemize}
%
If |currentSend = Some(x)| and |receiversCalled = n|, the synchronisation can
happen.  
%
\begin{scala}
  def checkLog(n: Int, events: Array[LogEvent]) = {
    def mkError(i: Int) = 
      s"Error at index $i: ${events(i)}\n"+events.mkString("\n")
    var currentSend: Option[Int] = None; var currentVal = -1
    var receiversCalled = 0; var receiversToReturn = 0
    for(i <- 0 until events.length){
      events(i) match{
        case BeginSend(x) => 
          assert(currentSend == None, mkError(i)); currentSend = Some(x)
        case EndSend => assert(currentSend == None, mkError(i))
        case BeginReceive => receiversCalled += 1
        case EndReceive(x) => 
          assert(receiversToReturn > 0, mkError(i))
          assert(x == currentVal, mkError(i)); receiversToReturn -= 1
      }
      if(currentSend != None && receiversCalled == n){
        // The threads can now synchronise.
        currentVal = currentSend.get; currentSend = None
        receiversCalled = 0; receiversToReturn = n
      }
    }
  }

  def doTest = {
    val n = Random.nextInt(10); val ab = new AtomicBroadcast[Int](n)
    val log = new Log1(n+1)
    val receivers = || (for(i <- 0 until n) yield receiver(i, ab, log))
    run(sender(n, ab, log) || receivers)
    checkLog(n, log.get)
    ab.shutdown()
  }
\end{scala}
\end{answerS}


Possible \framebox{extra} exercises:
One family problem (2023 exam);
Same-arg synchronisation (2022 exam);
ABC problem (2021 exam).


