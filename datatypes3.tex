\section{Testing for Linearizability}

We now consider how we can test whether the implementation of a total
concurrent queue from Section~\ref{sec:total-queue} is linearizable, and more
generally, how we can test other implementations of concurrent datatypes for
linearizability.

The basic idea is as follows. We run some threads on the concurrent datatype,
performing random operations, and record the history of operation calls and
returns.  We then search for a corresponding sequential history that explains
linearizability, and that is a valid history for the corresponding sequential
datatype.  If none is found, we signal an error.  We can repeat this procedure
many times.

Algorithms for searching for the corresponding sequential history have been
investigated by Wing and Gong~\cite{wing-gong}, and by
me~\cite{gavin:lin-testing}.  The SCL library includes the latter, which can
be used without understanding the underlying algorithms. 

Figure~\ref{fig:queue-lin-tester} gives a stripped-down testing script for a
concurrent total queue.  (The full version, on the book website, can be used
with multiple concurrent queue implementations, and replaces the numerical
constants by variables, specifiable via the command line.)

%%%%%

\begin{figure}
\begin{scala}[numbers = left]
import scala.collection.immutable.Queue

object QueueTest{
  /* The concurrent datatype, and sequential specification type. */
  type ConcQueue = TotalQueue[Int]; type SeqQueue = Queue[Int]

  /* Sequential operations. */
  def seqEnqueue(x: Int)(q: SeqQueue): (Unit, SeqQueue) = ((), q.enqueue(x))£\label{line:seqEnqueue}£
  def seqDequeue(q: SeqQueue): (Option[Int], SeqQueue) =   
    if(q.isEmpty) (None, q) 
    else{ val (r,q1) = q.dequeue(); (Some(r), q1) }

  /** A worker in the testing system. */
  def worker(me: Int, log: LinearizabilityLog[SeqQueue, ConcQueue]) = {£\label{line:queue-lin-tester-worker}£
    val random = new scala.util.Random
    for(i <- 0 until 200)
      if(random.nextFloat() <= 0.3){
        val x = random.nextInt(20)
        log(_.enqueue(x), s"enqueue($x)", seqEnqueue(x))£\label{line:queue-lin-tester-worker-enqueue}£
      }
      else log(_.dequeue(), "dequeue", seqDequeue)£\label{line:queue-lin-tester-worker-dequeue}£
  }

  /** Perform a single test. */
  def doTest = {
    val concQueue = new ServerTotalQueue[Int]; val seqQueue = Queue[Int]()
    val tester = 
      LinearizabilityTester[SeqQueue,ConcQueue](seqQueue, concQueue, 4, worker)£\label{line:construct-lin-tester}£
    if(tester() <= 0) sys.exit()
    concQueue.shutdown()
  }

  def main(args: Array[String]) = {
    for(r <- 0 until 10000){ doTest; if(r%20 == 0) print(".") }
    println()
  }
}
\end{scala}
\caption{A linearizability tester for a concurrent total queue.}
\label{fig:queue-lin-tester}
\end{figure}

%%%%%

The test program works on a |TotalQueue[Int]|; we define |ConcQueue| as a type
synonym.

The test program also requires a corresponding sequential specification
object.  The underlying linearizability framework requires this to be
\emph{immutable} (unlike in Section~\ref{sec:linearization}): the state of the
specification object should not change; instead, operations produce new
instances of the same type.  The framework also requires the specification
object to be deterministic.  Here, we use an immutable queue
from the Scala API; we define |SeqQueue| as a type synonym. 

In general, given a type |S| of the sequential specification object, for each
operation |op: A| on the concurrent datatype, we need a corresponding function
|seqOp: S => (A, S)| on the sequential datatype, which returns the expected
returned value\footnote{More precisely, the two values should be equal, as
  tested using the ``{\scalashape ==}'' method.} (of type~|A|), paired with
the new value of the sequential datatype.

For the case of a concurrent total queue, the relevant definitions are from
line~\ref{line:seqEnqueue}.  These are simple wrappers around the API
operations.  The operation |seqEnqueue(x)| corresponds to |enqueue(x)| on the
concurrent queue.  Given a sequential queue~|q|, this returns the unit value
(matching the return from~|enqueue|) and a new sequential queue, formed by
enqueuing~|x| on~|q|.  The operation |seqDequeue| corresponds to |dequeue()|
on the concurrent queue.  When given an empty queue~|q|, it returns |None| (to
match the value expected from |dequeue()|), and the original state of~|q|.
When given a nonempty queue, it performs a dequeue using the API
operation, which returns the value~|r| dequeued and the new state of the
queue, and then constructs the appropriate result (including |Some(r)|, to
match the value expected from |dequeue()|).

%%%%%

The function |worker| (from line~\ref{line:queue-lin-tester-worker}) defines
a worker that performs and logs operations on the concurrent datatype,
associating each concurrent operation with a corresponding operation on the
sequential datatype.  Here, the worker performs 200 operations; each is (with
probability 0.3) an enqueue of a random value, or (with probability~0.7) a
dequeue.


The worker takes a log object |log| as a parameter; each operation is
performed and logged via a call to |log| (more precisely, the |apply| function
of~|log|; see Scala box~\ref{sb:apply}), taking three parameters:
%
\begin{itemize}
\item
The operation to be performed on the concurrent datatype;

\item A string describing the operation: this is used in debugging output in
  the case that a non-linearizable history is found, and is also used
  internally for optimisations; semantically different operations should have
  distinct strings;

\item
The corresponding function on the sequential datatype.
\end{itemize}
%
The call to |log| logs the invocation of the concurrent operation,
performs the concurrent operation, and logs the result returned.

In Figure~\ref{fig:queue-lin-tester},
line~\ref{line:queue-lin-tester-worker-enqueue} performs |enqueue(x)| on the
concurrent queue, and associates it with the function |seqEnqueue(x)| on the
specification object.  Line~\ref{line:queue-lin-tester-worker-dequeue}
performs |dequeue()| on the concurrent queue, and associates it with the
function |seqDequeue| on the specification object.  Each uses an anonymous
function (see Scala box~\ref{sb:anon-function}) to define the operation on the
concurrent queue: each underscore (``\SCALA{_}'') represents a hole into which
the concurrent queue is placed.

The function |doTest| performs a single test.  This starts by creating a
concurrent queue to test, and a corresponding sequential specification object.
Line~\ref{line:construct-lin-tester} constructs a linearizability testing
object~|tester|.  This takes two type parameters, representing the type of
sequential specification objects and concurrent objects (we could omit these
type parameters, but I find it useful to include them).  It takes value
parameters representing the specification object, the concurrent object to be
tested, the number of worker threads to run (here~4), and the definition of a
worker.
%
The expression |tester()| then runs the testing object (by calling its |apply|
function).  This runs the workers concurrently, logging the operation calls on
the concurrent datatype.  It then tests whether the resulting history is
linearizable, returning a positive result if so, and otherwise printing
suitable debugging information on the screen.  The |doTest| function
terminates the program if the history is not linearizable.  Finally, it shuts
down the concurrent queue.

The |main| function calls |doTest| many times. 

Figure~\ref{fig:lin-testing-error} gives sample debugging information for a
(deliberately) incorrect implementation of a |TotalQueue|.  More precisely,
the implementation uses buffered channels, rather than synchronous channels;
we saw earlier why this is incorrect.  The test was run with just three worker
threads, each performing a single operation, so as to keep the history short.
The error is reflected by the value of |None| returned (at line~5) by the
|dequeue| done by thread~|2|.  Each of threads~|0| and~|1| had completed an
|enqueue| by this point, so the |dequeue| should have returned one of their
values; in fact, the |enqueue|s were concurrent, so could have been linearized
in either order, and so either of their values would have been allowed.

%%%%%  

\begin{figure}
\begin{verbatim}
0: 0 invokes enqueue(69)
1: 1 invokes enqueue(60)
2: 0 returns ()
3: 1 returns ()
4: 2 invokes dequeue
5: 2 returns None
-- Previous event not linearized
-- Allowed return values: Some(69), Some(60)
\end{verbatim}
\caption{Output of {\scalashape QueueTest} on a non-linearizable history.}
\label{fig:lin-testing-error}
\end{figure}

%%%%%

Linearizability testers for other concurrent datatypes take a very similar
form, as you will see if you complete the exercises at the end of this
chapter.  My normal approach to writing such a tester is to take an existing
one and to adapt it.

Under the bonnet, the linearizability tester performs a search, in effect,
considers all possible linearizations of the history---i.e.~all ways of
ordering the operations consistent with the history---and testing whether that
ordering is compatible with the sequential datatype.  The search can complete
quite quickly in practice; for example, the tester in
Figure~\ref{fig:queue-lin-tester}, which performs 10,000 tests, with a total
of 8,000,000 operation executions, takes slightly more than two minutes to run
on my computer.

However, the search to linearize a particular history can be very large, and
so be time-consuming.  Indeed, the problem of deciding whether a given history
is linearizable is NP-complete in general, so there are bound to be some bad
cases.  It is therefore worth taking steps to limit the time taken by the
search.  A good pragmatic approach is to start with short runs---a small
number of threads performing a small number of operations each---and to move
to longer runs if the testing goes well.

Testing certain concurrent datatypes requires particular considerations.  In
the case of a queue, if two |enqueue|s happen concurrently, they could be
linearized in either order; the linearizability tester has to consider both
possibilities; which one is correct becomes apparent only when one of the
values is dequeued.  If the queue holds quite a lot of values, there might be
several pairs that were enqueued concurrently, and the linearizability tester
has to consider all possibilities.  This means that the search space can grow
exponentially with the length of the queue.  To avoid this, we should take
steps to make it unlikely that the queue holds too many values.  In
Figure~\ref{fig:queue-lin-tester}, we did this by making dequeues more
frequent than enqueues.

By default, the linearizability tester performs the logging of operation calls
and returns using thread-local logs, pairing each event with a timestamp.  We
saw a similar technique in an earlier chapter.  If you're using an operating
system that doesn't support timestamps property, setting the optional
parameter |tsLog| to |false| will use a different type of log; for example:
%
\begin{scala}
  val tester = LinearizabilityTester[SeqQueue, ConcQueue](
    seqQueue, concQueue, 4, worker, tsLog = false)
\end{scala}

It is worth reiterating that the sequential specification object must be
immutable: its state should not change.  Failing to respect this requirement
is a fairly common error.

It is beneficial to ensure that the specification object has a natural
definition for equality, and an associated hash code.  This is true for the
immutable |Queue|s we used here: two such objects are equal if they contain
the same sequence of data; and in this case their hash codes will be equal.
The same is true for other objects in the |scala.collection.immutable|
package.  The linearization checker uses a hash set to keep track of nodes of
the search graph it has seen previously, so as to avoid repeating work.  For
this to be effective, it needs to be able to identify when two sequential
specification objects represent the same state.

It is also worth reiterating that the strings used in logging semantically
different operations should be distinct.  My normal style is to use a string
that looks like the syntax of the operation call, including any parameters:
this is what is most useful in debugging output if an error of linearizability
is found.  But also these strings are used internally as an optimisation which
assumes such distinct strings (the tester records the effect of each
sequential operation on the specification object, using the string as a key;
this avoids repeating the same operation on the same specification object).

