\section{Testing the Resource Allocation Server}

We want to test the resource allocation server.  The property we want to
test is that two clients never hold the same resource: if a resource~$r$ is
allocated to client~$c_1$, it can't be allocated to another client~$c_2$ until
$c_1$ has returned it.

We can perform this test by running a number of clients that randomly request
and return resources, and log those actions.  We can then test whether the
resulting history of log events is valid.

%%% Note: we could use linearisation testing here; but there are a few
%%% wrinkles, so this probably isn't the best example for introducing the
%%% technique. 

%%%%%

We use a log of type |Log[LogEvent]|, where |LogEvent| is defined in
Figure~\ref{fig:RATest-1}.  
%% Subsequently, we can use the |get| method on the
%% log, to get the results of logging as an |Array[LogEvent]|.
%
A single client thread for testing is also defined in
Figure~\ref{fig:RATest-1}.   This uses a
queue |got| to keep track of the resources it currently holds.  On each
iteration it either tries to acquire another resource, or return a resource if
it has one.  In each case, it adds a suitable event to the log.
%%%%%

\begin{figure}
\begin{scala}
  trait LogEvent
  case class GotResource(c: ClientId, r: Resource) extends LogEvent
  case class ReturnedResource(c: ClientId, r: Resource) extends LogEvent

  /** A client */
  def client(me: ClientId, resourceServer: RAServer, log: Log[LogEvent]) = thread{
    var got = new scala.collection.mutable.Queue[Resource]()
    val random = new scala.util.Random
    for(_ <- 0 until iters){
      if(random.nextInt(2) == 0){ // Acquire new resource
	resourceServer.requestResource(me) match{
          case Some(r) =>  log.add(me, GotResource(me, r)); got.enqueue(r)
          case None => {}  // try again
        }
      }
      else if(!got.isEmpty){     // Return resource.
	val r = got.dequeue()
        log.add(me, ReturnedResource(me, r))
	resourceServer.returnResource(me, r)
      }
    }
  }
\end{scala}
\caption{Testing a resource allocation server (part~1).}
\label{fig:RATest-1}
\end{figure}

%%%%%

Note that we log acquiring resources \emph{after} the resource is acquired,
and log returning resources \emph{before} the resource is returned.  This
means that the period during which the log indicates that the thread holds a
resource is a \emph{subset} of the true time.
%
This is necessary to avoid false positives (i.e.~the testing signalling an
error, when in fact there is none).  If we were to return a resource before
logging the return, it is possible that a client would be slow in logging the
return, and that another client might have obtained the same resource in the
mean time, giving a false positive.

By avoiding false positives, we can run the test program unsupervised,
maybe for a long time, knowing that any error it signals is genuine.
%
There is a small risk of false negatives (i.e.~the testing failing to
detect an incorrect behaviour).  But we would expect these to be rare: if
there is a bug, doing enough testing will almost certainly also produce true
positives.


%%%%%

\begin{figure}
\begin{scala}
  /** Check that events represents a valid log.  */
  def checkLog(events: Array[LogEvent]): Boolean = {
    val held = Array.fill(numResources)(false); var error = false; var i = 0
    while(i < events.size && !error){
      events(i) match{
        case GotResource(_, r) =>
          if(held(r)){ // Error!
            println("Error found:\n"+events.take(i+1).mkString("\n"))
            error = true
          }
          else held(r) = true
        case ReturnedResource(_, r) => held(r) = false
      }
      i += 1
    }
    !error
  }

  /** Run a single test. */
  def runTest(resourceServer: RAServer) = {
    val log = new Log[LogEvent](numClients)
    run(|| (for (i <- 0 until numClients) yield client(i, resourceServer, log)))
    if(!checkLog(log.get)) sys.exit()
    resourceServer.shutdown()
  }
\end{scala}
\caption{Testing a resource allocation server (part~2).}
\label{fig:RATest-2}
\end{figure}

%%%%%

The function |runTest(resourceServer)| (Figure~\ref{fig:RATest-2}) performs a
single test on |resourceServer|.  This runs |numClients| clients, sharing a
log.  It uses the |checkLog| function (described below) to check whether the
log represents a correct execution.  It shuts down |resourceServer|, to
terminate the server thread: this is necessary when we perform many tests on
different |RAServer| objects; if we did not do this, we would end up with many
server threads still running, which would consume system resources.

The |checkLog| function traverses the log, keeping track (in array~|held|) of
which resources are currently held by threads.  If it finds that a resource
that is currently held is allocated a second time, this constitutes an error:
it prints the log up to that point to help with debugging.

The program's |main| function calls |runTest| many times. 

\begin{instruction}
Study the details of the test program.
\end{instruction}

%%%%%%%%%%%%%%%%%%%%%%%%%%%%%%%%%%%%%%%%%%%%%%%%%%%%%%% 

\section{Supporting Arbitrary Many Clients}

So far, we have assumed a fixed number of clients, with this number known when
the resource server is created.  This allowed us to create one reply channel
per client.  However, in many circumstances, this assumption doesn't hold.

If we do not know the number of clients, an alternative is for each client to
create a new reply channel for each request, and to send the reply channel
within the request.  The server can then reply on this reply channel.  
%
This technique is illustrated in Figure~\ref{fig:RAServer-replyChan} (eliding
code that is unchanged from earlier). 

%%%%%

\begin{figure}
\begin{scala}
  private type ReplyChan = Chan[Option[Resource]]

  private val acquireRequestChan = new SyncChan[ReplyChan]

  def requestResource(me: ClientId): Option[Resource] = {
    val replyChan = new OnePlaceBuffChan[Option[Resource]]
    acquireRequestChan!replyChan  // Send request.
    replyChan?() // Wait for response.
  }

  private def server = thread{
    ...
    serve(
      acquireRequestChan =?=> { replyChan => 
	var r = 0
	while(r < numResources && !free(r)) r += 1
	if(r == numResources) replyChan!None
        else{ free(r) = false; replyChan!Some(r) }
      }
      | ... // As previously. 
    )
  }
\end{scala}
\caption{The resource server, using reply channels.}
\label{fig:RAServer-replyChan}
\end{figure}

%%%%%

This implementation can be tested in the same way as the previous
implementation, since it provides the same interface and functionality: only
internal details have changed. 

\begin{instruction}
Study the details of the new implementation.
\end{instruction}

We will use reply channels like this in most subsequent client-server
examples. 

With this implementation, the client sends a request, and then stops and waits
for a response.  In some cases, the client could do some useful work while
waiting for the request to be serviced:
%
\begin{scala}
  <make request>
  <other useful work>
  <obtain response>
\end{scala}
%
We could change the interface of the object to support this, decoupling the
request for a resource from obtaining the resource.
The fact that we have used buffered reply channels means that the server can
send the response even if the client isn't yet ready for it.  

%%%%%%%%%%%%%%%%%%%%%%%%%%%%%%%%%%%%%%%%%%%%%%%%%%%%%%%

\section{Buffering Requests}

Previously, if the server couldn't meet a request, it immediately replied with
a |None| value.  An alternative is to store such requests until they can be
met.  
%
Figure~\ref{fig:totalRAServer} gives code outlining this approach.  We omit
code that is identical to earlier.

It makes sense to change the |requestResource| operation to return a result of
type |Resource| (rather than |Option[Resource]|), since it can never return
the |None| value.  The types of the channels are adapted accordingly, but
otherwise the client-side code is unchanged.

%%%%%

\begin{figure}
\begin{scala}
  private type ReplyChan = Chan[Resource]

  /* Channel for requesting a resource. */
  private val acquireRequestChan = new SyncChan[ReplyChan]

  /** Request a resource. */
  def requestResource(me: ClientId): Resource = {
    val replyChan = new OnePlaceBuffChan[Resource]
    acquireRequestChan!replyChan  // Send request.
    replyChan?() // Wait for response.
  }

  private def server = thread{
    // Record whether resource £i£ is available in £free(i)£.
    val free = Array.fill(numResources)(true)
    // Reply channels for requests that cannot be served immediately.
    val pending = new scala.collection.mutable.Queue[ReplyChan]
    // Invariant: if £pending£ is non-empty, then all entries in £free£ are false.
    var done = false

    serve(!done)(
      acquireRequestChan =?=> { replyChan =>
	var r = 0
	while(r < numResources && !free(r)) r += 1
	if(r == numResources) 
          pending.enqueue(replyChan) // Client has to wait.
        else{  // Pass resource r back to client. 
	  free(r) = false; replyChan!r
        }
      }
      | returnChan =?=> { r =>
          if(pending.nonEmpty)
            pending.dequeue()!r // Allocate £r£ to blocked client.
          else free(r) = true
      }
      | shutdownChan =?=> { _ => done = true }
    )
    acquireRequestChan.close(); returnChan.close(); shutdownChan.close()
  }
\end{scala}
\caption{A total resource server (code similar to earlier is omitted).}
\label{fig:totalRAServer}
\end{figure}

The server maintains a queue |pending| that holds the reply channels
corresponding to blocked |requestResource| operations.  This queue is nonempty
only if all entries in |free| are false: otherwise there would be a free
resource that could be allocated to the blocked operation.  When the server
receives a request for a resource, but there is no free resource, it adds the
reply channel to |pending|.  When a resource is returned, if |pending| is
nonempty, the server dequeues the first reply channel, and uses it to send the
resource to the corresponding channel.

\begin{instruction}
Study the details of the implementation.
\end{instruction}

With this implementation, the |requestResource| operation can take effect only
when there is an unallocated resource: we say that it is a \emph{partial
  operation}.  By contrast, in the previous implementations, the operation was
a \emph{total operation}, that could take effect in any state.  (The terms
``partial'' and ``total'' are by analogy with partial and total functions,
which are defined on some or all potential arguments, respectively.)

This approach can lead to a deadlock if all resources are allocated, and all
clients are requesting more.  There is no good solution to this problem:
sometimes there are simply insufficient resources available.  

The possibility of deadlock also has implications for testing.  We want a
testing harness that can run unsupervised, maybe for a long time, without
deadlocking.  To achieve this, we need to ensure that the clients never
request so many resources that the system deadlocks; but we do need to ensure
that the tests cover cases where all the resources are allocated.

My approach is to arrange that each client in the testing harness holds at
most two resources, and when it does hold two, to return one on the next
iteration.  In addition, I arrange that there are slightly more resources than
clients.  This means that if the server is unable to immediately satisfy a
request, and so that request is queued, then there must be some other client
that holds two resources, and so will return one on the next iteration.  With
such a definition for clients, the testing harness is almost identical to
previously. 

It is reasonable to assume that the server is able to keep up with clients,
i.e.,~that it is able to deal with each message faster, on average, than
clients attempt to send new messages.  This implies that when a client
attempts to send a message to the server, the server eventually receives it.
Put another way, we assume that clients don't make so many requests that the
server is overwhelmed.

The use of a queue then ensures a degree of fairness.  Assuming resources are
repeatedly returned, once a client attempts to obtain a resource, its message
on |acquireRequestChan| will eventually be received, so the reply channel will
be enqueued; and eventually the reply channel will reach the front of the
queue, and the request serviced.

%%%%%%%%%%%%%%%%%%%%%%%%%%%%%%%%%%%%%%%%%%%%%%%%%%%%%%%%%%%%


There is a simpler way to achieve the same effect of buffering requests that
cannot be served immediately, albeit it the cost of losing this fairness.
Rather than the server explicitly queueing pending requests, the clients can
be blocked on the |acquireRequestChan| channel until there is a resource free.
This can be achieved simply by adding a variable that keeps track of the
number of free resources, and a suitable guard on the |acquireRequestChan|
branch of the |serve|, to block communications when there are no free
resources.%
%
\begin{scala}
  private def server = thread{
    val free = Array.fill(numResources)(true)
    // Invariant: £numFree£ is the number of free resources, i.e. the number 
    // of £true£s in £free£.
    var numFree = numResources
    serve(
      numFree > 0 && acquireRequestChan =?=> { replyChan => 
	var r = 0; while(r < numResources && !free(r)) r += 1
	assert(r < numResources); free(r) = false; numFree -= 1
        replyChan!r
      }
      | returnChan =?=> { r => free(r) = true; numFree += 1 }
      | shutdownChan =?=> { ... } // As before.
    )
  }
\end{scala}
%
The assertion in the |acquireRequestChan| branch holds because of
the invariant concerning |numFree|.

When the server has no free resources, all client threads wanting to obtain a
resource will be blocked on |acquireRequestChan|.
%
When a resource is returned, those clients will compete to communicate on the
channel.  The one that is able to send first will obtain the resource.
%
This means that this version is probably less fair than the previous one: an SCL
channel is not guaranteed to be fair between the threads trying to use it.
Whether this is important depends on the use case.






